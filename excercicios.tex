%!TEX root = fundamentos.tex
\chapter*{Exerc\ih cios Resolvidos}
\addcontentsline{toc}{chapter}{Exerc\ih cios Resolvidos}

%%%%%%%%%%%%%%%%%%%%%%%%%%%%%%%%%%%%%%%%%%%%%%%%%%%%%%%%%%%%%%%%%%%%%%%%
%%%%%%%%%%%%%%%%%%%%%%%%% secao 1.2 %%%%%%%%%%%%%%%%%%%%%%%%%%%%%%%%%%%%
%%%%%%%%%%%%%%%%%%%%%%%%%%%%%%%%%%%%%%%%%%%%%%%%%%%%%%%%%%%%%%%%%%%%%%%%
\paragraph{Exerc\ih cios \ref{eounao}}
\begin{enumerate}[{\bf 1.}]
%excercicio1
\item Determine os valores verdade das seguintes proposi\coes 
\begin{enumerate}[a)]
\item $3\leq7$ e $4$ \'e um inteiro \ih mpar. 

{\bf{\it Resposta:} F}

\item $3\leq7$ ou $4$ \'e um inteiro \ih mpar. 

\item $2+1=3$ mas $4<4$. 

{\it Resposta:} {\bf F} (lembre que ``mas'' tem significado l\'ogico de ``e'')

\item $5$ \'e \ih mpar ou divis\ih vel por $4$. 

{\bf{\it Resposta:} V}

\item N\ao \'e verdade que $2+2=5$ e $5>7$.

\item N\ao \'e verdade que $2+2=5$ ou $5>7$. 

{\bf{\it Resposta:} V}

\item $3 \geq 3$. 

{\bf{\it Resposta:} V}
\end{enumerate}

%excercicio2
\item Suponha que representamos ``$7$ \'e um n\'umero par'' por \pp e ``$3+1=4$'' por \qq e ``$24$ \'e di\ih sivel por $8$'' por $r$. 
\begin{enumerate}[a)]
\item Escreva na forma s\ih mb\'olica e determine os valores verdade para:
\begin{enumerate}[i)]
\item $3+1 \neq 4$ e 24 \'e div\ih sivel por 8. 

{\bf{\it Resposta:} ${\bf \lnot q \ee r}$. F.}

\item N\ao \'e verdade que $7$ \'e \ih mpar ou $3+1=4$. 

{\bf{\it Resposta:} ${\bf\lnot (\lnot p \ou q)}$. F.}

\item $3+1=4$ mas 24 n\ao \'e div\ih sivel por 8.
\end{enumerate}

\item Escreva o que vem a seguir em palavras e determine os valores verdade para:
\begin{enumerate}[i)]
\item \pp $\ou$ $\nao$ $q$.

{\bf{\it Resposta:} ${\bf 7}$ \'e um n\'umero par ou ${\bf 3+1\neq 4}$. Falso.}

\item $\nao$ ($r$ $\ee$ $q$).

\item $\nao$ $r$ $\ou$ $\nao$ $q$.
\end{enumerate}
\end{enumerate}

%excercicio3
\item Construa a tabela verdade para:
\begin{enumerate}[a)]
\item $\nao$ $p$ $\ou$ $q$.

\item $\nao$ $p$ $\ee$ $q$.

\item ($\nao$ $p$ $\ou$ $q$) $\ee$ $r$.

\item\label{3d} $\nao$ ($p$ $\ee$ $q$).

{\bf{\it Resposta:}}
\begin{table}[h]
\centering
\begin{tabular}{|l c r|l c c c c r|}
\hline
\pp & & \qq &  &   $\nao$       & \pp  & $\ee$  & \qq  &      \\
\hline
V   & & V   &  &   {\bf F}      &  V   &   V    &  V   &       \\
V   & & F   &  &   {\bf V}      &  V   &   F    &  F   &       \\
F   & & V   &  &   {\bf V}      &  F   &   F    &  V   &      \\
F   & & F   &  &   {\bf V}      &  F   &   F    &  F   &     \\
\hline
\end{tabular}
\end{table}


\item $\nao$ $p$ $\ee$ $\nao$ $q$.

\item\label{3f}$\nao$ $p$ $\ou$ $\nao$ $q$.

{\bf{\it Resposta:}}
\begin{table}[h]
\centering
\begin{tabular}{|l c r|l c c c c c r|}
\hline
\pp & & \qq &  & $\nao$   & \pp  &   $\ou$       & $\nao$ & \qq  &     \\
\hline
V   & & V   &  &   F      &  V   &   {\bf V}    &   F     &  V   &     \\
V   & & F   &  &   F      &  V   &   {\bf V}    &   V     &  F   &     \\
F   & & V   &  &   V      &  F   &   {\bf V}    &   F     &  V   &    \\
F   & & F   &  &   V      &  F   &   {\bf V}    &   V     &  F   &   \\
\hline
\end{tabular}
\end{table}


\item $p$ $\ou$ $\nao$ $p$.

\item $\nao$ ($\nao$ $p$).
\end{enumerate}


%excercicio4
\item Construa nega\coes \'uteis para:
\begin{enumerate}[a)]
\item $3-4<7$. {\bf{\it Resposta:} $3-4 \geq 7$.}
\item $3+1=5$ e $2 \leq 4$.
\item $8$ \'e divis\ih vel por 3 mas $4$ n\ao \'e.
\end{enumerate}

%excercicio5
\item Suponha que definimos o conectivo $\star$ dizendo que $p \star q$ \'e verdade somente quando \qq \'e verdade e \pp \'e falso, e \'e falso caso contr\'ario.
\begin{enumerate}[a)]
\item Escreva a tabela verdade de $p \star q$. 

{\bf{\it Resposta:}}
\begin{table}[h]
\centering
\begin{tabular}{|l c r|c|}
\hline
\pp & & \qq & \pp $\star$ \qq \\
\hline
V   & & V   & F \\
V   & & F   & F \\
F   & & V   & V \\
F   & & F   & F \\
\hline
\end{tabular}
\end{table}

\newpage
\item Escreva a tabela verdade de $q \star p$.

{\bf{\it Resposta:}}
\begin{table}[h]
\centering
\begin{tabular}{|l c r|c|}
\hline
\pp & & \qq & \qq $\star$ \pp \\
\hline
V   & & V   & F \\
V   & & F   & V \\
F   & & V   & F \\
F   & & F   & F \\
\hline
\end{tabular}
\end{table}

\item Escreva a tabela verdade de $(p \star p) \star q$.

{\bf{\it Resposta:}}
\begin{table}[h]
\centering
\begin{tabular}{|l c r|c c c|}
\hline
\pp & & \qq & \pp $\star$ \pp & $\star$ & \qq \\
\hline
V   & & V   &        F        &  {\bf V}&  V   \\
V   & & F   &        F        &  {\bf F}&  F    \\
F   & & V   &        F        &  {\bf V}&  V    \\
F   & & F   &        F        &  {\bf F}&  F    \\
\hline
\end{tabular}
\end{table}
\end{enumerate}

%excercicio6
\item Vamos denotar o ``ou exclusivo'' \`as vezes utilizado nas conversas do dia a dia por $\oplus$. Portanto, $p\oplus q$ ser\'a verdade exatamente quando uma condi\cao de $p,q$ \'e verdade e falso caso contr\'ario.
\begin{enumerate}[a)]
\item Escreva a tabela verdade de $p \oplus q$.

{\bf{\it Resposta:}}
 \begin{table}[h]
\centering
\begin{tabular}{|l c r|c|}
\hline
\pp & & \qq & \pp $\oplus$ \qq \\
\hline
V   & & V   & F \\
V   & & F   & V \\
F   & & V   & V \\
F   & & F   & F \\
\hline
\end{tabular}
\end{table}

\item Escreva a tabela verdade de $p \oplus p$ e $(p \oplus q) \oplus q$.

{\bf{\it Resposta:}}
 \begin{table}[H]
\centering
\begin{tabular}{|l c r|c|}
\hline
\pp & & \pp & \pp $\oplus$ \pp \\
\hline
V   & & V   & F \\
F   & & F   & F \\
\hline
\end{tabular}
\end{table}

\begin{table}[H]
\centering
\begin{tabular}{|l c r|l c c c c c c c c c r|}
\hline
\pp & & \qq &    &   & (\pp & & $\oplus$ & & $q$) & & $\oplus$ &\qq &  \\
\hline
V & & V &  &    & V & & F & & V & & {\bf V} & V&  \\
V & & F &  &    & V & & V & & F & & {\bf V} & F&  \\
F & & V &  &    & F & & V & & V & & {\bf F} & V&  \\
F & & F &  &    & F & & F & & F & & {\bf F} & F&  \\
\hline
\end{tabular}
\end{table}

\item Mostre que ``e/ou'' realmente significa ``e ou ou'', isto \'e, a tabela verdade para $(p \ee q)\oplus(p \oplus q)$ \'e a mesma tabela verdade que $(p \ou q)$.

{\bf{\it Resposta:}}
\begin{table}[h]
\centering
\begin{tabular}{|l c r|l c c |c c c c c c c  r|}
\hline
\pp & & \qq &  &$p \ou q$ &  & &($p \ee q$)&  &$\oplus$&  & $(p \oplus q)$&   &     \\
\hline
V   & & V   &  &   {\bf V}      &  & &    V      &  &    {\bf V}    &  &       F         &   &     \\
V   & & F   &  &   {\bf V}      &  & &    F      &  &    {\bf V}    &  &       V         &   &     \\
F   & & V   &  &   {\bf V}      &  & &    F      &  &    {\bf V}    &  &       V         &   &    \\
F   & & F   &  &   {\bf F}      &  & &    F      &  &    {\bf F}    &  &       F         &   &   \\
\hline
\end{tabular}
\end{table}

\item Mostre que n\ao faz diferen\cc a se tomamos o ``ou'' em ``e/ou'' como sendo inclusivo ($\ou$) ou exclusivo ($\oplus$).

{\bf{\it Resposta:}}
\begin{table}[H]
\centering
\begin{tabular}{|l c r|l c c c c c| c c c c c c|}
\hline
\pp & & \qq &  & ($p \ee q$) &  &$\oplus$ & & $(p \oplus q)$ & & ($p \ee q$)  & & $\ou$  & & $(p \ou q)$     \\
\hline
V   & & V   &  &    V        &  &   {\bf V}     & &      F         & &    V         & &    {\bf V}   & &     V            \\
V   & & F   &  &    F        &  &   {\bf V}     & &      V         & &    F         & &    {\bf V}   & &     V            \\
F   & & V   &  &    F        &  &   {\bf V}     & &      V         & &    F         & &    {\bf V}   & &     V            \\
F   & & F   &  &    F        &  &   {\bf F}     & &      F         & &    F         & &    {\bf F}   & &     F            \\
\hline
\end{tabular}
\end{table}
\end{enumerate}

%excercicio7
\item Explique a seguinte piada: Ansioso, o pai pergunta ao parteiro: ``Doutor, \'e homem ou mulher?'' O m\'edico responde: ``Sim.''
{\bf{\it Resposta:} Sejam \pp a proposi\cao ``\'e homem'' e \qq a proposi\cao ``\'e mulher''. Portanto, a pergunta feita pelo pai \'e $p\ou q$? Analisemos o caso, \pp \'e verdade, ent\ao \qq \'e falsa, neste caso $p\ou q$ \'e verdade. Por outro lado, se \pp \'e falsa, \qq \'e verdade, portanto $p\ou q$ \'e verdade. Por isso o m\'edico responde sim. }



\end{enumerate}

%%%%%%%%%%%%%%%%%%%%%%%%%%%%%%%%%%%%%%%%%%%%%%%%%%%%%%%%%%%%%%%%%%%%%%%%
%%%%%%%%%%%%%%%%%%%%%%%%% secao 1.3 %%%%%%%%%%%%%%%%%%%%%%%%%%%%%%%%%%%%
%%%%%%%%%%%%%%%%%%%%%%%%%%%%%%%%%%%%%%%%%%%%%%%%%%%%%%%%%%%%%%%%%%%%%%%%
\paragraph{Exerc\ih cios \ref{implicacao}}

\begin{enumerate}[{\bf 1.}]
%excercicio1
\item Quais das seguintes proposi\coes s\ao logicamente equivalentes?
\begin{enumerate}[a)]
\item $p \ee \lnot q$.
\item $p \to q$.
\item $\lnot(\lnot p \ou q)$.
\item $q \to \lnot p$.
\item $\nao p \ou q$.
\item $\lnot (p \to q)$.
\item $p \to \nao q$.
\item $\nao p \to \nao q$.
{\bf{\it Resposta:} a), c) e f) s\ao logicamente equivalentes, b) e e) s\ao l\'ogicamente equivalente, d) e g) s\ao logicamente equivalentes e h) n\ao \'e l\'ogicamente equivalente a nenhum outro.}
\end{enumerate}

%excercicio2
\item Mostre que os seguintes pares s\ao logicamente equivalentes:
\begin{enumerate}[a)]
\item $p\ee(q\ou r)$; $(p\ee q)\ou(p\ee r)$ .
\item $p\ou(q\ee r)$; $(p\ou q)\ee(p\ou r)$.
\item $p\leftrightarrow q$; $(p \to q)\ee(q \to p)$.
\item $p \to q$; $\lnot q \to \lnot p$.
\end{enumerate}

%excercicio3
\item Mostre que os seguintes pares n\ao s\ao logicamente equivalentes:
\begin{enumerate}[a)]
\item $\nao(p \ee q)$; $\nao p \ee \nao q$. 

{\bf{\it Resposta:} Usa-se as tabelas verdade ou, apenas, observe que quando ${\bf p}$ \'e V e ${\bf q}$ \'e F ent\ao ${\bf \nao(p \ee q)}$ \'e V enquanto ${\bf \nao p \ee \nao q}$ \'e F.}

\item $\nao(p \ou q)$; $\nao p \ou \nao q$. 

{\bf{\it Resposta:} Usa-se as tabelas verdade ou, apenas, observe que quando ${\bf p}$ \'e V e ${\bf q}$ \'e F ent\ao ${\bf \nao(p \ou q)}$ \'e F enquanto ${\bf \nao p \ou \nao p}$ \'e V.}

\item $p \to q$; $q \to p$.

{\bf{\it Resposta:} Usa-se as tabelas verdade ou, apenas, observe que quando ${\bf p}$ \'e V e ${\bf q}$ \'e F ent\ao ${\bf (p\to q)}$ \'e F enquanto ${\bf q\to p}$ \'e V.}

\item $\lnot (p \to q)$; $\lnot p \to \lnot q$.

{\bf{\it Resposta:} Usa-se as tabelas verdade ou, apenas, observe que quando ${\bf p}$ \'e V e ${\bf q}$ \'e V ent\ao ${\bf \lnot (p \to q)}$ \'e F enquanto ${\bf \lnot p \to \lnot q}$ \'e V.}
\end{enumerate}

%excercicio4
\item Determine:
\begin{enumerate}[a)]
\item A contrapositiva de $\lnot p\to q$. 

{\bf{\it Resposta:} ${\bf \nao q \to p}$.}

\item A rec\ih proca de $\lnot q \to p$.

{\bf{\it Resposta:} ${\bf p \to \nao q}$.}

\item O inverso do contr\'ario de $q \to \lnot p$.

\item A nega\cao de $p \to \lnot q$. 

{\bf{\it Resposta:} ${\bf \nao (p \to\nao q)}$.}

\item A rec\ih proca de $\lnot p \ee q$.
\end{enumerate}

%excercicio5
\item Indique quais das proposi\coes a seguir s\ao verdadeiras:
\begin{enumerate}[a)]
\item Se $2+1=4$ ent\ao $3+2=5$. {\bf{\it Resposta:} V}
\item Vermelho \'e branco se, e somente se, verde \'e azul. {\bf{\it Resposta:} V}
\item $2+1=3$ e $3+1=5$ implicam que $4$ \'e \ih mpar. {\bf{\it Resposta:} V}
\item Se $4$ \'e \ih mpar ent\ao $5$ \'e \ih mpar. {\bf{\it Resposta:} V}
\item Se $4$ \'e \ih mpar ent\ao $5$ \'e par. {\bf{\it Resposta:} V}
\item Se $5$ \'e \ih mpar ent\ao $4$ \'e \ih mpar.  {\bf{\it Resposta:} F}
\end{enumerate}

%excercicio6
\item Explique ou d\^e exemplos porque nenhum dos exemplos a seguir existem:
\begin{enumerate}[a)]
\item Uma implica\cao verdadeira com uma conclus\ao falsa.{\bf{\it Resposta:} Se $2+3=5$ ent\ao $4<2$.}
\item Uma implica\cao verdadeira com uma conclus\ao verdadeira.
\item Uma implica\cao falsa com uma conclus\ao verdadeira.
\item Uma implica\cao falsa com uma conclus\ao falsa.
\item Uma implica\cao falsa com uma hip\'otese falsa.
\item Uma implica\cao falsa com uma hip\'otese verdadeira.
\item Uma implica\cao verdadeira com uma hip\'otese verdadeira.
\item Uma implica\cao verdadeira com uma hip\'otese falsa. 
\end{enumerate}

%excercicio7
\item Traduza em s\ih mbolos:
\begin{enumerate}[a)]
\item \pp sempre que $q$.
\item \pp a menos que $q$.
\end{enumerate}

%excercicio8
\item D\^e a nega\cao para $p \leftrightarrow q$ na forma que n\ao envolva uma bicondicional.

{\bf{\it Resposta:} ${\bf (p\leftrightarrow q)\Leftrightarrow [(p\to q)\ee(q\to p)]}$. Mas ${\bf (p\to q)\Leftrightarrow \nao(p \ee \nao q)\Leftrightarrow(\nao p\ou q)}$. Assim, ${\bf (p\leftrightarrow q)\Leftrightarrow[(\nao p\ou q)\ee(\nao q\ou p)]}$. Logo, ${\bf (\nao p\leftrightarrow\nao q)\Leftrightarrow[(p\ou \nao q)\ee(q\ou \nao p)]}$.}

%excercicio9
\item Suponha que $p$, $\lnot q$ e $r$ s\ao verdade. Quais a seguir s\ao proposi\coes verdadeiras?
\begin{enumerate}[a)]
\item $p \to q$. {\bf{\it Resposta:} F}
\item $q \to p$. {\bf{\it Resposta:} V}
\item $p \to (q \ou r)$. {\bf{\it Resposta:} V}
\item $p \leftrightarrow q$. {\bf{\it Resposta:} F}
\item $p \leftrightarrow r$. {\bf{\it Resposta:} V}
\item $(p \ou q) \to p$. {\bf{\it Resposta:} V}
\item $(p \ee q) \to q$. {\bf{\it Resposta:} V}
\end{enumerate}

%excercicio10
\item Note que temos cinco ``conectivos'' l\'ogicos: $\ee$, $\ou$, $\to$, $\leftrightarrow$ e $\nao$, cada qual corresponde a uma constru\cao da linguagem comum. Do ponto de vista l\'ogico isto \'e de alguma forma um deperd\ih cio, desde que podemos expressar todos estes em termos de, apenas, $\nao$ e $\ee$. Ainda mais, se definirmos $p|q$ para ser falsa quando ambos \pp e \qq s\ao verdadeiros, e verdadeiro caso contr\'ario, podemos expressar todas as cinco formas em termos deste \'unico conectivo ($|$ \'e conhecido como Conectivo de Sheffer ou Conectivo Nou). Verifique parcialmente que os argumentos dados acima por
\begin{enumerate}[a)]
\item Econtrando a proposi\cao a qual equivale a $p \ou q$ usando apenas $\ee$ e $\nao$.

{\bf{\it Resposta:} ${\bf (p\ou q) \Leftrightarrow \nao(\nao p \ee \nao q)}$.}

\item Escrevendo a tabela verdade para $p|q$.

{\bf{\it Resposta:}}
 \begin{table}[h]
\centering
\begin{tabular}{|l c r|c|}
\hline
\pp & & \qq & \pp $|$ \qq \\
\hline
V   & & V   & F \\
V   & & F   & V \\
F   & & V   & V \\
F   & & F   & V \\
\hline
\end{tabular}
\end{table}

\item Mostrando que $p|p$ \'e logicamente equivalente a $\nao p$.

{\bf{\it Resposta:}}
 \begin{table}[H]
\centering
\begin{tabular}{|l|c|c |}
\hline
\pp       & \pp $|$ \pp   &   $\nao p$  \\
\hline
V         &   F     &     F       \\
F         &   V     &     V       \\
\hline
\end{tabular}
\end{table}

\item Mostrando que $(p|q)|(q|p)$ \'e logicamente equivalente a $p \ee q$.

{\bf{\it Resposta:}}
 \begin{table}[H]
\centering
\begin{tabular}{|l c r| c c c c c c c| c|}
\hline
\pp & & \qq & (\pp & $|$ & \qq) &    $|$    & (\qq & $|$ & \pp) & $p\ee q$ \\
\hline
V   & & V   &  V  &  F  &  V  &  {\bf V}  &  V  &  F  &  V  &  {\bf V} \\ 
V   & & F   &  V  &  V  &  F  &  {\bf F}  &  F  &  V  &  V  &  {\bf F} \\
F   & & V   &  F  &  V  &  V  &  {\bf F}  &  V  &  V  &  F  &  {\bf F} \\
F   & & F   &  F  &  V  &  F  &  {\bf F}  &  F  &  V  &  F  &  {\bf F} \\
\hline
\end{tabular}
\end{table}
\end{enumerate}

%excercicio11
\item Escreva a rec\ih proca, a nega\cao e a contrapositiva das seguintes afirma\cois:
\begin{enumerate}[a)]
\item C\ao que ladra n\ao morde.
\item Nem tudo que reluz \'e ouro.
\item O que n\ao mata engorda.
\item Quem n\ao tem c\ao ca\cc a com gato.
\item Em boca fechada n\ao entra mosca.
\item Onde h\'a fuma\cc a, h\'a fogo.
\end{enumerate}
\end{enumerate}

%%%%%%%%%%%%%%%%%%%%%%%%%%%%%%%%%%%%%%%%%%%%%%%%%%%%%%%%%%%%%%%%%%%%%%%%
%%%%%%%%%%%%%%%%%%%%%%%%% secao 1.4 %%%%%%%%%%%%%%%%%%%%%%%%%%%%%%%%%%%%
%%%%%%%%%%%%%%%%%%%%%%%%%%%%%%%%%%%%%%%%%%%%%%%%%%%%%%%%%%%%%%%%%%%%%%%%
\paragraph{Exerc\ih cios \ref{tautologias}}

\begin{enumerate}[{\bf 1.}]
%excercicio1
\item Verifique que 7 a), 9 b), 13 e 14 da lista acima s\ao tautologias.

{\bf{\it Resposta: 7 a) da lista}}
\begin{table}[H]
\centering
\begin{tabular}{|l c c c r|c c c c c c c|}
\hline
\pp & & \qq &  & \rr & $[(p$ & $\ou$ &   $(q\ou r)]$   & $\leftrightarrow$   & $[(p\ou q)$ & $\ou$ &  $r)]$   \\
\hline
V   & &  V  &  &  V  &   V   &   V   &        V        &   {\bf V}           &      V      &    V  &    V     \\
V   & &  V  &  &  F  &   V   &   V   &        V        &   {\bf V}           &      V      &    V  &    F      \\
V   & &  F  &  &  V  &   V   &   V   &        V        &   {\bf V}           &      V      &    V  &    V      \\
V   & &  F  &  &  F  &   V   &   V   &        F        &   {\bf V}           &      V      &    V  &    F      \\
F   & &  V  &  &  V  &   F   &   V   &        V        &   {\bf V}           &      V      &    V  &    V      \\
F   & &  V  &  &  F  &   F   &   V   &        V        &   {\bf V}           &      V      &    V  &    F      \\
F   & &  F  &  &  V  &   F   &   V   &        V        &   {\bf V}           &      F      &    V  &    V      \\
F   & &  F  &  &  F  &   F   &   F   &        F        &   {\bf V}           &      F      &    F  &    F      \\
\hline
\end{tabular}
\end{table}


{\bf{\it Resposta: 9 b) da lista}}
 \begin{table}[H]
\centering
\begin{tabular}{|c| c c c|}
\hline
\pp & $(p\ee {\bf c})$ &    $\leftrightarrow$    & ${\bf c}$  \\
\hline
V   &       F        &        {\bf V}          &   F   \\
F   &       F        &        {\bf V}          &   F   \\
\hline
\end{tabular}
\end{table}


{\bf{\it Resposta: 13 da lista}}
 \begin{table}[H]
\centering
\begin{tabular}{|l c r| c c c c c|}
\hline
\pp & & \qq & $p\to q$ &    $\leftrightarrow$    & ($\nao$\qq & $\to$ & $\nao$\pp) \\
\hline
V   & & V   &    V     &        {\bf V}          &   F  &    V  &      F \\
V   & & F   &    F     &        {\bf V}          &   V  &    F  &      F \\
F   & & V   &    V     &        {\bf V}          &   F  &    V  &      V \\
F   & & F   &    V     &        {\bf V}          &   V  &    V  &      V  \\
\hline
\end{tabular}
\end{table}

{\bf{\it Resposta: 14 da lista}}
 \begin{table}[H]
\centering
\begin{tabular}{|l c r| c c c c c c c|}
\hline
\pp & & \qq & $p\to q$ &    $\leftrightarrow$    & (\pp & $\ee$ & $\nao$\qq & $\to$ & ${\bf c})$ \\
\hline
V   & & V   &    V     &        {\bf V}          &   V  &    F  &      F     &   V   &  F         \\
V   & & F   &    F     &        {\bf V}          &   V  &    V  &      V     &   F   &  F         \\
F   & & V   &    V     &        {\bf V}          &   F  &    F  &      F     &   V   &  F         \\
F   & & F   &    V     &        {\bf V}          &   F  &    F  &      V     &   V   &  F         \\
\hline
\end{tabular}
\end{table}


%excercicio2
\item Determine quais das seguintes proposi\coes t\^em alguma forma presente na lista de tautologias (por exemplo, $(\nao q\ee p)\to\nao q$ tem a forma 18 da lista) e nestes casos, indique qual forma:
\begin{enumerate}[a)]
\item $\nao q\to(\nao q \ou\nao p)$. {\bf{\it Resposta:} 18 (simplifica\caoi)}
\item $q\to (q\ee\nao p)$. {\bf{\it Resposta:} Esta n\ao \'e uma tautologia, portanto n\ao est\'a na lista.}
\item $(r\to\nao p)\leftrightarrow(\nao r\ou\nao p)$. {\bf{\it Resposta:} 12a (implica\caoi)}
\item $(p\to\nao q)\leftrightarrow\nao(\nao p\to q)$. {\bf{\it Resposta:} Esta n\ao \'e uma tautologia, portanto n\ao est\'a na lista.}
\item $(\nao r\to q)\leftrightarrow(\nao q\to r)$. {\bf{\it Resposta:} 13 (contrapositiva)}
\item $(p\to(\nao r\ou q))\leftrightarrow((r\ee\nao q)\to\nao p)$. {\bf{\it Resposta:} 13 (contrapositiva)}
\item $r\to\nao(q\ee\nao r)$.   {\bf{\it Resposta:} 17 (adi\caoi)}
\item $(\nao q\ou p)\ee q)\to p$.  {\bf{\it Resposta:} 22 (silogismo disjuntivo)}
\end{enumerate}

%excercicio3
\item D\^e exemplos ou diga porque as proposi\coes a seguir n\ao existem:
\begin{enumerate}[a)]
\item Uma implica\cao l\'ogica com uma falsa conclus\aoi. 

{\bf{\it Resposta:} Se $2<1$ e $3=2+1$ ent\ao $2<1$. Esta implica\caoi, ${\bf (p\ee q)\to p}$, \'e uma implica\cao l\'ogica:}
 \begin{table}[H]
\centering
\begin{tabular}{|l c r|c c c|}
\hline
\pp & & \qq & $p\ee q$ &   $\to$   & \pp \\
\hline
V   & & V   &    V     &  {\bf V}  &  V  \\
V   & & F   &    F     &  {\bf V}  &  V  \\
F   & & V   &    F     &  {\bf V}  &  F  \\
F   & & F   &    F     &  {\bf V}  &  F  \\
\hline
\end{tabular}
\end{table}

\item Uma implica\cao l\'ogica com uma conclus\ao verdadeira.

\item Uma implica\cao l\'ogica com uma hip\'otese verdadeira e uma conclus\ao falsa.
\end{enumerate}

%excercicio4
\item Quais das seguintes s\ao corretas?
\begin{enumerate}[a)]
\item $(p\to(q\ou r))\Rightarrow(p\to q)$.  {\bf{\it Resposta:} N\ao \'e correto. Pode-se provar o contr\'ario com as tabelas verdade ou observando que quando ${\bf p}$ \'e V, ${\bf q}$ \'e F e ${\bf r}$ \'e V ent\ao ${\bf (p\to(q\ou r))}$ \'e V, mas ${\bf (p\to q)}$ \'e F.}


\item $((p\ou q)\to r)\Rightarrow(p\to r)$.
{\bf{\it Resposta:} Correto:}
\begin{table}[H]
\centering
\begin{tabular}{|l c c c r|c c c c c|}
\hline
\pp & & \qq &  & \rr & $[(p\ou q$ & $\to$ &       $r]$      &    $\to$            & $(p\to r)$   \\
\hline
V   & &  V  &  &  V  &   V        &   V   &        V        &   {\bf V}           &      V           \\
V   & &  V  &  &  F  &   V        &   F   &        F        &   {\bf V}           &      F           \\
V   & &  F  &  &  V  &   V        &   V   &        V        &   {\bf V}           &      V            \\
V   & &  F  &  &  F  &   V        &   F   &        F        &   {\bf V}           &      F            \\
F   & &  V  &  &  V  &   V        &   V   &        V        &   {\bf V}           &      V            \\
F   & &  V  &  &  F  &   V        &   F   &        F        &   {\bf V}           &      V            \\
F   & &  F  &  &  V  &   F        &   V   &        V        &   {\bf V}           &      V            \\
F   & &  F  &  &  F  &   F        &   V   &        F        &   {\bf V}           &      V            \\
\hline
\end{tabular}
\end{table}

\item $(p\ou(p\ee q))\iff p$.
{\bf{\it Resposta:} Correto:}
 \begin{table}[H]
\centering
\begin{tabular}{|l c r|c c c c c|}
\hline
\pp & & \qq & [\pp & $\ou$ & $(p\ee q)]$ & $\leftrightarrow$ & \pp \\
\hline
V   & & V   &   V  &   V   &    V        &      {\bf V}     &  V  \\
V   & & F   &   V  &   V   &    F        &      {\bf V}     &  V  \\
F   & & V   &   F  &   F   &    F        &      {\bf V}     &  F  \\
F   & & F   &   F  &   F   &    F        &      {\bf V}     &  F  \\
\hline
\end{tabular}
\end{table}


\item $((p\to q)\ee\nao p)\Rightarrow\nao q$. {\bf{\it Resposta:} N\ao \'e correto. Pode-se provar o contr\'ario com as tabelas verdade ou observando que quando ${\bf p}$ \'e F e ${\bf q}$ \'e V ent\ao ${\bf (p\to q)\ee\nao p}$ \'e V, mas ${\bf \nao q}$ \'e F.}
\end{enumerate}

%excercicio5
\item Quais das seguintes s\ao tautologias, contradi\coes ou nenhuma das duas?
\begin{enumerate}[a)]
\item $(p\ee\nao q)\to(q \ou \nao p)$. {\bf{\it Resposta:} Nenhuma das duas}
 \begin{table}[H]
\centering
\begin{tabular}{|l c r|c c c c c c c|}
\hline
\pp & & \qq & (\pp & $\ee$ & $\nao q)$ & $\rightarrow$     & (\qq  & $\ou$  & $\nao p$) \\
\hline
V   & & V   &   V  &   F   &    F        &      {\bf V}     &  V   &    V   &     F    \\
V   & & F   &   V  &   V   &    V        &      {\bf F}     &  F   &    F   &     F      \\
F   & & V   &   F  &   F   &    F        &      {\bf V}     &  V   &    V   &     V      \\
F   & & F   &   F  &   F   &    V        &      {\bf V}     &  F   &    V   &     V      \\
\hline
\end{tabular}
\end{table}

\item $\nao p \to p$. {\bf{\it Resposta:} Nenhuma das duas}
 \begin{table}[H]
\centering
\begin{tabular}{|l|c c c|}
\hline
\pp       & $\nao$ \pp   & $\to$       &  \pp  \\
\hline
V         &   F          &   {\bf V}   &  V  \\
F         &   V          &   {\bf F}   &  F  \\
\hline
\end{tabular}
\end{table}

\item $\nao p \leftrightarrow p$. {\bf{\it Resposta:} Contradi\cao}
 \begin{table}[H]
\centering
\begin{tabular}{|l|c c c|}
\hline
\pp       & $\nao$ \pp   & $\leftrightarrow$       &  \pp  \\
\hline
V         &   F          &   {\bf F}   &  V  \\
F         &   V          &   {\bf F}   &  F  \\
\hline
\end{tabular}
\end{table}

\item $(p\ee\nao p)\to p$. {\bf{\it Resposta:} Tautologia}
 \begin{table}[H]
\centering
\begin{tabular}{|l|c c c c c|}
\hline
\pp       & (\pp & $\ee$ & $\nao$\pp)   & $\to$       &  \pp  \\
\hline
V         &   V  &   F   &   F          &  {\bf V}    &   V    \\
F         &   F  &   F   &   V          &  {\bf V}    &   F    \\
\hline
\end{tabular}
\end{table}

\item $(p\ee\nao p)\to q$. {\bf{\it Resposta:} Tautologia}
 \begin{table}[H]
\centering
\begin{tabular}{|l c r|c c c c c|}
\hline
\pp & & \qq & (\pp & $\ee$ & $\nao$\pp)   & $\to$       &  \qq  \\
\hline
V   & &  V  &   V  &   F   &   F          &  {\bf V}    &   V    \\
V   & &  F  &   V  &   F   &   F          &  {\bf V}    &   F    \\
F   & &  V  &   F  &   F   &   V          &  {\bf V}    &   V    \\
F   & &  F  &   F  &   F   &   V          &  {\bf V}    &   F    \\
\hline
\end{tabular}
\end{table}


\item $(p\ee\nao q)\leftrightarrow(p\to q)$. {\bf{\it Resposta:} Contradi\cao}
 \begin{table}[H]
\centering
\begin{tabular}{|l c r|c c c c c|}
\hline
\pp & & \qq & (\pp & $\ee$ & $\nao$\qq)   & $\leftrightarrow$       &  $(p\to q)$  \\
\hline
V   & &  V  &   V  &   F   &   F          &  {\bf F}    &   V    \\
V   & &  F  &   V  &   V   &   V          &  {\bf F}    &   F    \\
F   & &  V  &   F  &   F   &   F          &  {\bf F}    &   V    \\
F   & &  F  &   F  &   F   &   V          &  {\bf F}    &   V    \\
\hline
\end{tabular}
\end{table}


\item $[(p\to q) \leftrightarrow r]\leftrightarrow[p\to(q\leftrightarrow r)]$. {\bf{\it Resposta: } Nenhuma das duas}
\begin{table}[H]
\centering
\begin{tabular}{|l c c c r|c c c c c c c|}
\hline
\pp & & \qq &  & \rr & $[(p\to q)$ & $\leftrightarrow$ &   $r]$   & $\leftrightarrow$   & $[p$ & $\to$ &  $(q\leftrightarrow r)]$   \\
\hline
V   & &  V  &  &  V  &   V   &   V   &        V        &   {\bf V}           &      V      &    V  &    V     \\
V   & &  V  &  &  F  &   V   &   F   &        F        &   {\bf V}           &      V      &    F  &    F      \\
V   & &  F  &  &  V  &   F   &   F   &        V        &   {\bf V}           &      V      &    F  &    F      \\
V   & &  F  &  &  F  &   F   &   V   &        F        &   {\bf V}           &      V      &    V  &    V      \\
F   & &  V  &  &  V  &   V   &   V   &        V        &   {\bf V}           &      F      &    V  &    V      \\
F   & &  V  &  &  F  &   V   &   F   &        F        &   {\bf F}           &      F      &    V  &    F      \\
F   & &  F  &  &  V  &   V   &   V   &        V        &   {\bf V}           &      F      &    V  &    F      \\
F   & &  F  &  &  F  &   V   &   F   &        F        &   {\bf F}           &      F      &    V  &    V      \\
\hline
\end{tabular}
\end{table}
\end{enumerate}

%excercicio6
\item Quais dos seguintes s\ao corretos?
\begin{enumerate}[a)]
\item $(p\leftrightarrow q)\Rightarrow(p\to q)$. {\bf{\it Resposta:} Correto, pois ${\bf (p\leftrightarrow q)\to(p\to q)}$ \'e uma tautologia, vide a tabela verdade abaixo:}

\begin{table}[H]
\centering
\begin{tabular}{|l c r|c c c|}
\hline
\pp & & \qq & $(p \leftrightarrow q)$ &   $\to$   & $(p\to q)$ \\
\hline
V   & &  V  &           V             &  {\bf V}  &     V      \\
V   & &  F  &           F             &  {\bf V}  &     F      \\
F   & &  V  &           F             &  {\bf V}  &     V      \\
F   & &  F  &           V             &  {\bf V}  &     V      \\
\hline
\end{tabular}
\end{table}


\item $(p\to q)\Rightarrow(p\leftrightarrow q)$. {\bf{\it Resposta:} Incorreto, pois ${\bf (p\to q)\to(p\leftrightarrow q)}$ n\~ao  \'e uma tautologia, vide a tabela verdade abaixo:}

\begin{table}[H]
\centering
\begin{tabular}{|l c r|c c c|}
\hline
\pp & & \qq & $(p \to q)$ &   $\to$   & $(p \leftrightarrow q)$ \\
\hline
V   & &  V  &    V        &  {\bf V}  &           V      \\
V   & &  F  &    F        &  {\bf V}  &           F      \\
F   & &  V  &    V        &  {\bf F}  &           F      \\
F   & &  F  &    V        &  {\bf V}  &           V      \\
\hline
\end{tabular}
\end{table}


\item $(p\to q)\Rightarrow q$. {\bf{\it Resposta:} Incorreto, pois ${\bf (p\to q)\to q}$ n\~ao  \'e uma tautologia, vide a tabela verdade abaixo:}

\begin{table}[H]
\centering
\begin{tabular}{|l c r|c c c|}
\hline
\pp & & \qq & $(p \to q)$ &   $\to$   & $q$ \\
\hline
V   & &  V  &    V        &  {\bf V}  &  V      \\
V   & &  F  &    F        &  {\bf V}  &  F      \\
F   & &  V  &    V        &  {\bf V}  &  V      \\
F   & &  F  &    V        &  {\bf F}  &  F      \\
\hline
\end{tabular}
\end{table}
\end{enumerate}

%excercicio7
\item $\to$ \'e associoativa? Isto \'e $((p \to q) \to r)\iff((p \to (q \to r))$.

{\bf{\it Resposta: } $\to$ n\ao \'e associativa, pois}
\begin{table}[H]
\centering
\begin{tabular}{|l c c c r|c c c c c c c|}
\hline
\pp & & \qq &  & \rr & $[(p\to q)$ & $\to$ &   $r]$   & $\leftrightarrow$   & $[p$ & $\to$ &  $(q\to r)]$   \\
\hline
V   & &  V  &  &  V  &   V   &   V   &        V        &   {\bf V}           &      V      &    V  &    V     \\
V   & &  V  &  &  F  &   V   &   F   &        F        &   {\bf V}           &      V      &    F  &    F      \\
V   & &  F  &  &  V  &   F   &   V   &        V        &   {\bf V}           &      V      &    V  &    V      \\
V   & &  F  &  &  F  &   F   &   V   &        F        &   {\bf V}           &      V      &    V  &    V      \\
F   & &  V  &  &  V  &   V   &   V   &        V        &   {\bf V}           &      F      &    V  &    V      \\
F   & &  V  &  &  F  &   V   &   F   &        F        &   {\bf F}           &      F      &    V  &    F      \\
F   & &  F  &  &  V  &   V   &   V   &        V        &   {\bf V}           &      F      &    V  &    V      \\
F   & &  F  &  &  F  &   V   &   F   &        F        &   {\bf F}           &      F      &    V  &    V      \\
\hline
\end{tabular}
\end{table}


%excercicio8
\item $\leftrightarrow$ \'e associoativa? Isto \'e $((p \leftrightarrow q) \leftrightarrow r)\iff((p \leftrightarrow (q \leftrightarrow r))$.

{\bf{\it Resposta: } $\leftrightarrow$ \'e associativa, pois}
\begin{table}[H]
\centering
\begin{tabular}{|l c c c r|c c c c c c c|}
\hline
\pp & & \qq &  & \rr & $[(p\leftrightarrow q)$ & $\leftrightarrow$ &   $r]$   & $\leftrightarrow$   & $[p$ & $\leftrightarrow$ &  $(q\leftrightarrow r)]$   \\
\hline
V   & &  V  &  &  V  &   V   &   V   &        V        &   {\bf V}           &      V      &    V  &    V     \\
V   & &  V  &  &  F  &   V   &   F   &        F        &   {\bf V}           &      V      &    F  &    F      \\
V   & &  F  &  &  V  &   F   &   F   &        V        &   {\bf V}           &      V      &    F  &    F      \\
V   & &  F  &  &  F  &   F   &   V   &        F        &   {\bf V}           &      V      &    V  &    V      \\
F   & &  V  &  &  V  &   F   &   F   &        V        &   {\bf V}           &      F      &    F  &    V      \\
F   & &  V  &  &  F  &   F   &   V   &        F        &   {\bf V}           &      F      &    V  &    F      \\
F   & &  F  &  &  V  &   V   &   V   &        V        &   {\bf V}           &      F      &    V  &    F      \\
F   & &  F  &  &  F  &   V   &   F   &        F        &   {\bf V}           &      F      &    F  &    V      \\
\hline
\end{tabular}
\end{table}

%excercicio9
\item Quais das seguintes proposi\coes verdadeiras s\ao tautologias?
\begin{enumerate}[a)]
\item Se $2+2=4$ ent\ao $5$ \'e \ih mpar. {\bf{\it Resposta:} N\ao \'e uma tautologia}
\item $3+1=4$ e $5+3=8$ implica $3+1=4$. {\bf{\it Resposta:} Tautologia}
\item $3+1=4$ e $5+3=8$ implica $3+2=5$.  {\bf{\it Resposta:} N\ao \'e uma tautologia}
\item Vermelho \'e amarelo ou vermelho n\ao \'e amarelo. {\bf{\it Resposta:} Tautologia}
\item Vermelho \'e amarelo e vermelho \'e vermelho. {\bf{\it Resposta:} N\ao \'e uma tautologia}
\item $4$ \'e \ih mpar ou $2$ \'e par e $2$ \'e \ih mpar implica que $4$ \'e \ih mpar.  {\bf{\it Resposta:} Tautologia}
\item $4$ \'e \ih mpar ou $2$ \'e par e $2$ \'e \ih mpar implica que $4$ \'e par.  {\bf{\it Resposta:} N\ao \'e uma tautologia}
\end{enumerate}

%excercicio10
\item Quais das seguintes s\ao consequ\^encias l\'ogicas do conjunto de proposi\coes $p\ou q$, $r\to\nao q$, $\nao p$?
\begin{enumerate}[a)]
\item $q$.
\item $r$.
\item $\nao p\ou s$.
\item $\nao r$.
\item $\nao(\nao q\ee r)$.
\item $q\to r$.
\end{enumerate}
\end{enumerate}
%%%%%%%%%%%%%%%%%%%%%%%%%%%%%%%%%%%%%%%%%%%%%%%%%%%%%%%%%%%%%%%%%%%%%%%%
%%%%%%%%%%%%%%%%%%%%%%%%% secao 1.5 %%%%%%%%%%%%%%%%%%%%%%%%%%%%%%%%%%%%
%%%%%%%%%%%%%%%%%%%%%%%%%%%%%%%%%%%%%%%%%%%%%%%%%%%%%%%%%%%%%%%%%%%%%%%%
\paragraph{Exerc\ih cios \ref{demonstracao}}

\begin{enumerate}[{\bf 1.}]
%excercicio1
\item Determine a validade dos seguintes argumentos usando tabelas verdade:

\begin{enumerate}[a)]
\item \begin{tabular}{l}
$p\to q$ \\
$\underline{\nao p\ou q}$ \\
$q\to p$.
\end{tabular}

{\bf{\it Resposta:} Inv\'alido,}

\begin{table}[H]
\centering
\begin{tabular}{|l c r|c c c c c|}
\hline
\pp & & \qq & $(p \to q)$ &   $\ee$   & $(\nao p\ou q)$ &   $\to$    &  $(q\to p)$ \\
\hline
V   & &  V  &       V     &     V     &         V       &   {\bf V}  &        V    \\
V   & &  F  &       F     &     F     &         F       &   {\bf V}  &        V    \\
F   & &  V  &       V     &     V     &         V       &   {\bf F}  &        F    \\
F   & &  F  &       V     &     V     &         V       &   {\bf V}  &        V    \\
\hline
\end{tabular}
\end{table}


\item \begin{tabular}{l}
$p\ou q$ \\
$r \to q$ \\
$\underline{q}$ \\
$\nao r$.
\end{tabular}

{\bf{\it Resposta:} Inv\'alido, pode-se mostrar usando a tabela verdade ou observar que quando \pp \'e V, \qq \'e V e $r$ \'e V ou seja, as hip\'otese s\ao verdadeiras mas a conclus\ao \'e falsa.}

\item \begin{tabular}{l}
$p\ou \nao q$ \\
$\underline{\nao p}$ \\
$\nao q$.
\end{tabular}

{\bf{\it Resposta:} V\'alido,}

\begin{table}[H]
\centering
\begin{tabular}{|l c r|c c c c c|}
\hline
\pp & & \qq & $(p \ou \nao q)$ &   $\ee$   & $\nao p$ &   $\to$    &  $\nao q$ \\
\hline
V   & &  V  &       V          &     F     &    F    &   {\bf V}   &        F    \\
V   & &  F  &       V          &     F     &    F    &   {\bf V}   &        V    \\
F   & &  V  &       F          &     F     &    V    &   {\bf V}   &        F    \\
F   & &  F  &       V          &     V     &    V    &   {\bf V}   &        V    \\
\hline
\end{tabular}
\end{table}
\end{enumerate}


%excercicio2
\item D\^e exemplos nos \ih tens a seguir sempre que poss\ih vel. Se n\ao for poss\ih vel, diga porque:
\begin{enumerate}[a)]
\item Um argumento inv\'alido com conclus\ao falsa.
\item Um argumento v\'alido com uma conclus\ao verdadeira. 
\item Um argumento inv\'alido com uma conclus\ao verdadeira.
\item Um argumento v\'alido com uma conclus\ao falsa.
\item Um argumento v\'alido com hip\'oteses verdeiras e uma conclus\ao falsa. {\bf{\it Resposta:} N\ao h\'a tal exemplo porque se um argumento \'e v\'alido e as hip\'oteses s\ao verdadeiras ent\ao a conclus\ao deve necessariamente ser verdadeira.}
\item Um argumento inv\'alido com hip\'oteses verdeiras e uma conclus\ao falsa.
\item Um argumento v\'alido com hip\'oteses falsas e uma conclus\ao verdadeira. {\bf{\it Resposta:} \begin{tabular}{l}
$2+2=5$ \\
\underline{$2+2=5$ implica $1<3$} \\
$1<3$.
\end{tabular}}
\end{enumerate}

%excercicio3
\item Determine a validade dos seguintes argumentos usando o princ\ih pios de demonstra\cao ou mostre por contra exemplo que \'e inv\'alido:

\begin{enumerate}[a)]
\item \begin{tabular}{l}
$\nao p\ou q$ \\
\underline{$p$} \\
$q$.
\end{tabular}

{\it Resposta:}

\begin{tabu}{l c l}
   & &  \\\tabucline[2pt]{-}
Proposi\cao & & Raz\ao\\\tabucline[2pt]{-}
1. $\nao p \ou q$ & & hip\'otese \\
2. $p\to q$ & & consequ\^encia l\'ogica de 1. \\
3. $p$ & & hip\'otese. \\
4. $\nao p$ & & consequ\^encia l\'ogica de 2. e 3. \\\tabucline[2pt]{-}
\end{tabu}

\item \begin{tabular}{l}
$p\to q$ \\
\underline{$r\to \nao q$} \\
$p\to\nao r$.
\end{tabular}

{\it Resposta:}

\begin{tabu}{l c l}
   & &  \\\tabucline[2pt]{-}
Proposi\cao & & Raz\ao\\\tabucline[2pt]{-}
1. $r\to\nao q$ & & hip\'otese \\
2. $q\to\nao r$ & & consequ\^encia l\'ogica de 1. (contrapositiva) \\
3. $p\to q$ & & hip\'otese. \\
4. $p\to \nao r$ & & consequ\^encia l\'ogica de 3. e 2. \\\tabucline[2pt]{-}
\end{tabu}

\item \begin{tabular}{l}
$\nao p\ou q$ \\
\underline{$\nao r\to \nao q$} \\
$p\to\nao r$.
\end{tabular}

{\it Resposta:} Inv\'alido, se \pp \'e V, $r$ \'e V e \qq \'e V, ent\ao todas as hip\'oteses s\ao verdade mas a conclus\ao \'e falsa. 

\item \begin{tabular}{l}
$q\ou\nao p$ \\
\underline{$\nao q$} \\
$p$.
\end{tabular}

{\it Resposta:} Inv\'alido, se \pp \'e F e \qq \'e F, ent\ao todas as hip\'oteses s\ao verdade mas a conclus\ao \'e falsa.

\item \begin{tabular}{l}
\underline{$\nao p$} \\
$p\to q$.
\end{tabular}

{\it Resposta:}

\begin{tabu}{l c l}
   & &  \\\tabucline[2pt]{-}
Proposi\cao & & Raz\ao\\\tabucline[2pt]{-}
1. $\nao (p\to q)$ & & hip\'otese (nega\cao da conclus\ao na prova indireta) \\
2. $p\ee\nao q$ & & consequ\^encia l\'ogica de 1. \\
3. $p$ & & consequ\^encia l\'ogica de 2. \\
4. $\nao p$ & & hip\'otese \\
5. $p\ee\nao p$ & & consequ\^encia l\'ogica de 3. e 4. (contradi\caoi) \\
6. $p\to q$ & & consequ\^encia l\'ogica de 5. (prova indireta) \\\tabucline[2pt]{-}
\end{tabu}

\item \begin{tabular}{l}
$(p\ee q)\to(r\ee s)$ \\
\underline{$\nao r$} \\
$\nao p\ou\nao q$.
\end{tabular}

{\it Resposta:}

\begin{tabu}{l c l}
   & &  \\\tabucline[2pt]{-}
Proposi\cao & & Raz\ao\\\tabucline[2pt]{-}
1. $\nao (\nao p\ou\nao q)$ & & hip\'otese (nega\cao da conclus\ao na prova indireta) \\
2. $p\ee q$ & & consequ\^encia l\'ogica de 1. \\
3. $(p\ee q)\to(r\ee s)$ & & hip\'otese \\
4. $r\ee s$ & & consequ\^encia l\'ogica de 2. e 3. \\
5. $r$ & & consequ\^encia l\'ogica de 4.\\
6. $\nao r$ & & hip\'otese \\
7. $r\ee \nao r$ & & consequ\^encia l\'ogica de 5. e 6. (contradi\caoi) \\
8. $\nao p\ou\nao q$ & & consequ\^encia l\'ogica de 7. (prova indireta) \\\tabucline[2pt]{-}
\end{tabu}

\item \begin{tabular}{l}
$p\to q$ \\
$\nao q \to \nao r$ \\
$s\to(p\ou r)$ \\
\underline{$s$} \\
$q$.
\end{tabular}

{\it Resposta:}

\begin{tabu}{l c l}
   & &  \\\tabucline[2pt]{-}
Proposi\cao & & Raz\ao\\\tabucline[2pt]{-}
1. $\nao q\to \nao r$ & & hip\'otese \\
2. $r\to q$ & & consequ\^encia l\'ogica de 1. \\
3. $p\to q$ & & hip\'otese \\
4. $(p\ou r)\to q$ & & consequ\^encia l\'ogica de 2. e 3. \\
5. $s$ & & hip\'otese \\
6. $s\to (p\ou r)$ & & hip\'otese \\
7. $(p\ou r)$ & & consequ\^encia l\'ogica de 5. e 6. \\
8. $q$ & & consequ\^encia l\'ogica de 4. e 7. \\\tabucline[2pt]{-}
\end{tabu}

\item \begin{tabular}{l}
$p\ou q$ \\
$q\to \nao r$ \\
\underline{$\nao r\to\nao p$} \\
$\nao(p\ee q)$.
\end{tabular}

{\it Resposta:}

\begin{tabu}{l c l}
   & &  \\\tabucline[2pt]{-}
Proposi\cao & & Raz\ao\\\tabucline[2pt]{-}
1. $p\ee q$ & & hip\'otese (nega\cao da conclus\ao na prova indireta) \\
2. $\nao(p\to\nao q)$ & & consequ\^encia l\'ogica de 1. \\
3. $q\to \nao r$ & & hip\'otese \\
4. $r\to \nao q$ & & consequ\^encia l\'ogica de 3. \\
5. $\nao r\to\nao p$ & & hip\'otese \\
6. $p\to r$ & & consequ\^encia l\'ogica de 5. \\
7. $p\to \nao q$ & & consequ\^encia l\'ogica de 6. e 4. \\
8. $(p\to \nao q)\ee\nao(p\to \nao q)$ & & consequ\^encia l\'ogica de 7. e 2. (contradi\caoi) \\
9. $\nao(p\ee q)$ & & consequ\^encia l\'ogica de 8. (prova indireta) \\\tabucline[2pt]{-}
\end{tabu}

Note que a hip\'otese $p\ou q$ n\ao foi utilizada.

\item \begin{tabular}{l}
$p\to q$ \\
$\nao r \to \nao q$ \\
\underline{$r\to\nao p$} \\
$\nao p$.
\end{tabular}

{\it Resposta:}

\begin{tabu}{l c l}
   & &  \\\tabucline[2pt]{-}
Proposi\cao & & Raz\ao\\\tabucline[2pt]{-}
1. $p$ & & hip\'otese (nega\cao da conclus\ao na prova indireta) \\
2. $\nao r\to\nao q$ & & hip\'otese \\
3. $q\to r$ & & consequ\^encia l\'ogica de 2. \\
4. $p\to q$ & & hip\'otese \\
5. $p\to r$ & & consequ\^encia l\'ogica de 4. e 3. \\
6. $r\to\nao p$ & & hip\'otese \\
7. $p\to \nao p$ & & consequ\^encia l\'ogica de 5. e 6. \\
8. $\nao p$ & & consequ\^encia l\'ogica de 1. e 7. \\
9. $p\ee \nao p$ & & consequ\^encia l\'ogica de 1. e 8. (contradi\caoi)\\
10. $\nao p$ & & consequ\^encia l\'ogica de 9. (prova indireta) \\\tabucline[2pt]{-}
\end{tabu}

\item \begin{tabular}{l}
\underline{$p\to\nao p$} \\
$\nao p$.
\end{tabular}

{\it Resposta:}

\begin{tabu}{l c l}
   & &  \\\tabucline[2pt]{-}
Proposi\cao & & Raz\ao\\\tabucline[2pt]{-}
1. $p$ & & hip\'otese (nega\cao da conclus\ao na prova indireta) \\
2. $p\to\nao p$ & & hip\'otese \\
3. $\nao p$ & & consequ\^encia l\'ogica de 1. e 2. \\
4. $p\ee\nao p$ & & consequ\^encia l\'ogica de 1. e 3. (contradi\caoi) \\
5. $\nao p$ & & consequ\^encia l\'ogica de 4. (prova indireta) \\\tabucline[2pt]{-}
\end{tabu}


\item \begin{tabular}{l}
$p\ou q$ \\
$p\to r$ \\
\underline{$\nao r$} \\
$q$.
\end{tabular}

{\it Resposta:}

\begin{tabu}{l c l}
   & &  \\\tabucline[2pt]{-}
Proposi\cao & & Raz\ao\\\tabucline[2pt]{-}
1. $\nao r$ & & hip\'otese \\
2. $p\to r$ & & hip\'otese \\
3. $\nao r\to\nao p$ & & consequ\^encia l\'ogica de 2. \\
4. $\nao p$ & & consequ\^encia l\'ogica de 1. e 3. \\
5. $p\ou q$ & & hip\'otese \\
6. $q$ & & consequ\^encia l\'ogica de 4. e 5. \\\tabucline[2pt]{-}
\end{tabu}

\item \begin{tabular}{l}
$p$ \\
$q\to\nao p$ \\
$\nao q\to(r\ou\nao s)$ \\
\underline{$\nao r$} \\
$\nao s$.
\end{tabular}

{\it Resposta:}

\begin{tabu}{l c l}
   & &  \\\tabucline[2pt]{-}
Proposi\cao & & Raz\ao\\\tabucline[2pt]{-}
1. $q\to\nao p$ & & hip\'otese \\
2. $p\to \nao q$ & & consequ\^encia l\'ogica de 1. \\
3. $\nao q\to(r\ou\nao s)$ & & hip\'otese \\
4. $p\to r\ou\nao s$ & & consequ\^encia l\'ogica de 2. e 3. \\
5. $p$ & & hip\'otese \\
6. $r\ou\nao s$ & & consequ\^encia l\'ogica de 4. e 5. \\
7. $\nao r$ & & hip\'otese \\
8. $\nao s$ & & consequ\^encia l\'ogica de 6. e 7. \\\tabucline[2pt]{-}
\end{tabu}

\item \begin{tabular}{l}
$p\to(q\ou s)$ \\
\underline{$q\to r$} \\
$p\to(r\ou s)$.
\end{tabular}

{\it Resposta:}

\begin{tabu}{l c l}
   & &  \\\tabucline[2pt]{-}
Proposi\cao & & Raz\ao\\\tabucline[2pt]{-}
1. $p\to(q\ou s)$ & & hip\'otese \\
2. $\nao q\ee\nao s\to\nao p$ & & consequ\^encia l\'ogica de 1. \\
3. $(\nao q\to(\nao s\to\nao p))$ & & consequ\^encia l\'ogica de 2. \\
4. $q\to r$ & & hip\'otese \\
5. $\nao r\to\nao q$ & & consequ\^encia l\'ogica de 4. \\
6. $(\nao r\to(\nao s\to\nao p))$ & & consequ\^encia l\'ogica de 5. e 3. \\
7. $\nao r\ee\nao s\to\nao p$ & & consequ\^encia l\'ogica de 6. \\
8. $p\to(r\ou s)$ & & consequ\^encia l\'ogica de 7. \\\tabucline[2pt]{-}
\end{tabu}

\item \begin{tabular}{l}
$p\to\nao q$ \\
$q\to p$ \\
\underline{$r\to p$} \\
$\nao q$.
\end{tabular}

{\it Resposta:}

\begin{tabu}{l c l}
   & &  \\\tabucline[2pt]{-}
Proposi\cao & & Raz\ao\\\tabucline[2pt]{-}
1. $q$ & & hip\'otese (nega\cao da conclus\ao na prova indireta) \\
2. $q\to p$ & & hip\'otese \\
3. $p\to\nao q$ & & hip\'otese \\
4. $q\to\nao q$ & & consequ\^encia l\'ogica de 2. e 3. \\
5. $\nao q$ & & consequ\^encia l\'ogica de 1. e 4.\\
6. $q\ee \nao q$ & & consequ\^encia l\'ogica de 1. e 5. (contradi\caoi) \\
7. $\nao q$ & & consequ\^encia l\'ogica de 6. (prova indireta) \\\tabucline[2pt]{-}
\end{tabu}

Note que a hip\'otese $r\to p$ n\ao foi utilizada.

\item \begin{tabular}{l}
$p\to q$ \\
$r\to s$ \\
\underline{$\nao(p\to s)$} \\
$q\ee\nao r$.
\end{tabular}

{\it Resposta:}

\begin{tabu}{l c l}
   & &  \\\tabucline[2pt]{-}
Proposi\cao & & Raz\ao\\\tabucline[2pt]{-}
1. $\nao(q\ee\nao r)$ & & hip\'otese (nega\cao da conclus\ao na prova indireta) \\
2. $\nao q\ou r$ & & consequ\^encia l\'ogica de 1. \\
3. $p\to q$ & & hip\'otese \\
4. $\nao q\to \nao p$ & & consequ\^encia l\'ogica de 3. \\
5. $r\to s$ & & hip\'otese \\
6. $(\nao q\ou r)\to (\nao p\ou s)$ & & consequ\^encia l\'ogica de 4. e 5. \\
7. $\nao p\ou s$ & & consequ\^encia l\'ogica de 2. e 6. \\
8. $p\to s$ & & consequ\^encia l\'ogica de 7.\\
9. $\nao(p\to s)$ & & hip\'otese \\
10. $(p\to s)\ee\nao(p\to s)$ & & consequ\^encia l\'ogica de 8. e 9. (contradi\caoi)\\
11. $q\ee\nao r$ & & consequ\^encia l\'ogica de 10. (prova indireta) \\\tabucline[2pt]{-}
\end{tabu}
\end{enumerate}
\end{enumerate}
%%%%%%%%%%%%%%%%%%%%%%%%%%%%%%%%%%%%%%%%%%%%%%%%%%%%%%%%%%%%%%%%%%%%%%%%
%%%%%%%%%%%%%%%%%%%%%%%%% secao 1.6 %%%%%%%%%%%%%%%%%%%%%%%%%%%%%%%%%%%%
%%%%%%%%%%%%%%%%%%%%%%%%%%%%%%%%%%%%%%%%%%%%%%%%%%%%%%%%%%%%%%%%%%%%%%%%
\paragraph{Exerc\ih cios \ref{quantificadores}}

\begin{enumerate}[{\bf 1.}]
%excercicio1
\item Traduza as seguintes senten\cc as para a forma simb\'olica, indicando as escolhas apropriadas para dom\ih nios:
\begin{enumerate}[a)]
\item Existe um inteiro $x$ tal que $4=x+2$.

{\bf{\it Resposta:} Seja $\mathbb{Z}$ o conjunto de n\'umeros inteiros e ${\bf p(x)}$ ``${\bf 4=x+2}$.'' Ent\ao a proposi\cao \'e ${\bf \exists x \textrm{ em } \mathbb{Z} \mid  p(x)}$.}

\item Para todos inteiros $x$, $4=x+2$.

{\bf{\it Resposta:} Seja $\mathbb{Z}$ o conjunto de n\'umeros inteiros e ${\bf p(x)}$ ``${\bf 4=x+2}$.'' Ent\ao a proposi\cao \'e ${\bf \forall x \textrm{ em } \mathbb{Z}, p(x)}$.}

\item Todo tri\^angulo equil\'atero \'e equi\^angulo.
\item Todos estudantes gostam de l\'ogica.

{\bf{\it Resposta:} Sejam ${\bf D}$ o conjunto de todos os estudantes e ${\bf p(x})$ ``${\bf x}$ gosta de l\'ogica.'' Ent\ao a proposi\cao \'e ${\bf \forall x \textrm{ em } D, p(x)}$.}

\item Alguns estudantes n\ao gostam de l\'ogica.
\item Nenhum homem \'e uma ilha.
\item Todo mundo que entende l\'ogica gosta dela.

{\bf{\it Resposta:} Sejam ${\bf D}$ o conjunto de todas as pessoas, ${\bf p(x)}$ ``${\bf x}$ entende l\'ogica'' e ${\bf q(x)}$ ``${\bf x}$ gosta de l\'ogica'' Ent\ao a proposi\cao \'e ${\bf \forall x \textrm{ em } D, p(x)\to q(x)}$.}

\item Cada pessoa tem uma m\~ae.
\item Entre todos os inteiros existem uns que s\ao primos.
\item Alguns inteiros s\ao pares e divis\ih veis por 3.
\item Alguns inteiros s\ao pares ou divis\ih veis por 3. 

{\bf{\it Resposta:} Sejam ${\bf D}$ o conjunto dos inteiros, ${\bf p(x)}$ ``${\bf x}$ \'e par'' e ${\bf q(x)}$ ``${\bf x}$ \'e divis\ih vel por $3$.'' Ent\ao a proposi\cao \'e ${\bf \exists x \textrm{ em } D \mid  p(x)\ou q(x)}$.}

\item Todos grupos c\ih clicos s\ao abelianos.

{\bf{\it Resposta:} Sejam ${\bf D}$ o conjunto dos grupos, ${\bf p(x)}$ ``${\bf x}$ \'e um grupo c\ih clico'' e ${\bf q(x)}$ ``${\bf x}$ \'e grupo abeliano.'' Ent\ao a proposi\cao \'e ${\bf \forall x \textrm{ em } D, p(x)\to q(x)}$.}

\item Pelo menos uma das letras de {\it banana} \'e uma vogal.

{\bf{\it Resposta:} Sejam ${\bf D}$ o conjunto de todas as letras, ${\bf p(x)}$ ``${\bf x}$ uma das letras de {\it banana}'' e ${\bf q(x)}$ ``${\bf x}$ \'e uma vogal'' Ent\ao a proposi\cao \'e ${\bf \exists x \textrm{ em } D \mid  p(x)\ee q(x)}$.}

\item Um dia no pr\'oximo m\^es \'e uma sexta-feira.
\item $x^2-4=0$ tem uma solu\cao positiva.
\item Cada solu\cao de $x^2-4=0$ \'e positiva.  

{\bf{\it Resposta:} Sejam ${\bf D}$ o conjunto dos n\'umeros reais, ${\bf p(x)}$ ``${\bf x}$ \'e solu\cao de ${\bf x^2-4=0}$'' e ${\bf q(x)}$ ``${\bf x}$ \'e positiva.'' Ent\ao a proposi\cao \'e ${\bf \forall x \textrm{ em } D, p(x)\to q(x)}$.}

\item Nenhuma solu\cao de $x^2-4=0$ \'e positiva.
\item Um candidato ser\'a o vencedor.
\item Cada elemento do conjunto $A$ \'e um elemento do conjunto $B$.

{\bf{\it Resposta:} Seja ${\bf q(x)}$ ``${\bf x}$ \'e um elemento do conjunto $B$.'' Ent\ao a proposi\cao \'e ${\bf \forall x \textrm{ em } A, q(x)}$.}
\end{enumerate}

%excercicio2
\item Encontre a nega\cao para cada uma das proposi\coes no exerc\ih cio acima.\\
{\bf{\it Resposta d):} ${\bf\exists x \textrm{ em } D \mid  \nao p(x).}$ Existe um estudante que n\ao gosta de l\'ogica.}\\
{\bf{\it Resposta k):} ${\bf\forall x \textrm{ em } D, \nao(p(x)\ou q(x)).}$ Todos os inteiros s\ao \ih mpares e n\ao divis\ih veis por $3$.}\\
{\bf{\it Resposta p):} ${\bf\bf\exists x \textrm{ em } D \mid  \nao (p(x)\to q(x)).}$ Existe uma solu\cao de ${\bf x^2-4=0}$ que n\ao \'e positiva.}\\

%excercicio3
\item Sejam $D$ o conjunto dos n\'umeros naturais (isto \'e, $D=\{1,2,3,4,5,\ldots\}$), $p(x)$ ``$x$ \'e par'', $q(x)$ ``$x$ \'e divis\ih vel por $3$'' e $r(x)$ ``$x$ \'e div\ih sivel por $4$.'' Para cada uma das proposi\coes abaixo, expresse em Portugu\^es, determine seu valor verdade e d\^e uma nega\cao em Portugu\^es.  
\begin{enumerate}[a)]
\item $\forall x \textrm{ em } D, p(x)$.
\item $\forall x \textrm{ em } D, p(x)\ou q(x)$.
\item $\forall x \textrm{ em } D, p(x)\to q(x)$. {\bf{\it Resposta:} Cada n\'umero natural par \'e divis\ih vel por $3$; falso; existe um n\'umero natural par que n\ao \'e divis\ih vel por $3$.}
\item $\forall x \textrm{ em } D, p(x)\ou r(x)$.
\item $\forall x \textrm{ em } D, p(x)\ee q(x)$.
\item $\exists x \textrm{ em } D \mid  r(x)$.
\item $\exists x \textrm{ em } D \mid  p(x)\ee q(x)$.
\item $\exists x \textrm{ em } D \mid  p(x)\to q(x)$.
\item $\exists x \textrm{ em } D \mid  q(x)\to q(x+1)$. {\bf{\it Resposta:} Existe um n\'umero natural tal que se \'e divis\ih vel por $3$ ent\ao o pr\'oximo n\'umero natural \'e divis\ih vel por $3$; verdade; todos n\'umeros naturais s\ao divis\ih veis por $3$ e o pr\'oximo n\'umero natural n\ao \'e divis\ih vel por $3$.}
\item $\exists x \textrm{ em } D \mid  p(x) \leftrightarrow q(x+1)$.
\item $\forall x \textrm{ em } D, r(x)\to p(x)$.
\item $\forall x \textrm{ em } D, p(x)\to\nao q(x)$.
\item $\forall x \textrm{ em } D, p(x)\to p(x+2)$.
\item $\forall x \textrm{ em } D, r(x)\to r(x+4)$.
\item $\forall x \textrm{ em } D, q(x)\to q(x+1)$.
\end{enumerate}

%excercicio4
\item Para cada uma das proposi\coes do exerc\ih cio acima (se poss\ih vel) d\^e um exemplo de um dom\ih nio $D'$ tal que as proposi\coes tenham o valor verdade oposto daquele que tinha em $D$, o conjunto dos n\'umeros naturais.\\
{\bf{\it Resposta c):} Se ${\bf D'}$ \'e o conjunto dos n\'umeros naturais divis\ih veis por $3$ ent\ao esta proposi\cao ser\'a verdade.}\\

%excercicio5
\item As seguintes proposi\coes s\ao sempre, \`as vezes ou nunca verdade? D\^e exemplos de dom\ih nios $D$ e a fun\cao proposicional $p$ ou raz\oes para justificar suas respostas. 
\begin{enumerate}[a)]
\item $[\forall x \textrm{ em } D, p(x)]\to[\exists x \textrm{ em } D \mid  p(x)]$.
\item $[\exists x \textrm{ em } D \mid  p(x)]\to[\forall x \textrm{ em } D, p(x)]$. {\bf{\it Resposta:} \`As vezes correta; ser\'a verdade quando $D$ cont\'em no m\'aximo um elemento ou se $p(x)$ \'e verdade para todo $x$ em $D$ e ser\'a falso se existir pelo menos um elemento $x$ em $D$ para o qual $p(x)$ \'e verdade e um elemento em $D$ para o qual \'e falso. }
\item $[\forall x \textrm{ em } D, \nao p(x)]\to\nao[\forall x \textrm{ em } D, p(x)]$.
\item $[\exists x \textrm{ em } D \mid  \nao p(x)]\to \nao[\exists x \textrm{ em } D \mid  p(x)]$.
\item $\nao[\forall x \textrm{ em } D, p(x)]\to[\forall x \textrm{ em } D, \nao p(x)]$.
\item $\nao[\exists x \textrm{ em } D \mid  p(x)]\to[\exists x \textrm{ em } D \mid  \nao p(x)]$.
\end{enumerate}
\end{enumerate}
%%%%%%%%%%%%%%%%%%%%%%%%%%%%%%%%%%%%%%%%%%%%%%%%%%%%%%%%%%%%%%%%%%%%%%%%
%%%%%%%%%%%%%%%%%%%%%%%%% secao 1.7 %%%%%%%%%%%%%%%%%%%%%%%%%%%%%%%%%%%%
%%%%%%%%%%%%%%%%%%%%%%%%%%%%%%%%%%%%%%%%%%%%%%%%%%%%%%%%%%%%%%%%%%%%%%%%
\paragraph{Exerc\ih cios \ref{mquantificadores}}

\begin{enumerate}[{\bf 1.}]
%excercicio1
\item Traduza as seguintes senten\cc as para a forma simb\'olica, indicando as escolhas apropriadas para dom\ih nios:
\begin{enumerate}[a)]
\item Para cada inteiro par $n$ existe um inteiro $k$ tal que $n=2k$.
\item Para cada reta $l$ e cada ponto $p$ que n\ao est\'a em $l$ existe uma reta $l'$ que passa por $p$ que \'e paralela a $l$.
\item Para cada $y$ em $B$ existe um $x$ em $A$ tal que $f(x)=y$.

{\bf{\it Resposta:} Seja ${\bf p(x,y)}$ ``${\bf f(x)=y}$.'' Ent\ao a proposi\cao \'e ${\bf \forall y \textrm{ em } B, \exists x \textrm{ em } A \mid  p(x,y)}$.}

\item Para cada $x$ no dom\ih nio de $f$ e para cada $\epsilon >0$ existe $\delta >0$ tal que $|x-c|<\delta$ implica $|f(x)-L|<\epsilon$.
\item Para cada $x$ em $G$ existe um $x'$ em $G$ tais que $xx'=e$.

{\bf{\it Resposta:}  Seja ${\bf p(x,y)}$ ``${\bf xy=e}$''. Ent\ao a proposi\cao \'e ${\bf \forall x \textrm{ em } G, \exists x' \textrm{ em } G  \mid  p(x,x')}$.}

\item Se todo inteiro \'e \ih mpar ent\ao todo inteiro \'e par. 

{\bf{\it Resposta:} Sejam ${\bf D}$ o conjunto dos inteiros, ${\bf p(x)}$ ``${\bf x}$ \'e \ih mpar.'' Ent\ao a proposi\cao \'e ${\bf(\forall x \textrm{ em } D, p(x))\to(\forall x \textrm{ em } D, \nao p(x))}$.}

\item Algu\'em ama algu\'em em algum momento.
\item Entre todas as pulgas do carpete existe uma para a qual existe em todos os cachorros no sof\'a uma mordida que aquela pulga fez.
\item Para cada inteiro $n$ existe outro inteiro maior que $2n$.

{\bf{\it Resposta:}  Sejam ${\bf D}$ o conjunto dos n\'umeros inteiros, ${\bf p(x,y)}$ ``${\bf x>2y}$''. Ent\ao a proposi\cao \'e ${\bf \forall n \textrm{ em } D, \exists m \textrm{ em } D  \mid  p(m,n)}$.}

\item A soma de quaisquer dois inteiros pares \'e par.

{\bf{\it Resposta:}  Sejam ${\bf D}$ o conjunto dos n\'umeros inteiros pares, ${\bf p(x,y)}$ ``${\bf x+y}$ pertence \`a D''. Ent\ao a proposi\cao \'e ${\bf \forall x,y \textrm{ em } D, p(x,y)}$.}

\item Todo subconjunto fechado e limitado de $\mathbb{R}$ \'e compacto.

{\bf{\it Resposta:}  Sejam ${\bf D}$ o conjunto dos subconjuntos de $\mathbb{R}$, ${\bf p(x)}$ ``${\bf x}$ \'e fechado'', ${\bf q(x)}$ ``${\bf x}$ \'e limitado'', ${\bf r(x)}$ ``${\bf x}$ \'e compacto''. Ent\ao a proposi\cao \'e \\ ${\bf \forall x \textrm{ em } D, p(x)\ee q(x)\to r(x)}$.}
\end{enumerate}

%excercicio2
\item Encontre a nega\cao para cada uma das proposi\coes no exerc\ih cio acima.\\
{\bf{\it Resposta f):} Todo inteiro \'e \ih mpar e existe um inteiro \ih mpar.}\\

%excercicio3
\item Sejam $p(x,y)$ representando ``$x+2>y$'' e $D$ o conjunto dos n\'umeros naturais ($D=\{1,2,3,\ldots\}$, tamb\'em denotado por $\mathbb{N}$). Escreva em palavras e determine o valor verdade de
\begin{enumerate}[a)]
\item $\forall x \textrm{ em } D, \exists y \textrm{ em } D \mid  p(x,y)$.
\item $\exists x \textrm{ em } D \mid  \forall y \textrm{ em } D, p(x,y)$.
\item $\forall x \textrm{ em } D, \forall y \textrm{ em, } p(x,y)$. {\bf{\it Resposta:} Para cada n\'umero natural ${\bf x}$ e cada n\'umero natural ${\bf y}$, ${\bf x+2>y}$. Falsa.}
\item $\exists x \textrm{ em } D \mid  \exists y \textrm{ em } D \mid  p(x,y)$.
\item $\forall y \textrm{ em } D, \exists x \textrm{ em } D \mid  p(x,y)$.
\item $\exists y \textrm{ em } D \mid  \forall x \textrm{ em } D, p(x,y)$.
\end{enumerate} 

%excercicio4
\item Sejam $D=\{1,2\}$, $p(x)$ ``$x$ \'e par'' e $q(x)$ ``$x$ \'e \ih mpar.'' Escreva em detalhes as seguintes quantifica\coes como conjun\coes e disjun\coes  das interpreta\coes (como feito no come\cc o desta se\caoi):
\begin{enumerate}[a)]
\item $\forall x \textrm{ em } D, [p(x)\ee q(x)]$.
\item $[\forall x \textrm{ em } D, p(x)]\ee[\forall x \textrm{ em } D, q(x)]$.
\item $\forall x \textrm{ em } D, [p(x)\ou q(x)]$. {\bf{\it Resposta:} [$1$ \'e par ou $1$ \'e \ih mpar] e [$2$ \'e par ou $2$ \'e \ih mpar.]}
\item $[\forall x \textrm{ em } D, p(x)]\ou[\forall x \textrm{ em } D, q(x)]$.
\item $\exists x \textrm{ em } D \mid  [p(x)\ee q(x)]$.
\item $[\exists x \textrm{ em } D \mid  p(x)]\ee[\exists x \textrm{ em } D \mid  q(x)]$.
\item $\exists x \textrm{ em } D \mid  [p(x)\ou q(x)]$.
\item $[\exists x \textrm{ em } D \mid  p(x)]\ou[\exists x \textrm{ em } D \mid  q(x)]$.
\item $\exists x \textrm{ em } D \mid  [p(x)\to q(x)]$.
\item $[\exists x \textrm{ em } D \mid  p(x)]\to[\exists x \textrm{ em } D \mid  q(x)]$.
\end{enumerate}

%excercicio5
\item D\^e alguns exemplos para mostrar que as seguintes implica\coes l\'ogicas n\ao s\ao equival\^encias l\'ogicas:
\begin{enumerate}[a)]
\item $\{[\forall x \textrm{ em } D, p(x)]\ou[\forall x \textrm{ em } D, q(x)]\}\Rightarrow \forall x \textrm{ em } D, [p(x)\ou q(x)]$.
\item $\exists x \textrm{ em } D \mid  [p(x)\ee q(x)]\Rightarrow\{[\exists x \textrm{ em } D \mid  p(x)]\ee[\exists x \textrm{ em } D \mid  q(x)]\}$. {\bf{\it Resposta:} Sejam ${\bf D}$ o conjunto dos n\'umeros naturais, ${\bf p(x)}$ ``$x$ \'e par'' e ${\bf q(x)}$ ``x \'e \ih mpar.''}
\item $\exists x \textrm{ em } D \mid  [p(x)\to q(x)]\Rightarrow\{[\exists x \textrm{ em } D \mid  p(x)]\to[\exists x \textrm{ em } D \mid  q(x)]\}$.
\end{enumerate}

%excercicio6
\item Determine a rela\cao (se existir uma) entre
\[
\exists x \textrm{ em } D \mid  [p(x)\to q(x)]
\]
e
\[
[\exists x \textrm{ em } D \mid  p(x)]\to[\exists x \textrm{ em } D \mid  q(x)].
\]

%excercicio7
\item Mostre que a segunda equival\^encia l\'ogica em cada uma dos pares pode ser obtida da primeira por nega\caoi: 
\begin{enumerate}[a)]
\item 
\[
[\exists x \textrm{ em } S \mid  \exists y \textrm{ em } T \mid  p(x,y)]\Leftrightarrow[\exists y \textrm{ em } T \mid  \exists x \textrm{ em } S \mid  p(x,y)]
\]
e
\[
[\forall x \textrm{ em } S, \forall y \textrm{ em } T, p(x,y)]\Leftrightarrow[\forall y \textrm{ em } T, \forall x \textrm{ em } S, p(x,y)]
\]

\item 
\[
\{[\forall x \textrm{ em } D, p(x)]\ee[\forall x \textrm{ em } D, q(x)]\}\Leftrightarrow \forall x \textrm{ em } D, [p(x)\ee q(x)]
\]
e
\[
\exists x \textrm{ em } D \mid  [p(x)\ou q(x)]\Leftrightarrow\{[\exists x \textrm{ em } D \mid  p(x)]\ou[\exists x \textrm{ em } D \mid  q(x)]\}.
\]
\end{enumerate}

%excercicio8
\item Considere as seguinte proposic\aoi: ``Para toda galinha na gaiola e para toda cadeira na cozinha existe uma frigideira no arm\'ario tal que se o ovo da galinha est\'a na frigideira ent\ao a galinha est\'a a menos de dois metros da cadeira.''
\begin{enumerate}[a)]
\item Traduza esta proposic\ao para a forma simb\'olica.
\item Expresse a nega\cao em s\ih mbolos e em Portugu\^es.
\item D\^e dois exemplos de ciscunst\^ancias nas quais a proposi\cao seria verdade.
\item D\^e dois exemplos de ciscunst\^ancias nas quais a proposi\cao seria falsa.
\end{enumerate}
\end{enumerate}
%%%%%%%%%%%%%%%%%%%%%%%%%%%%%%%%%%%%%%%%%%%%%%%%%%%%%%%%%%%%%%%%%%%%%%%%
%%%%%%%%%%%%%%%%%%%%%%%%% secao 1.8 %%%%%%%%%%%%%%%%%%%%%%%%%%%%%%%%%%%%
%%%%%%%%%%%%%%%%%%%%%%%%%%%%%%%%%%%%%%%%%%%%%%%%%%%%%%%%%%%%%%%%%%%%%%%%
\paragraph{Exerc\ih cios \ref{metdem}}

\begin{enumerate}[{\bf 1.}]
%excercicio1
\item Escreva a primeira e  \'ultima linhas da demonstra\cao direta, por contrapositiva e indireta dos seguintes teoremas abaixo: 
\begin{enumerate}[a)]
\item Se $m$ \'e um inteiro par ent\ao $m^2$ \'e par. 

{\bf{\it Resposta:} Direta:(Seja ${\bf m}$ \'e um inteiro par.) e (Assim ${\bf m^2}$ \'e par.); Contrapositiva:(Suponha que ${\bf m^2}$ n\ao seja par, ou seja, \ih mpar.) e (Assim ${\bf m}$ \'e \ih mpar.); Indireta:(Suponha que ${\bf m}$ seja inteiro par e que ${\bf m^2}$ se \ih mpar.) e (Alguma contradi\caoi.)}

\item Se $f$ \'e uma fun\cao diferenci\'avel ent\ao $f$ \'e uma fun\cao cont\ih nua. 

{\bf{\it Resposta:} Direta:(Seja ${\bf f}$ uma fun\cao diferenci\'avel.) e (Assim ${\bf f}$ \'e cont\ih nua.); Contrapositiva:(Suponha que ${\bf f}$ n\ao seja uma fun\cao cont\ih nua.) e (Assim ${\bf f}$ n\ao \'e diferenci\'avel.); Indireta:(Suponha que ${\bf f}$ seja uma fun\cao diferenci\'avel a qual n\ao \'e cont\ih nua.) e (Alguma contradi\caoi.)}

\item $L$ \'e uma tranforma\cao linear injetora se e somente se $Ker(L)=\{0\}$.
\item Se ($a_n$) \'e monot\^onica e limitada ent\ao ($a_n$) converge.
\item A imagem homom\'orfica de um grupo c\ih clico \'e um grupo c\ih clico. 

{\bf{\it Resposta:} Direta:(Seja ${\bf H}$ a imagem homom\'ofica de um grupo c\ih clico ${\bf G}$.) e (Assim, ${\bf H}$ \'e c\ih clico.); Contrapositiva:(Suponha que ${\bf H}$ n\ao seja um grupo c\ih clico.) e (Assim ${\bf H}$ n\ao \'e a imagem homom\'orfica de qualquer grupo c\ih clico.); Indireta:(Suponha que ${\bf H}$ \'e a imagem homom\'orfica de um grupo c\ih clico e que $H$ n\ao seja c\ih clico.) e (Alguma contradi\caoi.)}

\item Se o \'unico termos n\ao zero de uma expans\ao $p-$\'adica de $n$ \'e $1$ ent\ao $n=p^k$ para algum $k\leq 0$.
\item Se $f$ n\ao \'e cont\ih nua em $c$ ent\ao $\lim_{x\to c}f(x)$ n\ao existe ou $\lim_{x\to c}f(x)\neq f(c)$. 
\item Todo conjunto fechado e limitado de $\mathbb{R}$ \'e compacto.
\item Se $m$ \'e um inteiro da forma $2,4,p^n,2p^n$ onde $p$ \'e um primo \ih mpar e $n$ \'e um inteiro positivo ent\ao $m$ tem ra\ih zes primitivas.
\end{enumerate}

%excercicio2
\item Determine quais das seguintes ``demonstra\cois'' s\ao corretas e quais s\ao incorretas. Se a demonstra\cao est\'a correta, indique o tipo e se a demonstra\cao est\'a incorreta, indique porque a demonstra\cao \'e incorreta.
\\
\\
{\bf Teorema:} Se $x$ e $y$ s\ao inteiros pares ent\ao $x-y$ \'e um inteiro par.

\begin{enumerate}[a)]
\item ``Demonstra\cao 1'': Suponha que $x$ e $y$ s\ao ambos inteiros \ih mpares. Ent\ao existem inteiros $j,k$ tais que $x=2j+1$ e $y=2k+1$. Assim,
\[
x-y=2j+1-(2k+1)=2(j-k)
\] 
que \'e par.

{\bf{\it Resposta:} Incorreta. A hip\'otese est\'a incorreta, apesar de a conclus\ao estar correta.}

\item ``Demonstra\cao 2'': Suponha que $x-y$ seja par e $x$ \ih mpar. Ent\ao existem inteiros $j,k$ tais que $x-y=2j$ e $x=2k+1$. Assim,
\[
y=y-x+x=-2j+(2k+1)=2(k-j)+1
\] 
portanto $y$ \'e \ih mpar, uma contradi\caoi.
\item ``Demonstra\cao 3'': Suponha que $x-y$ seja \ih mpar. Ent\ao existe um inteiro $j$ tal que $x-y=2j+1$. Se $y$ \'e par , oteorema est\'a demonstrado. Portanto, suponha que $y$ seja \ih mpar, digamos $y=2k+1$ para algum inteiro $k$. Assim,
\[
x=x-y+y=2j+1-(2k+1)=2(j-k)+1
\]
logo $x$ \'e par e a demonstra\cao est\'a completa.  

{\bf{\it Resposta:} Incorreta. ${\bf x-y}$ \ih mpar \'e a nega\cao da conclus\ao, portanto, para usar o m\'etodo da contrapositiva para demonstrar, ter\ih amos que mostrar que ${\bf x}$ \'e \ih mpar ou ${\bf y}$ \'e \ih mpar, o que esta ``demonstra\caoi'' n\ao o faz.}

\item ``Demonstra\cao 4'': Suponha que $x$ seja par e $x-y$ seja par tamb\'em. Ent\ao existem inteiros $j,k$ tais que $x=2j$ e $x-y=2k$. Assim,
\[
y=x-(x-y)=2j-2k=2(j-k)
\]
logo $y$ tamb\'em \'e par.

\item ``Demonstra\cao 5'': Suponha que $x,y$ sejam pares e $x-y$ \ih mpar. Ent\ao existem inteiros $j,k$ tais que $x=2j$ e $y=2k$. Assim,
\[
x-y=2j-2k=2(j-k)
\] 
portanto $x-y$ \'e par. Mas isto contradiz nossa premissa que $x-y$ \'e \ih mpar, logo a demonstra\cao est\'a completa.

{\bf{\it Resposta:} Correta. Demonstra\cao indireta.}

\item ``Demonstra\cao 6'': Suponha que $x-y$ seja \ih mpar, digamos $x-y=2j+1$ para algum inteiro $j$. Se $x$ \'e \ih mpar o teorema estar\'a demonstrado. Portanto, assuma que $x$ seja par, digamos $x=2k$ para algum inteiro $k$. Ent\aoi,
\[
y=x-(x-y)=2k-(2j+1)=2(k-j)-1=2(k-j-1)+1
\]
logo, $y$ \'e \ih mpar e o teorema est\'a demonstrado.
\item ``Demonstra\cao 7'': Suponha que $x$ e $y$ sejam ambos pares. Ent\ao existem inteiros $j,k$ tais que $x=2j$ e $y=2k$. Assim,
\[
x-y=2j-2k=2(j-k)
\]
portanto, $x-y$ \'e par.
\item ``Demonstra\cao 8'': Suponha que $x-y$ seja par. Ent\ao se $x$ for \ih mpar, o teorema estar\'a demonstrado. Logo, suponha que $x$ seja par. Ent\ao existem inteiros $j,k$ tais que $x-y=2j$ e $x=2k$. Assim, 
\[
y=x-(x-y)=2k-2j=2(k-j)
\]
logo, $y$ tamb\'em \'e par.
\item ``Demonstra\cao 9'': Suponha que $x-y$ seja \ih mpar, digamos $x-y=2j+1$ para algum inteiro $j$. Ent\ao se $x$ \'e \ih mpar, digamos $x=2k+1$ para algum $k$, teremos
\[
y=x-(x-y)=2k+1-(2j+1)=2(k-j)
\]
logo $y$ \'e par e a demonstra\cao est\'a completa.
\item ``Demonstra\cao 10'': Suponha que $x$ e $y$ sejam \ih mpares e $x-y$ \ih mpar tamb\'em. Ent\ao existem inteiros $j,k$ tais que $x=2j+1$ e $y=2k+1$. Assim,
\[
x-y=2j+1-(2k+1)=2(j-k)
\]
logo, $x-y$ \'e \ih mpar e par, uma contradi\caoi.
\end{enumerate}

%excercicio3
\item D\^e a demonstra\cao direta, por contrapositiva e indireta (se poss\ih vel) de:
\begin{enumerate}[a)]
\item Se $x$ \'e um inteiro par e $y$ \'e um inteiro \ih mpar ent\ao $x+y$ \'e um inteiro \ih mpar. 

{\bf{\it Resposta:} \underline{Direta}: Suponha que ${\bf x}$ seja par e ${\bf y}$ \ih mpar. Ent\ao existem inteiros ${\bf j,k}$ tais que ${\bf x=2k}$ e ${\bf y=2j+1}$. Assim,
\[
{\bf x+y=2k+2j+1=2(k+j)+1}
\]
e portanto ${\bf x+y}$ \'e \ih mpar.

\noindent \underline{Contrapositiva}: Suponha que ${\bf x+y}$ seja par. Ent\ao existe um inteiro ${\bf k}$ tal que ${\bf x+y=2k}$. Se ${\bf x}$ for \ih mpar ent\ao a demonstra\cao estar\'a completa. Suponha que ${\bf x}$ seja par, digamos ${\bf x=2j}$ para algum inteiro ${\bf j}$. Ent\aoi,
\[
{\bf y=2k-2j=2(k-j)}
\]
portanto, ${\bf y}$ \'e par e a demonstra\cao est\'a completa.

\noindent \underline{Indireta}: Suponha que ${\bf x}$ seja par, que ${\bf y}$ seja \ih mpar e que ${\bf x+y}$ seja par. Ent\ao esxistem inteiros ${\bf j,k}$ tais que ${\bf y=2j+1}$ e ${\bf x+y=2k}$. Logo,
\[
{\bf x=x+y-y=2j+1-2k=2(j-k)+1}
\]
\'e \ih mpar, contradizendo o fata que ${\bf x}$ \'e par, por hip\'otese.}

\item Se $x$ e $y$ s\ao inteiros \ih mpares ent\ao $xy$ \'e um inteiro \ih mpar.
\end{enumerate}

%excercicio4
\item Para as seguintes conjecturas, demonstre que \'e verdade ou d\^e um contra-exemplo para mostrar que \'e falso: 
\begin{enumerate}[a)]
\item Se $x$ \'e um inteiro e $4x$ \'e par ent\ao $x$ \'e par. 

{\bf{\it Resposta:} Falsa, por exemplo, tome ${\bf x=3}.$}

\item Se $x$ \'e um inteiro par ent\ao $4x$ \'e par.

{\bf{\it Resposta:} Verdade. Suponha que ${\bf x}$ \'e um inteiro par, ent\ao exite um inteiro ${\bf k}$ tal que ${\bf x=2k}$. Logo,
\[
{\bf 4x=4(2k)=2(4k),}
\]
\'e par.}

\item Se $x$ \'e um inteiro e $x^2$ \'e par ent\ao $x$ \'e par.

{\bf{\it Resposta:} Verdade. Suponha que ${\bf x}$ seja \ih mpar, digamos ${\bf x=2k+1}$ para algum inteiro ${\bf k}$. Ent\ao 
\[
{\bf x^2=(2k+1)^2=4k^2+4k+1=2(2k^2+2k)+1}
\] 
\'e \ih mpar. [Nota: Usamos a demonstra\cao por contrapositiva aqui, para a demonstra\cao direta ter\ih amos algum trabalho para ir de ${\bf x^2}$ para ${\bf x}$.]}

\item Se $x$ \'e um inteiro e $3x$ \'e par ent\ao $x$ \'e par.

{\bf{\it Resposta:} Verdade. Suponha que ${\bf x}$ seja \ih mpar, ent\ao existe ${\bf k}$ inteiro tal que ${\bf x=2k+1}$. Logo,
\[
{\bf 3x=3(2k+1)=6k+3=6k+2+1=2(3k+1)+1}
\]
\'e \ih par.}

\item Se $x,y,z$ s\ao inteiros e $x+y+z$ \'e \ih mpar ent\ao um n\'umero de $x,y,z$ \'e \ih mpar.

{\bf{\it Resposta:} Verdade. Se ${\bf x+y+x}$ \'e \ih mpar, ent\ao existe ${\bf k}$ inteiro tal que ${\bf x+y+z=2k+1}$. Se um dos ${\bf x,y,z}$ \'e \ih mpar, digamos ${\bf x}$, ent\ao existe ${\bf j}$ inteiro tal que ${\bf x=2j+1}$. Portanto,
\[
{\bf y+z=(x=y+z)-x=2k=1-(2j+1)=2(k-j)}
\]
\'e par. Mas se a soma de dois inteiros \'e par, ent\ao ou ambos s\ao pares ou ambos s\'ao \ih mpares.}
\end{enumerate}

%excercicio5
\item Pareceria que poderia existir uma quarta forma de demonstra\caoi, uma demonstra\cao indireta da contrapositiva de um teorema. Explique porque este fato n\ao foi mencionado na discuss\ao acima.

{\bf{\it Resposta:} A indireta da contrapositiva \'e:
\[
(\nao q \ee \nao(\nao p)\to c) \iff (\nao q \ee p\to c) \iff (p \ee \nao q\to c)
\]
que \'e o mesmo que a reduc\ao ao absurso.
}
\end{enumerate}
%%%%%%%%%%%%%%%%%%%%%%%%%%%%%%%%%%%%%%%%%%%%%%%%%%%%%%%%%%%%%%%%%%%%%%%%
%%%%%%%%%%%%%%%%%%%%%%%%% secao 2.1 %%%%%%%%%%%%%%%%%%%%%%%%%%%%%%%%%%%%
%%%%%%%%%%%%%%%%%%%%%%%%%%%%%%%%%%%%%%%%%%%%%%%%%%%%%%%%%%%%%%%%%%%%%%%%
\paragraph{Exerc\ih cios \ref{conjuntos}}

\begin{enumerate}[{\bf 1.}]
%excercicio1
\item Sejam
\begin{eqnarray*}
\mathbb{U}&=&\{1,2,3,4,5,6,7,8\},\\
A&=&\{1,2,3,4\},\\
B&=&\{x:(x-2)^2(x-3)=0\},\\
C&=&\{x:\espaco x \textrm{ \'e \ih mpar}\}.
\end{eqnarray*}
Encontre:
\begin{enumerate}[a)]
\item $A\cup B$. {\bf{\it Resposta:} ${\bf A}$}
\item $A\cap(B\cup C)$. {\bf{\it Resposta:} ${\bf \{1,2,3\}}$}
\item $C-A$. {\bf{\it Resposta:} ${\bf \{5,7\}}$}
\item $C\uni A^C$. {\bf{\it Resposta:} ${\bf \{1,3,5,6,7,8\}}$}
\item $(A\uni C)^C$. {\bf{\it Resposta:} ${\bf \{6,8\}}$}
\item $A^C\inter C^C$. {\bf{\it Resposta:} ${\bf \{6,8\}}$}
\item $\mathbb{P}(B)$. {\bf{\it Resposta:} ${\bf \{\varnothing,\{2\},\{3\},B\}}$}
\end{enumerate}

%excercicio2
\item Escreva em Portugu\^es a nega\cao de $A\subseteq B$ dada nesta se\caoi.
{\bf{\it Resposta:} A defini\cao dada no texto \'e ${\bf A \subseteq B \leftrightarrow (\forall x,\espaco x\in A \to x\in B)}$. Portanto, ${\bf A \not\subseteq B \leftrightarrow (\exists x \mid  \espaco x\in A \ee x\notin B)}$. Assim em Portugu\^es temos, existe um elemento pertencente a ${\bf A}$ que n\ao pertence \`a ${\bf B}$.}

%excercicio3
\item Seja $\mathbb{U}=\mathbb{R}$, o conjunto dos n\'umeros reais. Considere os seguintes conjuntos:
\begin{equation*}
 \begin{aligned}
(a,b)&=\{x:\espaco a<x<b\},\\
(a,b]&=\{x:\espaco a<x\leq b\},\\
[a,b)&=\{x:\espaco a\leq x< b\},\\
[a,b]&=\{x:\espaco a\leq x\leq b\},\\
(-\infty,a)&=\{x:\espaco x<a\},\\
(-\infty,a]&=\{x:\espaco x\leq a\},\\
(a,\infty)&=\{x:\espaco a<x\},\\
[a,\infty)&=\{x:\espaco a\leq x\}.
 \end{aligned}
\end{equation*}
Encontre:
\begin{enumerate}[a)]
\item $[1,3]\inter(2,4)$. {\bf{\it Resposta:} ${\bf (2,3]}$}
\item $(-\infty,2)\inter[-1,0]$. {\bf{\it Resposta:} ${\bf [-1,0]}$}
\item $(-\infty,2)\inter[-1,3]$. {\bf{\it Resposta:} ${\bf [-1,2)}$}
\item $[0,10]\uni(1,11)$. {\bf{\it Resposta:} ${\bf [0,11)}$}
\item $(0,\infty)\inter(-\infty,1)$. {\bf{\it Resposta:} ${\bf (0,1)}$}
\item $(1,\infty)\inter(-\infty,0)$. {\bf{\it Resposta:} ${\bf \varnothing}$}
\item $[-2,0]\uni[0,2]$. {\bf{\it Resposta:} ${\bf [-2,2]}$}
\item $[-2,0]\uni(0,2]$. {\bf{\it Resposta:} ${\bf [-2,2]}$}
\item $[-2,0)\uni(0,2]$. {\bf{\it Resposta:} ${\bf [-2,0)\uni(0,2]}$}
\item $[-2,0]\uni[2,0]$. {\bf{\it Resposta:} ${\bf \varnothing}$}
\item $(0,4]^C$. {\bf{\it Resposta:} ${\bf [-\infty,0]\uni(4,\infty]}$}
\item $\mathbb{P}([1,1])$. {\bf{\it Resposta:} ${\bf \{\varnothing,\{1\}\}}$}
\item $\mathbb{P}([0,1])$. {\bf{\it Resposta:} Para este precisamos de un n\'umero infinito de folhas de papel.}
\end{enumerate}

%excercicio4
\item Mostre que dois conjuntos vazios, mencionados na discuss\ao de conjuntos vazios, s\ao iguais.

%excercicio5
\item \label{conjuntos5}Suponha que $A$, $B$, e $C$ sejam conjuntos e $\mathbb{U}$ \'e o conjunto universal. Prove que:
\begin{enumerate}[a)]
\item $A\uni\varnothing=A$.

{\bf{\it Resposta:} Primeiro mostremos que ${\bf A\uni\varnothing\subseteq A}$. Seja ${\bf x\in A\uni\varnothing}$. Ent\ao ${\bf x\in A}$ ou ${\bf x\in\varnothing}$. Mas ${\bf x\notin\varnothing}$, logo ${\bf x\in A}$, assim ${\bf A\uni\varnothing\subseteq A}$. Agora, suponha que ${\bf x\in A}$. Ent\ao ${\bf x\in A}$ ou ${\bf x\in\varnothing}$ portanto ${\bf x\in A\uni\varnothing}$. Portanto, ${\bf A\uni\varnothing =A}$.}
\item $A\inter\varnothing=\varnothing$.
\item $A-\varnothing=A$.
\item $A\uni\mathbb{U}=\mathbb{U}$.
\item $A\inter\mathbb{U}=A$.
\item $A\uni A^C=\mathbb{U}$.
\item $A\inter A^C=\varnothing$.
\item $A-A=\varnothing$.
\item $A-B\subseteq A$.

{\bf{\it Resposta:} Seja ${\bf x\in A-B}$, logo ${\bf x\in A}$ e ${\bf x\notin B}$. Assim, ${\bf x\in A}$ e portanto ${\bf A-B\subseteq A}$.}

\item $A\inter B \subseteq A$.
\item $A\uni B \supseteq A$.
\item $A\inter B\subseteq A\uni B$.

{\bf{\it Resposta:} Seja ${\bf x\in A\inter B}$, ent\ao ${\bf x\in A}$ e ${\bf x\in B}$. Logo, ${\bf x\in A}$ ou ${\bf x\in B}$ que implica que ${\bf A\inter B\subseteq A\uni B}$.}

\item $(A^C)^C=A$.

{\bf{\it Resposta:} Seja ${\bf x\in(A^C)^C}$, portanto ${\bf x\notin A^C}$. Mas, ${\bf x\notin A^C}$ implica que ${\bf x\in A}$ e assim, ${\bf (A^C)^C\subseteq A}$. Agora, seja ${\bf x\in A}$, portanto ${\bf x\notin A^C}$ e assim, ${\bf x\in(A^C)^C}$. Logo, ${\bf A\subseteq (A^C)^C}$.}

\item $(A\uni B)^C = A^C\inter B^C$.
\item $(A\inter B)^C = A^C\uni B^C$.

{\bf{\it Resposta:} Seja ${\bf x\in (A\inter B)^C}$, assim ${\bf x\notin A\inter B}$ ou seja, ${\bf x\notin A}$ ou ${\bf x\notin B}$, portanto ${\bf x\in A^C}$ ou ${\bf x\in B^C}$. Logo, ${\bf x\in A^C\uni B^C}$. Por outro lado, seja ${\bf x\in A^C\uni B^C}$, assim ${\bf x\notin A}$ e ${\bf x\notin B}$, portanto ${\bf x\notin A\inter B}$ que implica que ${\bf x\in (A\inter B)^C}$.}

\item $A\uni(B-A)=A\uni B$.

{\bf{\it Resposta:} Seja ${\bf x\in A\uni(B-A)}$. Se ${\bf x\in A}$, o teorema estar\'a demonstrado, logo suponha que ${\bf x\in B-A}$. Ent\ao ${\bf x\in B}$ e ${\bf x\notin A}$. Assim, ${\bf x\in A\uni B}$ e portanto, ${\bf A\uni(B-A)\subseteq A\uni B}$. Agora, suponha que ${\bf x\in A\uni B}$. Ent\ao ${\bf x\in A}$ ou ${\bf x\in B}$. Se ${\bf x\in A}$, o resultado est\'a provado, portanto suponha que ${\bf x\notin A}$. Ent\ao ${\bf x\in B}$ e consequentemente temos ${\bf x\in B-A}$ e assim completamos a demonstra\caoi.}
\item $(A\uni B)-(A\inter B)=(A-B)\uni(B-A)$.

\item $A-(B\uni C)=(A-B)\inter(A-C)$.

{\bf{\it Resposta:} Seja ${\bf x\in A-(B\uni C)}$. Assim, ${\bf x\in A}$ e ${\bf x\notin B\uni C}$. Portanto, ${\bf x\in A}$ e ${\bf x\notin B}$ e ${\bf x\notin C}$. Logo, ${\bf x\in A-B}$ e ${\bf x\in A-C}$. Portanto, ${\bf x\in (A-B)\inter(A-C)}$. Agora, seja ${\bf x\in (A-B)\inter(A-C)}$, ent\ao ${\bf x\in A-B}$ e ${\bf x\in A-C}$. Logo, ${\bf x\in A}$ e ${\bf x\notin B}$ e, ${\bf x\in A}$ e ${\bf x\notin C}$. Portanto, ${\bf x\in A}$ e ${\bf x\notin B\uni C}$ e assim $x\in A\inter(B\uni C)$.}

\item $A\uni(B\inter C)=(A\uni B)\inter(A\uni C)$.
\item $A\inter(B\uni C)=(A\inter B)\uni(A\inter C)$.
\end{enumerate}

%excercicio6
\item \label{conjuntos6} Suponha que $A$, $B$, $C$ e $D$ sejam conjuntos e $\mathbb{U}$ \'e o conjunto universal. Para cada dos seguintes teoremas enuncie as hip\'oteses e conclus\aoi e indique a forma de uma demonstra\cao direta. Ent\ao escreva a demonstra\cao para cada um deles.
\begin{enumerate}[a)]
\item $A\subseteq\varnothing\leftrightarrow A=\varnothing$.
\item $A\subset B\ee B\subset C\to A\subset C$.
\item $A\subseteq B\leftrightarrow A\uni B=B$. 

{\bf{\it Resposta:} Primeiro mostremos que ${\bf A\subseteq B\to A\uni B=B}$. Seja ${\bf x\in B}$. Ent\ao ${\bf x\in A}$ ou ${\bf x\in B}$ assim ${\bf x\in A\uni B}$. Agora, seja ${\bf x\in A\uni B}$. Portanto, ${\bf x\in A}$ ou ${\bf x\in B}$. Se ${\bf x\in B}$ o teorema estar\'a pronto, portanto, suponha ${\bf x\in A}$. Como ${\bf A\subseteq B}$ temos ${\bf x\in B}$ e assim ${\bf A\uni B=B}$. Agora, mostremos que ${\bf A\uni B=B\to A\subseteq B}$. Seja ${\bf x\in A}$. Ent\ao ${\bf x\in A\uni B}$ e, portanto, ${\bf x\in B}$ pois ${\bf A\uni B=B}$.}

\item $A\subseteq B\leftrightarrow \mathbb{P}(A)\subseteq\mathbb{P}(B)$.
\item $A\subseteq B^C\leftrightarrow A\inter B=\varnothing$.

{\bf{\it Resposta:} Nessa demonstração temos uma bicondicional então, para demonstrar ${\bf A\subseteq B^C\leftrightarrow A\inter B=\varnothing}$, temos que demonstrar ${\bf A\subseteq B^C\to A\inter B=\varnothing}$ \\ e ${\bf A\inter B=\varnothing\to A\subseteq B^C}$.

Para demonstrar que ${\bf A\subseteq B^C\to A\inter B=\varnothing}$, a hipótese é ${\bf A\subseteq B^C}$ e a conclusão é ${\bf A\inter B=\varnothing}$. Para mostrar que um determinado conjunto é vazio, assumimos que este conjunto tem um elemento ${\bf x}$ e usando a hipótese chegamos a uma contradição. De fato, seja ${\bf x\in A\inter B}$, portanto ${\bf x\in A}$ e ${\bf x\in B}$. A hipótese diz que ${\bf A\subseteq B^C}$, portanto ${\bf x\in B^C}$, logo ${\bf x\notin B}$, contradição. Portanto, ${\bf A\inter B=\varnothing}$.

Para demostrar que ${\bf A\inter B=\varnothing\to A\subseteq B^C}$, a hipótese é ${\bf A\inter B=\varnothing}$ e a conclusão é ${\bf A\subseteq B^C}$. Assim, seja ${\bf x\in A}$ como, por hipótese, ${\bf A\inter B=\varnothing}$, então ${\bf x\notin B}$ e assim ${\bf x\in B^C}$. Portanto, ${\bf A\subseteq B^C}$. }

\item $(A\uni B=C\ee A\inter B=\varnothing)\to B=C-A$.

{\bf{\it Resposta:} A hipótese desta demonstração é ${\bf A\uni B=C}$ e ${\bf A\inter B=\varnothing}$. Queremos demonstrar que ${\bf B=C\backslash A}$. Para isso, devemos mostrar duas inclusões ${\bf B\subseteq C\backslash A}$ e ${\bf C \backslash A\subseteq B}$. Para mostrar a primeira inclusão, suponha que ${\bf x\in B}$, logo ${\bf x\in A\uni B}$. Como ${\bf A\uni B=C}$, então ${\bf x\in C}$. Novamente, como ${\bf x\in B}$ e, por hipótese, ${\bf A\inter B=\varnothing}$, temos que ${\bf x\notin A}$. Assim, ${\bf x\in C}$ e ${\bf x\notin A}$, portanto ${\bf x\in C\backslash A}$. Com isso, provamos que ${\bf B\subseteq C\backslash A}$. Para provar a segunda inclusão, seja ${\bf x\in C\backslash A}$, portanto ${\bf x\in C}$ e ${\bf x\notin A}$. Do fato que ${\bf x\notin A}$ e da hipótese ${\bf A\inter B=\varnothing}$, concluímos que ${\bf x\in B}$, logo ${\bf C \backslash A\subseteq B}$ e consequentemente ${\bf B=C \backslash A}$.}

\item $(A\subseteq C\ee B\subseteq C)\leftrightarrow A\uni B\subseteq C$.
\item $(A\subseteq C\ee B\subseteq D)\to(A\uni B\subseteq C\uni D)$.
\item $[(A\inter C=A\inter B)\ee(A\uni C=A\uni B)]\to B=C$.
\item $A\subseteq B\leftrightarrow A^C\uni B=\mathbb{U}$.
\item $A-B\subseteq B\leftrightarrow A\subseteq B$.

{\bf{\it Resposta:} Primeiro mostremos que ${\bf A-B\subseteq B\rightarrow A\subseteq B}$. De fato, suponha ${\bf A-B\subseteq B}$ e ${\bf A\not\subseteq B}$, ou seja, ${\bf \exists x\in A}$ tal que ${\bf x\notin B}$, portanto ${\bf x\in A-B}$, como ${\bf A-B\subseteq B}$, ent\ao ${\bf x\in B}$, contradi\caoi. Desejamos agora demonstrar que ${\bf A\subseteq B\rightarrow A-B\subseteq B}$. De fato, seja ${\bf x\in A-B}$, ent\ao ${\bf x\in A}$ e ${\bf x\notin B}$, mas como ${\bf A\subseteq B}$, ent\ao ${\bf x\in B}$, contradi\caoi. Logo, ${\bf A-B=\varnothing\subseteq B}$.}


\item $A\inter B=\mathbb{U}\leftrightarrow A=B=\mathbb{U}$.
\item $A\uni B\neq\varnothing\leftrightarrow A\neq\varnothing \ou B\neq\varnothing $.
\item $\mathbb{P}(A)=\mathbb{P}(B)\to A=B$.

{\bf{\it Resposta:} Se ${\bf A\neq B}$, ent\ao ${\bf \exists x\in A}$ tal que ${\bf x\notin B}$, logo ${\bf \{x\}\in \mathbb{P}(A)}$ e ${\bf \{x\}\notin \mathbb{P}(B)}$, logo ${\bf \mathbb{P}(A)\neq\mathbb{P}(B)}$.}
\end{enumerate}

%excercicio7
\item {\bf Acredite se quiser:}  

\noindent \textit{\textbf{Conjectura:}} Seja $A$ e $B$ conjuntos tais que $A\subseteq B$. Ent\ao $A-B=\varnothing$.

\noindent \textit{\textbf{``Demonstra\caoi'':}} Suponha que $A$ e $B$ sejam conjuntos com $A\subseteq B$. Seja $x\in A-B$. Ent\ao $x\in B$ e $x\notin A$. Mas $A\subseteq B$ assim $x\notin A$ implica $x\notin B$, uma contradi\caoi. Logo, $A-B=\varnothing$.

\noindent \textit{\textbf{``contra-exemplo'':}} Seja $A=\{1,2,3\}$, $B=\{2,3\}$. Ent\ao $A\subseteq B$ mas $A-B\neq\varnothing$.

{\bf{\it Resposta:} Dica: A conjectura \'e verdadeira.}

%excercicio8
\item {\bf Acredite se quiser:}  

\noindent \textit{\textbf{Conjectura:}} Sejam $A,B,C,D$ conjuntos com $A\subset C$ e $B\subset D$. Ent\ao $A\uni B \subset C\uni D$. 

\noindent \textit{\textbf{``Demonstra\caoi'':}} Suponha $A,B,C,D$ sejam conjuntos tais que $A\subset C$ e $B\subset D$. Seja $x\in A\uni B$. Ent\ao $x\in A$ ou $x\in B$. Suponha que $x\in A$. Ent\aoi, como $A\subset C$, $x\in C$. Assim, $x\in C\uni D$. Se $x\in B$, tamb\'em obtemos $x\in C\uni D$, pois $B\subset D$. Portanto, $A\uni B\subset C\uni D$.

\noindent \textit{\textbf{``contra-exemplo'':}} Sejam $A=\{1\}$, $B=\{2\}$ e $C=D=\{1,2\}$. Ent\ao $A\subset B$, $C\subset D$ mas $A\uni B \not\subset C\uni D$.

{\bf{\it Resposta:} Dica: A conjectura \'e falsa.}

%excercicio9
\item Com os exerc\ih cios acredite se quiser existem oito possibilidades: a conjectura \'e verdadeira ou falsa, a demonstra\cao \'e correta ou n\ao e o contra-exemplo \'e correto ou n\aoi. Qual destas oito possibilidades n\ao podem ocorrer?

%excercicio10
\item Sejam $A$, $B$ e $C$ conjuntos. Mostre usando alguns resultados dos exerc\ih cios \ref{conjuntos5} e \ref{conjuntos6} ao inv\'es do nosso m\'etodo usual de ir ao princ\ih pio com as defini\cois:
\begin{enumerate}[a)]
\item $A\subseteq B \to A\inter B^C=\varnothing$.

{\bf{\it Resposta:} ${\bf A\subseteq B \espaco\xrightarrow[]{6j}\espaco A^C\uni B=\mathbb{U} \espaco\xrightarrow[5m]{5o}\espaco (A\inter B^C)^C=\mathbb{U} \espaco\xrightarrow[]{5m}\espaco A\inter B^C=\varnothing.}$}

\item $A\uni(A\inter B)=A$.

{\bf{\it Resposta:} ${\bf A\uni(A\inter B)\stackrel{\text{5j}}{\subseteq}A\uni A=A}$ e ${\bf A\stackrel{\text{5k}}{\subseteq} A\uni(A\inter B)}$.}

\item $A\inter(A^C\uni B)=A\inter B$. 

{\bf{\it Resposta:} ${\bf A\inter(A^C\uni B)\stackrel{\text{5t}}{=}(A\inter A^C)\uni(A\inter B)\stackrel{\text{5g}}{=}\varnothing\uni(A\inter B)\stackrel{\text{5a}}{=}A\inter B}$.}

\item $A\inter C=\varnothing \to A\inter(B\uni C)=A\inter B$.

{\bf{\it Resposta:} ${\bf A\inter(B\uni C)\stackrel{\text{5t}}{=}(A\inter B)\uni(A\inter C)\stackrel{\text{hip.}}{=}(A\inter B)\uni\varnothing\stackrel{\text{5a}}{=}A\inter B}$.}

\item $A\subseteq B \to A=B-(B-A)$.

{\bf{\it Resposta:} ${\bf B-(B-A)\stackrel{\text{Teo}}{=}B-(B\inter A^C)\stackrel{\text{Teo}}{=}B\inter(B\inter A^C)^C\stackrel{\text{5o}}{=}B\inter(B^C\uni A)\stackrel{\text{5t}}{=}}$ \\ ${\bf (B\inter B^C)\uni(B\inter A)\stackrel{\text{5g}}{=}\varnothing\uni(A\inter B)\stackrel{\text{5a}}{=}A\inter B\stackrel{\text{Hip}}{=}A}$.}
\end{enumerate}

%excercicio11
\item Suponha que qualquer cole\cao se objetos pudesse ser um conjunto. Ent\ao poder\ih amos ter o ``conjunto de todos os conjuntos.'' Considere o subconjunto $S$ do conjunto de todos os conjuntos dados por
\[
S=\{A:A\notin A\}.
\]
Assim $S$ \'e o conjunto de todos os conjuntos que n\ao s\ao elementos de si mesmos.
\begin{enumerate}[a)]
\item D\^e exemplos de dois conjuntos que s\ao elementos de $S$.

{\bf{\it Resposta:} ${\bf A=\{1,2,3,4,\ldots\}}$.}

\item D\^e exemplos de dois conjuntos que n\ao s\ao elementos de $S$.

{\bf{\it Resposta:} ${\bf A=\{A,\{1\},\{2\},\{3\},\{4\},\ldots\}}$.}

\item Mostre que $S\notin S$.

{\bf{\it Resposta:} Se ${\bf S\in S}$, ent\ao ${\bf S\notin S}$, absurdo.}

\item Mostre que $S\notin S^C$.

{\bf{\it Resposta:} Se ${\bf S\in S^C}$, ent\ao ${\bf S\notin S}$, logo ${\bf S\in S}$ e portanto, ${\bf S\notin S^C}$, absurdo.}

\end{enumerate}

%excercicio12
\item Sejam $A$ e $B$ conjuntos. Considere as seguintes conjecturas. Prove as verdadeiras e d\^e contra-exemplos para as falsas.
\begin{enumerate}[a)]
\item $\mathbb{P}(A)\uni\mathbb{P}(B)\subseteq\mathbb{P}(A\uni B)$.

{\bf{\it Resposta:} Se ${\bf C\in \mathbb{P}(A)\uni\mathbb{P}(B)\espaco\Rightarrow\espaco C\in \mathbb{P}(A) \text{ ou } C\in \mathbb{P}(B) \espaco\Rightarrow\espaco C\subseteq A \text{ ou } C\subseteq B  \espaco\Rightarrow\espaco}$ \\ ${\bf C\subseteq A\uni B \espaco\Rightarrow\espaco C\in \mathbb{P}(A\uni B)}$.}

\item $\mathbb{P}(A)\inter\mathbb{P}(B)\subseteq\mathbb{P}(A\inter B)$.
\item $\mathbb{P}(A\uni B)\subseteq \mathbb{P}(A)\uni\mathbb{P}(B)$. 

{\bf{\it Resposta:} A conjectura \'e falsa, considere ${\bf A=\{a\}}$, logo ${\bf \mathbb{P}(A)=\{\varnothing,\{a\}\}}$ e ${\bf B=\{b\}}$, logo ${\bf \mathbb{P}(B)=\{\varnothing,\{b\}\}}$. Assim, ${\bf A\uni B=\{a,b\}}$ e ${\bf \mathbb{P}(A\uni B)=\{\varnothing,\{a\},\{b\},\{a,b\}\}}$.}

\item $\mathbb{P}(A\inter B)\subseteq \mathbb{P}(A)\inter\mathbb{P}(B)$.
\item $\mathbb{P}(A\inter B)\subseteq\mathbb{P}(A\uni B)$.
\end{enumerate}

%excercicio13
\item Sejam $A$, $B$, $C$ e $D$ conjuntos tais que $A\subset C$ e $B\subset D$. Demonstre ou d\^e um contra-exemplo para a conjectura: $A\inter B\subset C\inter D$.

%excercicio14
\item Sejam $A$ e $B$ conjuntos de proposi\cois. Dizemos que $A$ \'e mais forte que $B$, denotado por
\[
A\Longrightarrow B,
\]
e se somente se,
\[
\forall p\in B,\exists q_1,q_2,\ldots,q_n \in A \mid  (q_1\ee q_2\ee \ldots \ee q_n)\Rightarrow p.
\]
Assim, se $A$ \'e mais forte que $B$, ent\ao toda proposi\cao pertencente a $B$ \'e uma consequ\^encia l\'ogica de uma conjun\cao de proposi\coes pertencentes a $A$. Por exemplo, se
\begin{equation*}
 \begin{aligned}
A&=\{p\ou q, \nao q, r\to q\},\\
B&=\{p, \nao r, \nao q, s\ou\nao q\},\\
C&=\{p\ou q,q\},
 \end{aligned}
\end{equation*}
ent\ao $A\Longrightarrow B$ mas $A\not\Longrightarrow C$.
\begin{enumerate}[a)]
\item Escreva em s\ih mbolos e em Portug\^es $\nao(A\Longrightarrow B)$.
\item D\^e (outro) exemplo de conjuntos de proposi\coes $A,B,C$ tais que $A\Longrightarrow B$ mas $A\not\Longrightarrow C$. 
\item Mostre que para qualquer conjunto de proposi\coes A, $A\Longrightarrow \varnothing$.
\item Mostre que para qualquer conjunto de proposi\coes A, $A\Longrightarrow A$.
\item Mostre que se $A$ e $B$ s\ao quaisquer conjuntos de proposi\cois, $A\subseteq B$ implica $B\Longrightarrow A$.
\item Se $A\Longrightarrow B$ e $B\Longrightarrow A$, seria este o caso que $A=B$?
\item Se $A \Longrightarrow B$ e $C\Longrightarrow D$, ent\ao $A\uni C\Longrightarrow B\uni D$?
\end{enumerate}
\end{enumerate}
%%%%%%%%%%%%%%%%%%%%%%%%%%%%%%%%%%%%%%%%%%%%%%%%%%%%%%%%%%%%%%%%%%%%%%%%
%%%%%%%%%%%%%%%%%%%%%%%%% secao 2.2 %%%%%%%%%%%%%%%%%%%%%%%%%%%%%%%%%%%%
%%%%%%%%%%%%%%%%%%%%%%%%%%%%%%%%%%%%%%%%%%%%%%%%%%%%%%%%%%%%%%%%%%%%%%%%
\paragraph{Exerc\ih cios \ref{conjuntosverdade}}

\begin{enumerate}[{\bf 1.}]
%excercicio1
\item Sejam $D=\{1,2,3,4,5,6,7,8\}$, $p(x)$ ``x \'e par'', $q(x)$, ``$x$ \'e \ih mpar'' e $r(x)$ ``$x$ \'e um n\'umero primo.'' Encontre:
\begin{enumerate}[a)]
\item O conjunto verdade para $p(x)\ee q(x)$.
\item O conjunto verdade para $r(x)\to\nao p(x)$.  {\bf{\it Resposta:} ${\bf \{1,3,4,5,6,7,8\}}$.}
\item $P^C\uni Q$.
\item A fun\cao proposicional que tem $\{1,2,3,5,7\}$ como seu conjunto verdade. {\bf{\it Resposta:} ${\bf r(x)}$.}
\item O conjunto verdade de $\exists x \in D \mid  r(x)\to p(x)$.
\item O conjunto verdade de $\forall x\in D, p(x)\ou q(x)$. {\bf{\it Resposta:} Verdade.}
\item O conjunto verdade de $[\forall x\in D, p(x)]\ou[\forall x\in D, q(x)]$. 
\item O conjunto verdade de $[\forall x\in D, q(x)]\to[\forall x\in D, r(x)]$.
\end{enumerate}

%excercicio2
\item Encontre o conjunto verdade para a fun\cao proposicional ``$x^2-x-2\leq 0$.'' Tome $\mathbb{R}$ como dom\ih nio.

{\bf{\it Resposta:} ${\bf [-1,2]}$.}

%excercicio3
\item Considere os seguintes pares de proposi\cois. Para cada proposi\cao do par determine as condi\coes para $P,Q$ que garantam que ela seja verdade. E mostrar que sempre que a segunda (de cada par) seja verdade, a primeira dever\'a ser verdade. D\^e um exemplo para mostrar que a primeira pode ser verdade e a segunda falsa.
\begin{enumerate}[a)]
\item $[\forall x\in D, p(x)\ou q(x)]; [\forall x\in D, p(x) \ou \forall x \in D, q(x)]$.

{\bf{\it Resposta:} A primeira se\'a verdade quando ${\bf P\uni Q=D}$; a segunda ser\'a verdade quando ${\bf P=D}$ ou ${\bf Q=D}$. Certamente a segunda condi\cao implica a primeira. Se ${\bf D=\mathbb{N}}$, ${\bf p(x)}$, ${\bf p(x)}$ \'e ``${\bf x}$ \'e par'' e ${\bf q(x)}$ \'e ``${\bf x}$ \' \ih mpar'' ent\ao a primeira \'e verdade e a segunda \'e falsa.}

\item $[\exists x\in D\mid  p(x)\ee \exists x\in D\mid  q(x)];[\exists x\in D\mid  p(x)\ee q(x)]$.

{\bf{\it Resposta:} A primeira se\'a verdade quando ${\bf P\neq\varnothing}$ e ${\bf Q\neq\varnothing}$; a segunda ser\'a verdade quando ${\bf P\inter Q\neq\varnothing}$ . Certamente a segunda condi\cao implica a primeira. Se ${\bf D=\mathbb{N}}$, ${\bf p(x)}$, ${\bf p(x)}$ \'e ``${\bf x}$ \'e par'' e ${\bf q(x)}$ \'e ``${\bf x}$ \' \ih mpar'' ent\ao a primeira \'e verdade e a segunda \'e falsa.}

\item $[\exists x\in D\mid  p(x)\to q(x)];[\exists x\in D\mid  p(x) \to \exists x\in D\mid  q(x)]$.
\end{enumerate}

%excercicio4
\item {\bf Acredite se quiser:}  

\noindent \textit{\textbf{Conjectura:}} Sejam $A,B$ conjuntos com $B\subseteq A$ e $p$ uma fun\cao proposicional. Ent\ao $\forall x\in A, p(x)$ implica $\forall x\in B, p(x)$. {\bf{\it Resposta:} Verdadeira.}

\noindent \textit{\textbf{``Demonstra\caoi'':}} Suponha que $A,B$ sejam conjuntos como acima e que a implica\cao seja falsa. Ent\ao existe um $z\in B$ tal que $p(z)$ seja falsa. Como $B\subseteq A, \espaco z\in A$. Mas isto significa $\forall x \in A, p(x)$ \'e falso, uma contradi\caoi. {\bf{\it Resposta:} Verdadeira. A demonstra\cao \'e indireta, ${\bf (p\to q) \Longleftrightarrow \nao(p\to q)\to}$ contradi\caoi.}

\noindent \textit{\textbf{``contra-exemplo'':}} Sejam $p(x)$ ``$x<2$,'' $A=\{1,2,3\}$ e $B=\varnothing$. Ent\ao $B\subseteq A$, $\forall x \in A, p(x)$ \'e falso e $\forall x \in B, p(x)$ \'e verdadeiro. {\bf{\it Resposta:} Falsa. Para um contra-exemplo ser v\'alido, devemos encontrar um hip\'otese correta com uma conclus\ao falsa. N\ao \'e o caso aqui.}
\end{enumerate}
%%%%%%%%%%%%%%%%%%%%%%%%%%%%%%%%%%%%%%%%%%%%%%%%%%%%%%%%%%%%%%%%%%%%%%%%
%%%%%%%%%%%%%%%%%%%%%%%%% secao 2.3 %%%%%%%%%%%%%%%%%%%%%%%%%%%%%%%%%%%%
%%%%%%%%%%%%%%%%%%%%%%%%%%%%%%%%%%%%%%%%%%%%%%%%%%%%%%%%%%%%%%%%%%%%%%%%
\paragraph{Exerc\ih cios \ref{relacoes}}

\begin{enumerate}[{\bf 1.}]
%excercicio1
\item Sejam $A=\{a,b,c\}$, $B=\{1,2\}$ e $C=\{4,5,6\}$
\begin{enumerate}[a)]
\item Liste os elementos de $A\times B$, $B\times A$, e $A\times C$.
\item D\^e exemplos de rela\coes de $A$ em $B$ e de $B$ em $A$, cada uma dos quais tem quatro elementos.
\item D\^e exemplo de uma rela\cao sim\'etrica em $C$ que tenha tr\^es elementos.
\end{enumerate}

%excercicio2
\item Suponha $A=\{1,2,3\}$. Para cada uma das rela\coes dadas abaixo, liste os elementos de $R$, encontre $\Dom(R)$ e $\Ima(R)$ de diga quais das propriedades da defini\cao \ref{reldef1} $R$ tem:
\begin{enumerate}[a)]
\item $R$ \'e a rela\cao $<$ em A. {\bf{\it Resposta:} ${\bf \{(1,2),(1,3),(2,3)\}}$, ${\bf R}$ \'e transitiva, antissim\'etrica, irreflexiva, completa e assim\'etrica.}
\item $R$ \'e a rela\cao $\geq$ em A.
\item $R$ \'e a rela\cao $\subset$ em $\mathbb{P}(A)$.
\end{enumerate}

%excercicio3
\item Suponha que $A,B,C,D$ sejam conjuntos. Prove ou d\^e contra-exemplos para as seguintes conjecturas:
\begin{enumerate}[a)]
\item $A\times(B\uni C)= (A\times B)\uni(A\times C)$.
\item $A\times(B\inter C)=(A\times B)\inter(A\times C)$.
\item $(A\times B)\inter(A^C\times B)=\varnothing$.
\item $(A\subseteq B\ee C\subseteq D)\to A\times C\subseteq B\times D$. {\bf{\it Resposta:} Verdade. Seja ${\bf (x,y)\in A\times C}$. Ent\ao ${\bf x\in A}$ e ${\bf y\in C}$. Mas como ${\bf A\subseteq B}$ e ${\bf C\subseteq D}$ temos que ${\bf x\in B}$ e ${\bf y\in D}$ Assim ${\bf (x,y)\in B\times D}$.}
\item $A\uni(B\times C)=(A\uni B)\times(A\uni C)$.
\item $A\inter(B\times C)=(A\inter B)\times(A\inter C)$.
\item $(A\times B)\inter(C\times D)=(A\inter C)\times(B\inter D)$.
\item $A\times(B-C)=A\times B-A\times C$.
\end{enumerate}

%excercicio4
\item Suponha que $R$ seja uma rela\cao em um conjunto n\ao vazio $A$. D\^e a forma de demonstra\cao direta para:
\begin{enumerate}[a)]
\item $R$ \'e antissim\'etrica.
\item $R$ \'e irreflexiva. {\bf{\it Resposta:} Seja ${\bf x\in A}$ \ldots ent\ao ${\bf (x,x)\notin R}$.}
\item $R$ \'e assim\'etrica.
\end{enumerate}

%excercicio5
\item Sejam $A=\{1,2,3,4\}$ e $R$ uma rela\cao em $a$. D\^e um exemplo utilizando a representa\cao gr\'afica do que segue, assim como foi feito anteriormente para $R$ reflexiva e sim\'etrica.
\begin{enumerate}[a)]
\item Transitiva
\item Irreflexiva
\item Assim\'etrica
\item Rela\cao de equival\^encia
\end{enumerate}

%excercicio6
\item Sejam $A,B$ conjuntos com $R,S$ rela\coes de $A$ para $B$. Demonstre:
\begin{enumerate}[a)]
\item $\Dom(R\uni S)=\Dom(R)\uni \Dom(S)$.
\item $\Dom(R\inter S)\subseteq \Dom(R)\inter \Dom(S)$ e d\^e um exemplo para mostrar que a igualdade n\ao \'e v\'alida.
\item $\Ima(R\uni S)=\Ima(R)\uni \Ima(S)$
\item $\Ima(R\inter S)\subseteq \Ima(R)\inter \Ima(S)$ e d\^e um exemplo para mostrar que a igualdade n\ao \'e v\'alida.
\end{enumerate}

%excercicio7
\item Seja $A$ um conjunto n\ao vazio. Mostre que
\begin{enumerate}[a)]
\item Se $R=A\times A$ ent\ao $R$ \'e reflexiva, sim\'etrica, transitiva e completa. O que poderia se dizer se $R$ fosse assim\'etrica ou antissim\'etrica?  {\bf{\it Resposta:} (Resposta parcial) Como  ${\bf R=\{(x,y): x,y\in A\}}$, certamente ${\bf \{(x,x): x\in A\}\subseteq R}$. Se $A$ cont\'em mais de um elemento, ${\bf R}$ n\ao ser\'a assim\'etrica ou antissim\'etrica.}
\item Se $R=\varnothing$ ent\ao $R$ \'e sim\'etrica, transitiva, assim\'etrica, antissim\'etrica, irreflexiva mas n\ao reflexiva.
\item Se $R=\{(a,a):a\in A\}$ ent\ao $R$ \'e uma rela\cao de equival\^encia e tamb\'em \' antissim\'etrica mas n\ao assim\'etrica.
\end{enumerate}

%excercicio8
\item Referente aos exemplos dados anteriormente nessa se\caoi, mostre que:
\begin{enumerate}[a)]
\item Exemplo \ref{relex1} \'e assim\'etrico mas n\ao reflexiva. {\bf{\it Resposta:} ${\bf R}$ n\ao \'e reflexiva pois ningu\'em \'e pai ou m\~ae de si mesmo. Se ${\bf xRy}$ ent\ao ${\bf y}$  \'e pai/m\~ae de ${\bf x}$ o que significa que ${\bf x}$ n\ao \'e pai/m\~ae de ${\bf y}$ logo temos ${\bf \nao(xRy)}$ \'e verdade e portanto ${\bf R}$ \'e assim\'etrica.} 
\item Exemplos \ref{relex3}, \ref{relex11} s\ao rela\coes de equival\^encia.
\item Exemplo \ref{relex3}, \ref{relex4}, \ref{relex9} s\ao ordens parciais.
\item Exemplo \ref{relex2}, \ref{relex10} n\ao s\ao completas.
\item Exemplo \ref{relex9} \'e uma ordem total.
\end{enumerate}

%excercicio9
\item Seja $R$ a rela\cao de equival\^encia dada no exemplo \ref{relex11} acima. Determine todos os elementos nestes conjuntos:
\begin{enumerate}[a)]
\item $\{x:xR1\}$. {\bf{\it Resposta:} ${\bf \{1,6,11,16,\ldots\}}.$}
\item $\{x:xR2\}$.
\item $\{x:xR7\}$.
\end{enumerate}

%excercicio10
\item Seja $R$ a rela\cao $|$ em $\mathbb{Z}$ descrita no exemplo \ref{relex10} acima.
\begin{enumerate}[a)]
\item Liste tr\^es elementos de $\mathbb{Z}\times\mathbb{Z}$ que n\ao sejam elementos de $R$.
\item Quais dos elementos $(0,0),(0,1),(1,0)$ s\ao elementos de $R$? {\bf{\it Resposta:} (Resposta parcial) ${\bf (0,1)\notin R}$.}
\item Prove o seguinte:
\begin{enumerate}[i)]
\item $\forall n\in \mathbb{Z}, n|0$.
\item $\forall n\in \mathbb{Z}, 0|n\to n=0$. 
\item $\forall a,b,c\in \mathbb{Z}, (a|b\ee a|c)\to a|(b+c)$.
\item $\forall a,b,c\in \mathbb{Z}, a|b\to a|bc$.
\end{enumerate}
\end{enumerate}

%excercicio11
\item Sejam $R,S$ rela\coes em um conjunto n\ao vazio $A$. Demonstre ou d\^e contra-exemplos para as seguintes conjecturas:
\begin{enumerate}[a)]
\item $R$ \'e completa $\to$ $R$ \'e reflexiva. {\bf{\it Resposta:} Falsa. Seja ${\bf A=\{1,2\}}$ e ${\bf R=\{(1,2),(2,1)\}}$.}
\item $R$ \'e transitiva e irreflexiva $\to$ $R$ \'e assim\'etrica.
\item $R$ \'e reflexiva $\to$ $R$ n\ao \'e assim\'etrica. 
\item $R$ \'e assim\'etrica $\to$ $R$ n\ao \'e reflexiva.
\item $\Dom(R)\inter \Ima(R)=\varnothing\to$ $R$ \'e transitiva, antissim\'etrica, irreflexiva e assim\'etrica.
\item $R$ uma rela\cao de ordem parcial estrita $\to$ $R$ \'e antissim\'etrica e assim\'etrica. 
\item $R$ n\ao reflexiva $\to$ $R$ \'e irreflexiva.
\item $R$ e $S$ sim\'etrica $\to$ $R\inter S$ sim\'etrica. {\bf{\it Resposta:} Seja ${\bf (x,y)\in R\inter S}$. Ent\ao ${\bf (x,y)}$ est\'a em ambos ${\bf R}$ e ${\bf S}$. Como ambos s\ao sim\'etricos, temos que ${\bf (y,x)}$ pertencem a ambos ${\bf R}$ e ${\bf S}$ logo, ${\bf (y,x)\in R\inter S}$.}
\item $R$ ou $S$ sim\'etrica $\to$ $R\inter S$ sim\'etrica.
\item $R$ e $S$ sim\'etrica $\to$ $R\uni S$ sim\'etrica.
\item $R$ ou $S$ reflexiva $\to$ $R\uni S$ reflexiva.
\item $R$ e $S$ transitiva $\to$ $R\uni S$ transitiva.
\item $R$ e $S$ transitiva $\to$ $R\inter S$ transitiva.
\end{enumerate}

%excercicio12
\item D\^e exemplos (se poss\ih vel) de rela\coes que sejam:
\begin{enumerate}[a)]
\item Reflexiva e sim\'etrica mas n\ao transitiva.

{\bf{\it Resposta:} Sejam ${\bf A=\{1,2,3\}}$, ${\bf R=\{(1,1),(2,2),(3,3),(2,3),(3,2),(1,2),(2,1)\}}$. Note que, ${\bf (1,2)\in R}$ e ${\bf (2,3)\in R}$ mas ${\bf (1,3)\notin R}$.}

\item Sim\'etrica e transitiva mas n\ao reflexiva.
\item Assim\'etrica mas n\ao antissim\'etrica.
\item Sim\'etrica e antissim\'etrica.
\item Nem reflexiva, nem irreflexiva.
\end{enumerate}

%excercicio13
\item\label{relexer13} Se $R$ \'e uma rela\cao de $A$ para $B$ e $C\subseteq A$, definimos {\it restri\cao de $R$ em $C$}, denotada por $R|_C$, como
\[
\{(x,y)\in R: x\in C\}.
\]
\begin{enumerate}[a)]
\item Sejam $A=B=\{1,2,3,4\}$ e $C=\{2,4\}$. Seja $R$ a rela\cao $<$ de $A$ em $B$. Encontre $R$ e $R|_C$. 
\item Se $R$ \'e uma rela\cao de $A$ em $B$ e $C\subseteq A$, mostre que $\Dom(R|_C)=\Dom(R)\inter C$. 
\item Se $R$ \'e uma rela\cao de $A$ em $B$ e $B\subseteq A$, $R|_B$ \'e uma rela\cao em $B$? Prove ou de um contra-exemplo. {\bf{\it Resposta:} Dica, esta \'e falsa.}
\end{enumerate}

%excercicio14
\item (Continua\cao do exerc\ih cio \ref{relexer13}) Suponha que $R$ seja uma rela\cao em $A$ com as propriedades listadas abaixo. Se $B\subseteq A$ e $R|_B$ \'e considerada como uma rela\cao em $A$, quais destas propriedades $R|_B$ deve tamb\'em ter? Prove ou d\^e contra-exemplos.
\begin{enumerate}[a)]
\item Sim\'etrica 
\item Transitiva
\item Antisim\'etrica
\end{enumerate}

%excercicio15
\item Suponha que tiv\'essemos definido o par ordenado $(a,b)$ pot
\[
(a,b)=\{\{a\},\{a,b\}\}.
\]
Mostre que com essa defini\cao temos
\[
(a,b)=(c,d)\leftrightarrow(a=c\ee b=d).
\]


%excercicio16
\item Suponha que tiv\'essemos definido ``tripla ordenada'' usando pares ordenados como
\[
(a,b,c)=((a,b),c).
\]
Mostre que a mesma tem a propriedade de ordena\cao desejada, isto \'e:
\[
(a,b,c)=(d,e,f)\textrm{ se e somente se }a=d,b=e,c=f.
\]


%excercicio17
\item Seja $R$ uma ordem total estrita em um conjunto n\ao vazio $A$. Mostre que $R$ tem a propriedade da ``tricotomia,'' isto \'e,
\[
\forall a,b\in A, \textrm{ exatamente um dos seguintes \'e verdade } a=b, aRb, bRa.
\]


%excercicio18
\item Seja $R$ uma rela\cao em um conjunto n\ao vazio $A$. O {\it fecho transitivo} de $R$ \'e a menor rela\cao transitiva contendo $R$, isto \'e, se $S$ \'e o fecho transitivo de $R$, e $T$ \'e uma rela\cao qualquer contendo $R$, ent\ao
\[
R\subseteq S\subseteq T.
\]
Criamos defini\coes similares para os fechos reflexivos e sim\'etricos. Vamos denotar estes fechos por $R_{trans}$, $R_{sim}$ e $R_{ref}$. 
\begin{enumerate}[a)]
\item Se $A$=\{1,2,3,4\} e $R=\{(1,2),(1,4),(2,3)\}$, encontre $R_{trans}$, $R_{sim}$ e $R_{ref}$. {\bf{\it Resposta:} ${\bf R_{sim}=\{(1,2),(1,4),(2,3),(2,1),(4,1),(3,2)\}}$.}
\item Demonstre ou d\^e contra-exemplo para as seguintes conjecturas ($R,S$ s\ao rela\coes em um conjunto n\ao vazio $A$):
\begin{enumerate}[i)]
\item $(R\uni S)_{trans}=R_{trans}\uni R_{trans}$.
\item $(R\inter S)_{trans}=R_{trans}\inter R_{trans}$.
\item $(R\uni S)_{sim}=R_{sim}\uni R_{sim}$.
\item $(R\inter S)_{sim}=R_{sim}\inter R_{sim}$.
\item $(R\uni S)_{ref}=R_{ref}\uni R_{ref}$.
\item $(R\inter S)_{ref}=R_{ref}\inter R_{ref}$.
\end{enumerate}
\item O que pode ser dito sobre os correpondentes conceitow de fechos antissim\'etrico e assim\'etrico?
\end{enumerate}

%excercicio19
\item Seja $R$ uma rela\cao em um conjunto n\ao vazio $A$. Suponha que $R$ \'e assim\'etrica e tamb\'em satifaz a condi\cao (\`as vezes tamb\'em chamada de {\it transitividade negativa}):
\[
\forall x,y\in A, xRz\to (xRy\ou yRz).
\]
\begin{enumerate}[a)]
\item Mostre $R$ \'e transitiva. Tais rela\coes s\ao \`as vezes chamadas de {\it ordens fracas}. 
\item Se $R$ fosse transitiva e assim\'etrica, ela tamb\'em deveria satifazer a condi\cao dada acima? Prove ou d\^e um contra-exemplo.
\end{enumerate}


%excercicio20
\item Seja $R$ uma rela\cao em um conjunto n\ao vazio $A$. Seja $x\in A$. Definimos uma {\it R-classe} de $x$, denotada por $<x>_R$, como
\[
<x>_R:=\{y:yRx\}.
\]
O s\ih mbolo $:=$ significa ``por defini\caoi.''
\begin{enumerate}[a)]
\item Seja $A=\{1,2,3,4\}$ e
\[
R=\{(1,2), (1,3), (2,1), (1,1), (2,3), (4,2)\}.
\] 
Encontre $<1>_R$, $<2>_R$, $<3>_R$ e $<4>_R$. {\bf{\it Resposta:} ${\bf <2>_R\espaco=\espaco \{1,4\}}$.}
\item Mostre que $R$ \'e reflexiva se e somente se $\forall x\in A, x\in <x>_R\espaco$. 
\item Mostre que $R$ \'e sim\'etrica se e somente se $\forall x,y\in A, x\in <y>_R\espaco\to\espaco y\in <x>_R$.
\item Mostre que $\forall x\in A, <x>_R\espaco\neq\espaco\varnothing$ se e somente se $\Ima(R)=A$.
\item Suponha que $\Dom(R)=A$ e que $R$ seja sim\'etrica e transitiva. Mostre que
\[
\forall x,y\in A, <x>_R\espaco\subseteq\espaco<y>_R\espaco\to\espaco xRy.
\]
Mostre tamb\'em que,
\[
<x>_R\espaco\subseteq\espaco<y>_R\espaco\to\espaco <x>_R\espaco=\espaco<y>_R.
\]
\item Suponha que $R$ seja sim\'etrica e transitiva. Mostre que:
\[
\forall x,y\in A, <x>_R\espaco\inter <y>_R\espaco\neq\espaco\varnothing\to\espaco <x>_R\espaco=<x>_R.
\]
\end{enumerate}

%excercicio21
\item {\bf Acredite se quiser:}  {\bf{\it Resposta:} Dica, A conjectura \'e falsa, assim como o contra-exemplo.}

\noindent \textit{\textbf{Conjectura:}} Suponha que $A$ e $B$ sejam conjuntos tais que $A\times B=B\times A$. Ent\ao $A=B$.

\noindent \textit{\textbf{``Demonstra\caoi'':}} Suponha $A\times B=B\times A$. Seja $a\in A$, com $b\in B$ tais que $(a,b)\in A\times B$. Como $A\times B=B\times A$, $(a,b)\in B\times A$. Assim $a\in B$ e portanto $A\subseteq B$. Um argumentos similar mostra que $B\subseteq A$.

\noindent \textit{\textbf{``contra-exemplo'':}} Sejam $A=\{1,2,3\}$ e $B=\varnothing$. En\tao $A\times B=B\times A=\varnothing$, mas $A\neq B$. 

%excercicio22
\item {\bf Acredite se quiser:}  

\noindent \textit{\textbf{Conjectura:}} Suponha que $R$ seja uma rela\cao em um conjunto n\ao vazio $A$. Se $R$ \'e n\ao sim\'etrica ent\ao $R$ \'e assim\'etrica. 

\noindent \textit{\textbf{``Demonstra\caoi'':}} Seja $R$ uma rela\cao em um conjunto n\ao vazio $A$. Suponha $a,b\in A$ com $(a,b)\in R$. Como $R$ n\ao \'e sim\'etrica, $(b,a)\notin R$ logo $R$ \'e assim\'etrica.

\noindent \textit{\textbf{``contra-exemplo'':}} Sejam $A=\{1,2,3\}$ e $R=\{(1,2),(2,1),(1,3)\}$. Ent\ao $R$ n\ao \'e sim\'etrica nem assim\'etrica.

%excercicio23
\item {\bf Acredite se quiser:}  

\noindent \textit{\textbf{Conjectura:}} Suponha que $R$ seja uma rela\cao em um conjunto n\ao vazio $A$. Se $R$ \'e sim\'etrica e transitiva ent\ao $R$ \'e reflexiva. 

\noindent \textit{\textbf{``Demonstra\caoi'':}} Suponha que $R$ seja uma rela\cao sim\'etrica e transitiva em um conjunto n\ao vazio $A$. Sejam $a,b\in A$ com $(a,b)\in R$. Como $R$ \'e sim\'etrica, $(b,a)\in R$. Mas $R$ tamb\'em \'e transitiva, logo temos que $(a,a)\in R$ e consequentemente $R$ \'e reflexiva.

\noindent \textit{\textbf{``contra-exemplo'':}} Sejam $a=\{a,b,c\}$ e $R=\{(a,b), (b,a), (a,c), (b,c), (a,a)\}$. Ent\ao $R$ \'e sim\'etrica e transitiva, mas n\ao \'e reflexiva pois $(b,b)\notin R$.
\end{enumerate}
%%%%%%%%%%%%%%%%%%%%%%%%%%%%%%%%%%%%%%%%%%%%%%%%%%%%%%%%%%%%%%%%%%%%%%%%
%%%%%%%%%%%%%%%%%%%%%%%%% secao 2.4 %%%%%%%%%%%%%%%%%%%%%%%%%%%%%%%%%%%%
%%%%%%%%%%%%%%%%%%%%%%%%%%%%%%%%%%%%%%%%%%%%%%%%%%%%%%%%%%%%%%%%%%%%%%%%
\paragraph{Exerc\ih cios \ref{mrelacoes}}

\begin{enumerate}[{\bf 1.}]

%excercicio1
\item Sejam $A=\{1,2,4\}$ e $B=\{1,3,4\}$. Sejam $R=\{(1,3),(1,4),(4,4)\}$ uma rela\cao de $A$ em $B$, $S=\{(1,1),(3,4),(3,2)\}$ uma rela\cao de $B$ em $A$ e $T=\varnothing$ uma rela\cao de $A$ em $B$. Encontre:
\begin{enumerate}[a)]
\item $\Dom(R)$. {\bf{\it Resposta:} ${\bf \Dom(R)=\{1,4\}}$.}
\item $\Dom(S)$. {\bf{\it Resposta:} ${\bf \Dom(S)=\{1,3\}}$.}
\item $\Dom(T)$. {\bf{\it Resposta:} ${\bf \Dom(T)=\varnothing}$.}
\item $\Ima(R)$. {\bf{\it Resposta:} ${\bf \Ima(R)=\{3,4\}}$.}
\item $\Ima(S)$. {\bf{\it Resposta:} ${\bf \Ima(S)=\{1,2,4\}}$.}
\item $\Ima(T)$. {\bf{\it Resposta:} ${\bf \Ima(S)=\varnothing}$.}
\item $S\circ R$.
\item $R\circ S$.
\item $\Dom(S\circ R)$.
\item $\Ima(S\circ R)$.
\item $\Dom(R\circ S)$.
\item $\Ima(R\circ S)$.
\item $R^{-1}$. {\bf{\it Resposta:} ${\bf R^{-1}=\{(3,1),(4,1),(4,4)\}}$.}
\item $S^{-1}$. {\bf{\it Resposta:} ${\bf S^{-1}=\{(1,1),(4,3),(2,3)\}}$.}
\item $I_A$. {\bf{\it Resposta:} ${\bf I_A=\{(1,1),(2,2),(4,4)\}}$.}
\item $I_B$. {\bf{\it Resposta:} ${\bf I_B=\{(1,1),(3,3),(4,4)\}}$.}
\item $R^{-1}\circ S^{-1}$.
\item $S^{-1}\circ R^{-1}$.
\item $(R\circ S)^{-1}$.
\item $(S\circ R)^{-1}$.
\item $T^{-1}$.
\item $I_B^{-1}$.
\item $(R\bola S)\bola R$.
\item $R\bola(S\bola R)$. 
\end{enumerate}

%excercicio2
\item Seja $R$ uma rela\cao em um conjunto n\ao vazio $A$. Mostre que:
\begin{enumerate}[a)]
\item $(R^{-1})^{-1}=R$
\item $I_A^{-1}-I_A$
\item $R$ \'e reflexiva se e somente se $I_A\subseteq R\subseteq R\bola R$.
\item $R$ \'e sim\'etrica se e somente se $R=R^{-1}$. 

{\bf{\it Resposta:} Suponha que ${\bf R}$ seja sim\'etrica. Seja ${\bf (x,y)\in R}$. Ent\ao ${\bf (y,x)\in R}$, logo ${\bf (x,y)\in R^{-1}}$ e ${\bf R\subseteq R^{-1}}$. Agora, suponha ${\bf (x,y)\in R^{-1}}$. Ent\ao ${\bf (y,x)\in R}$ logo ${\bf (x,y)\in R}$ e ${\bf R^{-1}\subset R}$ e assim temos que ${\bf R=R^{-1}}$. Para a outra implica\caoi, suponha que ${\bf R=R^{-1}}$. Seja ${\bf (x,y)\in R}$. Ent\ao ${\bf (x,y)\in R^{-1}}$ (pois ${\bf R=R^{-1}}$) portanto ${\bf (y,x)\in R}$ e ${\bf R}$ \'e sim\'etrica.} 

\item $R$ \'e transitiva se e somente se $R^{-1}$ \'e transitiva.
\item $R$ \'e uma rela\cao de equival\^encia e se somente se $R^{-1}$ \'e uma rela\cao de equival\^encia.
\item Suponha que $\Dom(R)=A$. $R$ \'e uma rela\cao de equival\^encia se e somente se $R=R^{-1}=R\bola R$.
\item $R$ \'a assim\'etrica se e somente se $R\inter R^{-1}=\varnothing$.
\item $R\uni R^{-1}=A\times A$ implica que $R$ \'e completa.
\item $R$ sim\'etrica implica $R\bola R$ \'e sim\'etrica.
\item $I_{\Dom(R)}\subseteq R^{-1}\bola R$.
\item $R$ \'e uma ordem parcial se e somente se $R^{-1}$ \'e uma ordem parcial.
\item $R$ \'e uma ordem parcial se e somente se $R\inter R^{-1}=I_A$ e $R\bola R=R$.
\item $R$ \'e uma ordem parcial estrita se e somente se $R^{-1}$ \'e uma ordem parcial estrita.
\end{enumerate}

%excercicio3
\item Seja $A$ um conjunto n\ao vazio com $R,S$ rela\coes em $A$. Considere as seguintes conjecturas. Prove as verdadeiras e d\^e exemplos para aquelas que s\ao falsas.
\begin{enumerate}[a)]
\item $R$ \'e sim\'etrica implica $R\bola R$ sim\'etrica.
\item $R\bola S^{-1}=S\bola R^{-1}$ implica que $R\bola S^{-1}$ seja sim\'etrica.
\end{enumerate}

%excercicio4
\item Seja $R$ uma rela\cao de $A$ em $B$ e $S$ uma rela\cao  de $B$ em $C$. Mostre que:
\begin{enumerate}[a)]
\item $\Dom(S\bola R)\subseteq \Dom(R)$.
\item $\Ima(S\bola R)\subseteq \Ima(S)$.
{\bf{\it Resposta:} Seja ${\bf y\in \Ima(S\bola R)}$. Ent\ao existe ${\bf x}$ tal que ${\bf (x,y)\in S\bola R}$. Mas isto significa que que existe ${\bf z}$ tal que ${\bf (x,z)\in R}$ e ${\bf (z,y)\in S}$, consequentemente, ${\bf y\in \Ima(S)}$.}

\item $\Ima(R)\subseteq \Dom(S)$ implica $\Dom(S\bola R)=\Dom(R)$. A rec\ih proca \'e verdadeira? 
\end{enumerate}

%excercicio5. 
\item Complete a demonstra\cao do teorema \ref{relto1}.

%excercicio6
\item Suponha que $R$ e $S$ sejam rela\coes de equival\^encia em um conjunto n\ao vazio $A$. Considere as seguintes conjecturas. Prove as verdadeiras e d\^e exemplos para aquelas que s\ao falsas. 
\begin{enumerate}[a)]
\item $R\uni S$ \'e uma rela\cao de equival\^encia implica que $R\bola S=S\bola R$.
\item $R\uni S= R\bola S$ implica que $R\uni S$ \'e uma rela\cao de equival\^encia.
\item $R\uni S= R\bola S$ implica que $R\bola S=S\bola R$.
\end{enumerate}

%excercicio7
\item Seja $R$ a rela\cao $<$ nos inteiros. Mostre que $R$ \'e uma ordem parcial estrita. Tamb\'em mostre que $R\uni I_{\mathbb{Z}}$ (que \'e $\leq$) \'e uma ordem parcial.

%excercicio8
\item Seja $R$ uma ordem parcial em um conjunto n\ao vazio $A$. Mostre que $R-I_A$ \'e uma ordem parcial estrita em $A$.

%excercicio9
\item Seja $R$ uma rela\cao em um conjunto n\ao vazio $A$. Prove ou d\^e contra-exemplos (refira-se ao excerc\ih cio \ref{relexer17}, da se\cao \ref{relacoes}):
\begin{enumerate}[a)]
\item $R_{ref}=R\uni I_A$. {\bf{\it Resposta:} Est\'a correto, falta a demonstra\caoi.}
\item $R_{sim}=R\uni R^{-1}$.
\item $R_{trans}=R\uni (R\bola R)$.
\end{enumerate}

%excercicio10
\item Sejam $R,S$ e $T$ rela\coes entre conjuntos. Determine algumas condi\coes sobre $R,S$ e $T$ para garantir as seguintes conclus\ois. Demonstre que suas conjecturas est\ao corretas.
\begin{enumerate}[a)]
\item $R\bola S=R\bola T$ implica $S=T$.
\item $S\bola R=T\bola R$ implica $S=T$.
\end{enumerate}

%excercicio11
\item {\bf Acredite se quiser:}  

\noindent \textit{\textbf{Conjectura:}} Seja $R$ uma rela\cao em um conjunto n\ao vazio $A$. Se $R$ \'e transitiva ent\ao $R\bola R$ \'e transitiva.

\noindent \textit{\textbf{``Demonstra\caoi'':}} Seja $R$ uma rela\cao transitiva em $A$. Sejam $a,b,c\in A$ com $(a,b),(b,c)\in R\bola R$. Ent\ao existem $d,e\in A$ tais que $(a,d),(d,b),(b,e),(e,c)\in R$. Como $R$ \'e transitiva $(a,b),(b,c)\in R$, isso implica que $(a,c)\in R\bola R$, logo $R\bola R$ \'e transitiva.

\noindent \textit{\textbf{``contra-exemplo'':}} Seja, $A=\{1,2,3\}$ $R=\{(1,2),(2,2),(2,3),(1,3)\}$. Assim,
\[
R\bola R=\{(1,3),(1,2),(2,3),(2,2)\},
\]
logo temos $R$ transitiva e $R\bola R$ n\aoi.

%excercicio12
\item {\bf Acredite se quiser:} {\bf{\it Resposta:} Dica, A demonstra\cao est\'a incorreta.} 

\noindent \textit{\textbf{Conjectura:}} Sejam $R,S$ rela\coes de equival\^encia em um conjunto n\ao vazio $A$. Se $R\bola S=S\bola R$ ent\ao $R\uni S$ \'e uma rela\cao de equival\^encia.

\noindent \textit{\textbf{``Demonstra\caoi'':}} Sejam $R,S$ como acima. Claramente, $R\uni S$ \'e reflefiva. Se $(a,b)\in R\uni S$ ent\ao $(a,b)\in R$ ou $(a,b)\in S$. Se $(a,b)\in R$, e como $R$ \'e sim\'etrica, $(b,a)\in R$, logo $(b,a)\in R\uni S$. Por argumentos semelhantes, se $(a,b)\in S$ ent\ao $(b,a)\in S$, isto demonstra que $R\uni S$ \'e sim\'trica. Aogra, sejam $(a,b),(b,c)\in R\uni S$. Se ambos pertencem a $R$, ou se ambos pertencem a $S$, a transitividade de cada um implica que $R\uni S$ \'e transitiva. Portanto, suponha que $(a,b)\in R$ e $(b,c)\in S$. Como $R\bola S=S\bola R$ e $(a,c)\in S\bola R$, $(a,c)\in R\bola S$. Assim existe $d\in A$ tal que $(a,d)\in S$ e $(d,c)\in R$. Mas, ambos $R$ e $S$ s\ao sim\'etricos, portanto $(c,d)\in R$ e $(d,a)\in S$. Logo, $(c,a)\in R$. Mas $R$ \'e sim\'etrica, assim temos $(a,c)\in R$ e consequentemente $R\uni S$ \'e transitiva. Argumentos similares podem ser utilizados no caso $(a,b)\in S$ e $(b,c)\in R$.

\noindent \textit{\textbf{``contra-exemplo'':}} Sejam 
\begin{equation*}
 \begin{aligned}
A=&\{a,b,c,d\},\\
R=&I_A\uni\{(a,b),(b,a),(a,c),(c,a)\},\\
S=&I_A\uni\{(c,d),(d,c),(a,c),(c,a),(d,a),(a,d)\}.
 \end{aligned}
\end{equation*}
Ent\ao $R,S$ s\ao rela\coes de equival\^encia com $R\bola S=S\bola R$, mas $R\uni S$ cont\'em $(b,a)$ e $(a,d)$ mas n\ao $(b,d)$ e assim n\ao \'e transitiva.

%excercicio13
\item {\bf Acredite se quiser:}  

\noindent \textit{\textbf{Conjectura:}} Seja $R$ uma ordem total estrita em um conjunto n\ao vazio $A$. Ent\ao $S=(A\times A)-R$ \'e uma ordem total em $A$. 

\noindent \textit{\textbf{``Demonstra\caoi'':}} Como $R$ \'e irreflexiva, $I_A\inter R=\varnothing$ portanto $I_A\subseteq S$ e $S$ \'e reflexiva. Agora suponha que $(a,b),(b,c)\in S$. $R$ \'e completa, assim como $(a,b),(b,c)\in S$, temos que $(b,a),(c,b)\in R$. A trnasitividade de $R$ implica que $(c,a)\in R$. Se $(a,c)\in R$ ent\ao $(a,a)\in R$, que \'e imposs\ih vel, consequentemente $(a,c)\in S$ e $S$ \'e transitiva. Agora, suponha que $(a,b),(b,a)\in S$. Ent\ao devemos ter $a=b$, caso contr\'ario $R$ n\ao seria completa. Assim $S$ \'e uma ordem parcial. Se $a,b\in A$, $a\neq b$ e $(a,b)\notin S$, ent\ao $(a,b)\in R$, e como notado anteriormente, isto implica $(b,a)\in S$ e consequentemente $S$ \'e completa e assim uma ordem total.

\noindent \textit{\textbf{``contra-exemplo'':}} Sejam $A=\{1,2,3\}$ e $R=\{(1,2),(2,3),(3,1)\}$. Ent\ao
\[
S=\{(1,1),(2,2),(3,3),(2,1),(1,3),(3,2)\}
\]
n\ao \'e uma ordem total pois n\ao \'e transitiva ($(1,3),(3,2)\in S$, mas $(1,2)\notin S$).
\end{enumerate}
%%%%%%%%%%%%%%%%%%%%%%%%%%%%%%%%%%%%%%%%%%%%%%%%%%%%%%%%%%%%%%%%%%%%%%%%
%%%%%%%%%%%%%%%%%%%%%%%%% secao 2.5 %%%%%%%%%%%%%%%%%%%%%%%%%%%%%%%%%%%%
%%%%%%%%%%%%%%%%%%%%%%%%%%%%%%%%%%%%%%%%%%%%%%%%%%%%%%%%%%%%%%%%%%%%%%%%
\paragraph{Exerc\ih cios \ref{equivalencia}}

\begin{enumerate}[{\bf 1.}]

%excercicio1
\item Sejam $A=\{1,2,3,4,5,6\}$ e $\Pi=\{\{2,4,6\},\{1,5\},\{3\}\}$. Liste os elementos de $A/\Pi$. Encontre $[2]_{A/\Pi}$. 
{\bf{\it Resposta:} ${\bf [2]_{A/\Pi}=\{2,4,6\}}$.}

%excercicio2
\item Seja $\Pi$ uma parti\cao de $A$ e sejam $B,C\in\Pi$. Mostre que se $B\inter C=\varnothing$ ent\ao $B=C$

%excercicio3
\item\label{eqrexcerc3} Sejam $\Pi_1$ e $\Pi_2$ parti\coes de $A$. Dizemos que $\Pi_1$ \'e {\it mais fina}\index{Parti\cao Mais Fina} que $\Pi_2$ e escrevemos $\Pi_1\preceq\Pi_2$ se e somente se $\forall B\in \Pi_1, \exists C\in \Pi_2 \mid  B\subseteq C$.
\begin{enumerate}[a)]
\item Se $A=\{1,2,3,4\}$, d\^e exemplos de parti\coes $\Pi_1$ e $\Pi_2$ tais que:
\begin{enumerate}[i.]
\item $\Pi_1\preceq\Pi_2$ {\bf{\it Resposta:} ${\bf \Pi_1=\{\{1\},\{2,3\},\{4\}\}}$ e ${\bf \Pi_2=\{\{1,2,3\},\{4\}\}}$.}
\item $\Pi_1$ n\ao \'e mais fino que $\Pi_2$ e $\Pi_2$ n\ao \'e mais fino que $\Pi_1$ {\bf{\it Resposta:} ${\bf \Pi_1=\{\{1,2\},\{3,4\}\}}$ e ${\bf \Pi_2=\{\{1,3\},\{2,4\}\}}$.}
\end{enumerate}
\item Sejam $\Pi_1$ e $\Pi_2$ como no exemplo \ref{eqexe1} desta se\cao e seja $\Pi$ qualquer outra parti\cao de $A$. Mostre que $\Pi_1\preceq\Pi\preceq\Pi_2$.
\end{enumerate}

%excercicio4
\item Seja $R$ uma rela\cao de equival\^encia em $A$. Mostre que $A/[A]_R=R$. {\bf{\it Resposta:} (Resposta parcial) Seja ${\bf (x,y)\in A/[A]_R }$. En\tao ${\bf x}$ e ${\bf y}$ s\ao ambos elementos da mesma classe de equival\^encia de $R$. Mas isso significa que ${\bf xRy}$ ou ${\bf (x,y)\in R}$. Consequentemente, ${\bf A/[A]_R\subseteq R}$.}

%excercicio5
\item Seja $\Pi$ uma parti\cao de $A$. Mostre que $[A]_{A/\Pi}=\Pi$.

%excercicio6
\item Sejam $R_1,R_2$ rela\coes de equival\^encia em A. Dizemos que $R_1$ \'e {\it mais fina}\index{Rela\caoi!Mais fina} que que $R_2$ e escrevemos $R_1\preceq R_2$ se e somente se $R_1\subseteq R_2$.
\begin{enumerate}[a)]
\item Seja $A=\{1,2,3,4\}$. D\^e exemplos de rela\coes de equival\^encia $R_1,R_2$ tais que:
\begin{enumerate}[i)]
\item $R_1\preceq R_2$. 
{\bf{\it Resposta:} ${\bf R_1=\{(1,1),(2,2),(3,3),(4,4)\}}$ ${\bf R_2=\{(1,1),(2,2),(3,3),(4,4),(1,2),(2,1)\}}$.}
\item $R_1$ n\ao \'e mais fino que $R_2$ e $R_2$ n\ao \'e mais fino que $R_1$. 
\end{enumerate}
\item Seja $A$ um conjunto n\ao vazio e seja $\Omega=\{R: R \textrm{ uma rela\cao de equival\^encia em $A$}\}$. Mostre que $\preceq$ \'e uma ordem parcial em $\Omega$. O que pode ser dito sobre $\preceq$ ser ou n\ao completa?
\item Se $R_1$ e $R_2$ s\ao rela\coes de equival\^encia em uma conjunto n\ao vazio $A$ com $R_1\preceq R_2$, a parti\cao induzida por $R_1$ \'e mais fina que a parti\cao induzida por $R_2$? A rec\ih proca? Ou nada?
\end{enumerate}

%excercicio7
\item\label{eqexcer7} Seja $\Psi$ e $\Pi$ parti\coes de um conjunto n\ao vazio $A$. Definimos
\[
\Psi\star\Pi=\{C\inter D: C\in\Psi, D\in\Pi,C\inter D\neq\varnothing\}.
\]
\begin{enumerate}[a)]
\item Seja $A=\{1,2,3,4,5\}$, $\Psi=\{\{1,2,3\},\{4,5\}\}$ e $\Pi=\{\{1,2\},\{3,4\},\{5\}\}$. Encontre $\Psi\star\Pi$.
{\bf{\it Resposta:} ${\bf \Psi\star\Pi=\{\{1,2\},\{3\},\{4\},\{5\}\}}$.}
\item Mostre que se $\Psi$ e $\Pi$ s\ao parti\coes de um conjunto n\ao vazio $A$, ent\ao $\Psi\star\Pi$ \'e uma parti\cao de $A$.
\item Mostre que $\Psi\star\Pi$ \'e mais fina que $\Psi$ e $\Pi$.
\end{enumerate}

%excercicio8
\item Vamos generalizar a rela\cao de equival\^encia dada no exemplo \ref{relex11} sa se\cao \ref{relacoes} e discutido no come\cc o desta se\caoi. Seja $m\in\mathbb{N}$. Se $x,y\in\mathbb{Z}$, dizemos que $x\equiv y(mod \espaco m)$ se e somente se $m|(x-y)$. [Note que: $x\equiv y(mod \espaco m)$ \'e lido como ``$x$ \'e congruente a $y$ m\'odulo $m$.''] Assim, a rela\cao de equival\^encia mencionada anteiriormente era a congru\^encia m\'odulo $5$. Mais uma nota\caoi, escreveremos as classes de equival\^encia de congru\^encia m\'odulo $m$ como $[x]_m$ e denotaremos o conjunto de todas as classe de equival\^encia m\'odulo $m$ por $\mathbb{Z}_m$. Assim, $\mathbb{Z}_5=\{[1]_5,[2]_5,[3]_5,[4]_5,[5]_5\}$.
\begin{enumerate}[a)]
\item Encontre $[3]_3,[2]_3,[5]_1,$. {\bf{\it Resposta:} ${\bf [2]_3=\{2,5,8,11,\ldots\}}$.}
\item Encontre duas solu\coes para cada uma das seguintes:
\begin{enumerate}[i)]
\item $x\equiv 3(mod \espaco 14)$.
\item $x^2\equiv 2(mod \espaco 7)$. {\bf{\it Resposta:} Uma solu\cao \'e ${\bf x=3}$.}
\item $x^2\equiv 3(mod \espaco 7)$.
\end{enumerate}
\item Sejam $m,n\in\mathbb{N}$. Mostre que se $m|n$ ent\ao $\mathbb{Z}_n$ \'e mais fina que $\mathbb{Z}_m$.
\item Seja $m\in\mathbb{N}$. Mostre que $\forall x,y,z\in\mathbb{Z}$, $x\equiv y(mod \espaco m)$ implica $x+z\equiv y+z(mod \espaco m)$ e $xz\equiv yz(mod \espaco m)$.
\end{enumerate}

%excercicio9
\item Seja $R$ e $S$ rela\coes de equival\^encia de um conjunto n\ao vazio $A$. Sabemos que $R\inter S$ \'e tamb\'em uma rela\cao de equival\^encia em $A$.
\begin{enumerate}[a)]
\item Seja $x\in A$. Mostre que $[x]_{R\inter S}=[x]_R\inter[y]_S$. 
\item Mostre que $[A]_{R\inter S}=[A]_R\star[A]_S$, onde $\star$ \'e a opera\cao definida no excerc\ih cio \ref{eqexcer7}.
\end{enumerate}

%excercicio10
\item Se $p,q\in\mathbb{N}$, sabemos do excerc\ih cio \ref{eqexcer7} que $\mathbb{Z}_p\star\mathbb{Z}_q$ \'e uma parti\cao de $\mathbb{Z}$.  Existe $n\in\mathbb{N}$ tal que $\mathbb{Z}_p\star\mathbb{Z}_q=\mathbb{Z}_n$? Se for verdade, demonstre o resultado. Se for falso, d\^e um contra-exemplo para mostrar que esta parti\cao n\ao \'e desta forma.
{\bf{\it Resposta:} Dica, compute alguns elementos de ${\bf \mathbb{Z}_2\star\mathbb{Z}_3}$.}

%excercicio11
\item {\bf Acredite se quiser:}  

\noindent \textit{\textbf{Conjectura:}} Seja $A$ um conjunto n\ao vazio e $\Pi,\Psi$ parti\coes de $A$. Se $\Pi\preceq\Psi$ e $\Psi\preceq\Pi$ ent\ao $\Pi=\Psi$.

\noindent \textit{\textbf{``Demonstra\caoi'':}} Sejam $\Pi,\Psi$ como acima e seja $B\in\Pi$. Como $\Pi\preceq\Psi$, existe $C\in\Psi $ tal que $B\subseteq C$. Mas como $\Psi\preceq\Pi$, $C\subseteq B$, e portanto $B=C$, logo $B\in\Psi$. Um argumento parecido mostra que $\Psi\subseteq\Pi$ e consequentemente temos $\Pi=\Psi$. 

\noindent \textit{\textbf{``contra-exemplo'':}} Seja $A=\{1,2,3\}$.
\begin{equation*}
 \begin{aligned}
A=&\{1,2,3,4,5\},\\
\Pi=&\{\{1,2\},\{3\},\{4,5\}\},\\
\Psi=&\{\{1\},\{2,3,4\},\{5\}\},\\
 \end{aligned}
\end{equation*}
Ent\ao $\Pi\preceq\Psi$ $(\{3\}\subseteq\{2,3,4\})$ e $\Psi\preceq\Pi$ $(\{1\}\subseteq\{1,2\})$, mas claramente $\Psi\neq\Pi$.
\end{enumerate}
%%%%%%%%%%%%%%%%%%%%%%%%%%%%%%%%%%%%%%%%%%%%%%%%%%%%%%%%%%%%%%%%%%%%%%%%
%%%%%%%%%%%%%%%%%%%%%%%%% secao 2.6 %%%%%%%%%%%%%%%%%%%%%%%%%%%%%%%%%%%%
%%%%%%%%%%%%%%%%%%%%%%%%%%%%%%%%%%%%%%%%%%%%%%%%%%%%%%%%%%%%%%%%%%%%%%%%
\paragraph{Exerc\ih cios \ref{funcoes}}

\begin{enumerate}[{\bf 1.}]

%excercicio1
\item Seja $A=\{1,2,3,4,5,6\}$ e seja $f:A\to A$ dada por 
\begin{equation*}
 f(x)=\left\{ \begin{array}{lc}
x+1, & \textrm{ se $x\neq 6$;} \\
1,    & \textrm{ se $x=    6$.} \\
\end{array}\right.   
\end{equation*}
\begin{enumerate}[a)]
\item Encontre $f(3)$, $f(6)$, $f\bola f(3)$, $f(f(2))$. {\bf{\it Resposta:} ${\bf f(3)=4}$.}
\item Encontre a pr\'e-imagem de $2$ e $1$.
\item Mostre que $f$ \'e bijetora. {\bf{\it Resposta:} (Resposta Parcial) Se ${\bf x=1}$ ent\ao ${\bf f(6)=x}$, se ${\bf x\neq 1}$ ent\ao ${\bf f(x-1)=x}$ portanto ${\bf f}$ \'e sobrejetora.}
\end{enumerate}

%excercicio2
\item Mostre que $f:\mathbb{R}\to\mathbb{R}$ dad por $f(x)=x^3$ \'e injetora e sobrejetora, enquanto que $g:\mathbb{R}\to\mathbb{R}$ dada por $g(x)=x^2-1$ n\ao \'e injetora nem sobrejetora. 

%excercicio3
\item\label{funcexerc3} Suponha $f:A\to B$ e $g:B\to C$. Mostre que $g\bola f:A\to C$.

%excercicio4
\item \begin{enumerate}[a)]
\item Sejam $A,B$ e $f:A\to B$ dados por:
\begin{equation*}
 \begin{aligned}
A=&\{1,2,3,4\},\\
B=&\{1,2,3,\},\\
f=&\{(1,3),(2,1),(3,1),(4,2)\}.
 \end{aligned}
\end{equation*}
Encontre $f^{-1}\bola f$.
{\bf{\it Resposta:} ${\bf f^{-1}\bola f=\{(1,1),(2,2),(3,3),(4,4),(2,3),(3,2)\}}$.}

\item Sejam $A,B$ conjuntos n\ao vazios e $f:A\to B$. Mostre que $f^{-1}\bola f$ \'e uma rela\cao de equival\^encia em $A$. (Note que $f^{-1}$ pode ou n\ao ser uma fun\caoi). Tamb\'em mostre que $[x]_{f^{-1}\bola f}=\{y: f(x)=f(y)\}$.
\end{enumerate}

%excercicio5
\item Seja $f:A\to B$. Prove que
\begin{enumerate}[a)]
\item $f$ \'e injetora se e somente se existe $g:B\to A$ tal que $g\bola f=I_A$.
{\bf{\it Resposta:} Dica - Defina ${\bf g}$ como ${\bf g(y)=x}$ quando ${\bf y\in \Ima(f)}$ e onde ${\bf f(x)=y}$. Para ${\bf y\notin \Ima(f)}$ defina ${\bf g(y)}$ arbitrariamente.}

\item $f$ \'e sobrejetora se e somente se existe $g:B\to A$ tal que $f\bola g=I_B$.
\item $f$ \'e sobrejetora se e somente se $f\bola f^{-1}=I_B$.
\end{enumerate}

%excercicio6
\item Sejam $f:A\to B$ e $g:B\to A$. Mostre que se $g\bola f=I_A$ e $f\bola g=I_B$ ent\ao $f$ \'e uma bije\cao e $g=f^{-1}$

%excercicio7
\item Seja $R$ uma rela\cao de equival\^encia em um conjunto n\ao vazio $A$. Definimos a rela\cao $\alpha$ de $A$ em $[A]_R$ por
\[
\alpha=\{(x,[x]_R): x\in A\}.
\]
\begin{enumerate}[a)]
\item Mostre que $\alpha:A\to [A]_R$.
\item Mostre que $\alpha$ \'e sobrejetora.
\item Sob quais condi\coes $\alpha$ ser\'a injetora?
\end{enumerate}

%excercicio8
\item Seja $f:A\to A$. Suponha que $f$ tamb\'em seja uma rela\cao de equival\^encia. O que podemos dizer sobre $f$? 
{\bf{\it Resposta:} Dica - Pense sobre que fun\coes s\ao reflexivas.}

%excercicio9
\item Sejam $f:A\to B$ e $g:A\to B$. Demonstre ou d\^e contra-exemplos para as seguintes conjecturas:
\begin{enumerate}[a)]
\item $f\uni g:A\to B$.
\item $f\inter g:A\to B$. {\bf{\it Resposta:} Dica - N\ao \'e verdade.}
\item $f\uni g:A\to B$ implica $f=g$.
\item $f\inter g:A\to B$ implica $f=g$.
\end{enumerate}

%excercicio10
\item Seja $f:A\to B$ e $g:C\to D$, com $A\inter C=\varnothing$. [Para refrescar sua mem\'oria sobre restri\cois, veja o excerc\ih cio \ref{relexer13} da se\cao \ref{relacoes}]. 
\begin{enumerate}[a)]
\item Mostre que $f\uni g:A\uni C\to B\uni D$.
\item Mostre que $f\uni g|_A=f$ e $f\uni g|_C=g$. 
\end{enumerate}

%excercicio11
\item Seja $f:\mathbb{R}\to\mathbb{R}$ definida por $f(x)=\textrm{sen}(x).$
\begin{enumerate}[a)]
\item Mostre que $f$ n\ao \'e injetora.
\item Mostre que $f|_{[\pi/2,\pi/2]}$ \'e injetora.
\end{enumerate}

%excercicio12
\item Seja $A$ um conjunto n\ao vazio e seja 
\[
\Psi=\{\phi: \phi \textrm{ \'e uma parti\cao de } A\}.
\] 
Lembre-se que $\preceq$ (mais fino que) \'e uma rela\cao de ordem parcial em $\Psi$. Seja
\[
\Re=\{R: R \textrm{ \'e uma rela\cao de equival\^encia em } A\}.
\]
Sabemos que existe uma bije\cao entre os elementos de $\Psi$ e $\Re$, assim denotemos a rela\cao de equival\^encia associada com a parti\cao $\theta$ por $R_{\theta}$. Definimos s rela\cao $\sqsubseteq$ em $\Re$ por
\[
R_{\phi}\sqsubseteq R_{\theta} \textrm{ se e somente se } \phi \preceq \theta.
\]
\begin{enumerate}[a)]
\item Mostre que $\sqsubseteq$ \'e uma ordem parcial em $\Re$.
\item Mostre (ou d\^e um contra-exemplo):
\[
R_{\phi}\sqsubseteq R_{\theta} \textrm{ se e somente se } R_{\phi}\subseteq R_{\theta}.
\]
\end{enumerate}

%excercicio13
\item Suponha que $f:A\to B$ e $g:B\to C$, onde $A,B$ e $C$ s\ao conjuntos n\ao vazios. Demonstre ou d\^e contra-exemplos para as seguintes conjecturas:
\begin{enumerate}[a)]
\item $g\bola f$ bije\cao implica que $f$ \'e injetora. 
\item $g\bola f$ bije\cao implica que $f$ \'e sobrejetora.
\item $g\bola f$ bije\cao implica que $g$ \'e sobrejetora. {\bf{\it Resposta:} Dica - \'E verdade.}
\item $g\bola f$ bije\cao implica que $g$ \'e injetora.
\end{enumerate}

%excercicio14
\item Seja $f:A\to B$, com $R$ uma ordem total estrita em $A$ e $S$ uma ordem total estrita em $B$. Dizemos que $f$ \'e {\it monot\^onica}\index{Fun\caoi!Monot\^onica} se e somente se $\forall x,y\in A,$ $xRy$ implica $f(x)Sf(y)$.
\begin{enumerate}[a)]
\item Com a ordem usual ($<$) em $\mathbb{R}$, de um exemplo de uma fun\cao que seja monot\^onica.
\item Com a ordem usual ($<$) em $\mathbb{R}$, de um exemplo de uma fun\cao que n\ao seja monot\^onica.
\item Se $f:A\to B$ \'e monot\^onica, mostre que $f$ \'e injetora.
\end{enumerate}

%excercicio15
\item {\bf Acredite se quiser:}  {\bf{\it Resposta:} Dica - A demonstra\cao n\ao est\'a correta.}

\noindent \textit{\textbf{Conjectura:}} Seja $f:A\to B$ e seja $R$ uma ordem total estrita em $A$. Definimos a rela\cao $S$ em $B$ por
\[
xSy \leftrightarrow \exists a,b\in A \mid  aRb\ee f(a)=x,f(b)=y.
\]
Ent\ao $S$ \'e uma ordem parcial estrita.

\noindent \textit{\textbf{``Demonstra\caoi'':}} Suponha que $f$ e $ R$ s\ao como acima e $S$ \'e definida como indicado. Seja $x\in B$ com $xSx$. Mas isto significa que existe $a\in A$ tal que $f(a)=x$. Assim $aRa$, que \'e imposs\ih vel, pois $R$ \'e irreflexiva, portanto $S$ \'e irreflexiva. Agora, suponha $x,y,z\in B$ com $xSy$, $ySz$. Ent\ao existem $a,b,c\in A$ tais que $f(a)=x$, $f(b)=y$ e $f(c)=z$ e $aRb$ e $bRc$. Como $R$ \'e transitiva, $aRc$ e portanto $xSz$, logo $S$ \'e transitiva.

\noindent \textit{\textbf{``contra-exemplo'':}} Sejam $A=\{1,2,3\}$, $B=\{1,2,3,4\}$ e $f:A\to B$ dada por $f(1)=1$, $f(2)=1$ e $f(3)=4$. Suponha,
\[
R=\{(1,2),(2,3),(1,3)\}
\]
Ent\ao $S=\{(1,2),(1,4)\}$, que \'e transitiva mas n\ao irreflexiva.

%excercicio16
\item {\bf Acredite se quiser:}  

\noindent \textit{\textbf{Conjectura:}} Seja $f:A\to B$ e $g:B\to A$. Se $g\bola f=I_A$ ent\ao $g=f^{-1}$.

\noindent \textit{\textbf{``Demonstra\caoi'':}} Sejam $f,g$ como acima e seja $x\in B$. Suponha que $y\in A$ \'e tal que $(x,y)\in g$. Seja $x\in B$ tal que $(y,z) \in f$. Como $g\bola f=I_A$, $(x,y)\in g$. Mas $(x,y)\in g$ e $g$ \'e uma fun\caoi, portanto $x=z$. Assim, $(x,y)\in f^{-1}$, logo $g\subseteq f^{-1}$. Agora suponha que $(x,y)\in f^{-1}$. Ent\ao $(y,x)\in f$. Como $g\bola f=I_A$, $(x,y)\in g$, logo $f^{-1}\subseteq g$ e portanto temos que $g=f^{-1}$.

\noindent \textit{\textbf{``contra-exemplo'':}} Sejam $A=\{1,2,3\}$, $B=\{1,2,3,4\}$ com
\[
f=\{(1,2),(2,1),(3,3)\}
\]
\[
g=\{(2,1),(1,2),(3,3),(4,3)\}
\]
Ent\ao $g\bola f=I_A$ mas $g\neq f^{-1}$, pois $(4,3)\in g$.

%excercicio17
\item {\bf Acredite se quiser:}  

\noindent \textit{\textbf{Conjectura:}} Seja $f:A\to B$. Se $f^{-1}\bola f=I_A$ ent\ao $f$ \'e injetora.

\noindent \textit{\textbf{``Demonstra\caoi'':}} Suponha que $f$ \'e como descrito acima e $x,y\in A$ com $f(x)=f(y)=z$. Ent\ao $f^{-1}(z)=x$ e $f^{-1}(z)=y$. Mas $f^{-1}$ \'e uma fun\caoi, portanto $x=y$ e assim $f$ \'e injetora.

\noindent \textit{\textbf{``contra-exemplo'':}} Sejam $A=\{a,b,c\}$ e $B=\{1,2,3\}$ com $f(a)=1$, $f(b)=2$ e $f(c)=2$. En\tao $f^{-1}=\{(1,a),(2,b),(2,c)\}$ mas $f$ n\ao \'e injetora.
\end{enumerate}
%%%%%%%%%%%%%%%%%%%%%%%%%%%%%%%%%%%%%%%%%%%%%%%%%%%%%%%%%%%%%%%%%%%%%%%%
%%%%%%%%%%%%%%%%%%%%%%%%% secao 2.7 %%%%%%%%%%%%%%%%%%%%%%%%%%%%%%%%%%%%
%%%%%%%%%%%%%%%%%%%%%%%%%%%%%%%%%%%%%%%%%%%%%%%%%%%%%%%%%%%%%%%%%%%%%%%%
\paragraph{Exerc\ih cios \ref{mfuncoes}}

\begin{enumerate}[{\bf 1.}]

%excercicio1
\item Sejam $A=\{1,2,3,4,5,6\}$, $B=\{2,3,4,5\}$ e $f:A\to B$ dada por $f(1)=f(4)=f(6)=3$; $f(2)=5$ e $f(3)=f(5)=4$. Encontre:
\begin{enumerate}[a)]
\item $f(\{1,2,3\})$, $f(A-\{2\})$ e $f(A)-\{2\}$. {\bf{\it Resposta:} ${\bf f(A-\{2\})=\{3,4\}}$.}
\item $f^{-1}(\{3\})$, $f^{-1}(\{4,5\})$ e $f^{-1}(\{2\})$.
\item $f(\{1,2\}\inter \{2,6\})$ e $f(\{1,2\})\inter f(\{2,6\})$. 
\end{enumerate}

%excercicio2
\item Seja $f:A\to B$. Mostre que:
\begin{enumerate}[a)]
\item $C\subseteq D\subseteq A$ implica $f(C)\subseteq f(D)$.
\item $C\subseteq A$ e $D \subseteq A$ implica que $f(C\uni D)=f(C)\uni f(D)$. 
\item $C\subseteq B$ e $D \subseteq B$ implica que $f^{-1}(C\uni D)=f^{-1}(C)\uni f^{-1}(D)$.
\item $C\subseteq B$ e $D \subseteq B$ implica que $f^{-1}(C\inter D)=f^{-1}(C)\inter f^{-1}(D)$.
\item $C\subseteq A$ implica $C\subseteq f^{-1}(f(C))$ e d\^e um exemplo para mostrar que a igualdade n\ao vale. 

{\bf{\it Resposta:} Resposta parcial - Seja ${\bf C=\{1\}}$. En\tao ${\bf f^{-1}(f(C))=\{-1,1\}}$.}

\item $C\subseteq B$ implica $f(f^{-1}(C))\subseteq C$ e d\^e um exemplo para mostrar que a igualdade n\ao vale. 
\end{enumerate}

%excercicio3
\item Seja $f:A\to B$. Para distinguir entre $f$ e a extens\ao de $f$ para subconjuntos de $A$, vamos definir $f^{\ast}$ a rela\cao de $\mathbb{P}(A)$ em $\mathbb{P}(B)$ por
\[
f^{\ast}=\{(C,f(C)): C\in \mathbb{P}(A)\},
\]
e $(f^{-1})^{\ast}$ a rela\cao de $\mathbb{P}(B)$ em $\mathbb{P}(A)$ por
\[
(f^{-1})^{\ast}=\{(C,f^{-1}(C)): C\in \mathbb{P}(B)\}.
\]
\begin{enumerate}[a)]
\item Mostre que $f^{\ast}:\mathbb{P}(A)\to\mathbb{P}(B)$.
\item Mostre que $(f^{-1})^{\ast}:\mathbb{P}(B)\to\mathbb{P}(A)$.
\item Mostre que $f$ injetora se e somente se $f^{\ast}$ injetora.

{\bf{\it Resposta:} Resposta parcial - Suponha que ${\bf f}$ seja injetora. Sejam ${\bf C,D\in\mathbb{P}(A)}$ com ${\bf f^{\ast}(C)=f^{\ast}(D)}$. Suponha $C\neq D$. Ent\aoi, existe ${\bf x\in C}$ tal que ${\bf x\notin D}$. Mas ${\bf f(x)\in f^{\ast}(C)=f^{\ast}(D)}$ assim dever\'a existir ${\bf y\neq x}$ pertencente ${\bf D}$ tal que ${\bf f(y)=f(x)}$. Mas isto contradiz a hip\'otese que ${\bf f}$ era injetora.}

\item Mostre que $f$ sobrejetora se e somente se $f^{\ast}$ sobrejetora.
\item Quando $(f^{-1})^{\ast}=(f^{\ast})^{-1}$?  
\end{enumerate}

%excercicio4
\item Seja $A$ um conjunto n\ao vazio e seja $F=\{f: f:A\to A\}$. Ent\ao $\bola$ (composi\cao de fun\caoi) \'e uma opera\cao bin\'aria em $F$. Para responder o que segue o leitor usar\'a alguns dos teoremas demonstrados anteriormente.
\begin{enumerate}[a)]
\item Mostre que $\bola$ \'e associativa. 
\item D\^e um exemplo para mostrar que $\bola$ n\ao \'e comutativa.
\item Mostre que $I_A$ \'e a identidade para $\bola$.
\item Quais elementos de $F$ t\^em inversa?
\item D\^e exemplos de fun\coes idempotentes. {\bf{\it Resposta:} ${\bf I_A}$ \'e idempotentes.}
\item Se $f$ \'e invert\ih vel e $f\bola g=f\bola h$, ent\ao $g=h$?
\item Mostre que se $f$ e $g$ s\ao invert\ih veis ent\ao $f\bola g$ \'e tamb\'em invert\ih vel. Qual \'e o inverso de $f\bola g$?
\end{enumerate}

Talvez agora os nomes identidade e inverso como usados com fun\coes assumem mais significado agora, para $I$ \'e a identidade de $\bola$ e $f^{-1}$ \'e o inverso de $f$.

%excercicio5
\item Seja $\bullet$ uma opera\cao bin\'aria em $A$. Mostre que:
\begin{enumerate}[a)]
\item Se $e$ \'e a identidade de $\bullet$ ent\ao $e$ \'e idempotente para $\bullet$.
\item Se $\bullet$ \'e associativa e comutativa e $a$ e $b$ s\ao ambos idempotentes para $\bullet$ ent\ao $a\bullet b$ \'e tamb\'em idempotente.
\item Se $\bullet$ \'e associativa e $x$ e $y$ s\ao invert\ih veis ent\ao $x\bullet y$ \'e tamb\'em idempotente. Expresse a inversa de $x\bullet y$ em termos das inversas de $x$ e $y$.

{\bf{\it Resposta:} Dica - Sejam ${\bf x^{\ast}}$ e ${\bf y^{\ast}}$ os inversos de ${\bf x}$ e ${\bf y}$ e compute ${\bf y^{\ast}\bullet x^{\ast}\bullet x\bullet y}$.}
\end{enumerate}

%excercicio6
\item Seja $A$ um conjunto n\ao vazio. Ent\ao $\uni$, $\inter$ e $-$ s\ao opera\coes bin\'arias em $\mathbb{P}(A)$. O leitor pode quere citar os teoremas demonstrados anteriormente e outros excerc\i cios para trabalhar nos seguintes itens:
\begin{enumerate}[a)]
\item Mostre que $\uni$ e $\inter$ s\ao associativas e comutativas.
\item D\^e exemplos para mostrar que $-$ n\ao \'e nem associativa nem comutativa.

{\bf{\it Resposta:} (Resposta Parical) Se ${\bf A=\{1,2\}}$ ent\ao ${\bf \{1\}-\{2\}\neq\{2\}-\{1\}}$.}

\item Mostre que cada elemento em $\mathbb{P}(A)$ \'e idempotente para $\uni$ e $\inter$.

{\bf{\it Resposta:} Como ${\bf B\uni B=B\inter B=B}$ para todos os conjuntos ${\bf B}$, assim todos os conjuntos s\ao idempotentes para ambos ${\bf \uni}$ e ${\bf \inter}$.}

\item Quais elementos s\ao idempotentes para $-$?
\item Quais elementos s\ao invert\ih veis para $\uni$, $\inter$ e $-$?
\end{enumerate}

%excercicio7
\item Seja $A$ um conjunto n\ao vazio. Definimos a opera\cao bin\'aria $\bullet$ em $\mathbb{P}(A)$ por
\[
X\bullet Y=(X-Y)\uni(Y-X).
\]
\begin{enumerate}[a)]
\item Mostre que $\bullet$ \'e comutativa. {\bf{\it Resposta:} Dica - Seria \'util utilizar ${\bf X\bullet Y = (X\inter Y^C)\uni(Y\inter X^C)}$.}
\item Mostre que $\bullet$ \'e associativa.
\item Qual a identidade para $\bullet$?
\item Mostre que cada elemento pertencente a $\mathbb{P}(A)$ \'e invert\ih vel.
\item Se $X\subseteq A$, qual \'e o inverso de $X$ para $\bullet$? 
\end{enumerate}

%excercicio8
\item Seja $F=\{f: f:\mathbb{R}\to\mathbb{R}, f(x)=ax+b, a\neq 0\}$. [$F$ \'e o conjunto de todas as fun\coes lineare n\ao constantes de $\mathbb{R}$ em $\mathbb{R}$.]
\begin{enumerate}[a)]
\item Mostre que $\bola$ (composi\cao de fun\cois) \'e uma opera\cao bin\'aria em $F$. {\bf{\it Resposta:} Dica - Mostre que se ${\bf f,g\in F}$ ent\ao ${\bf f\bola g\in F}$.}
\item Qual a identidade para $\bola$?
\item Quais elementos de $F$ s\ao invert\ih veis? {\bf{\it Resposta:} Dica - Pense nos gr\'aficos destas fun\cois.}
\item Se $f$ \'e invert\ih vel, qual a inversa de $f$?
\item Quais elementos de $F$ s\ao idempotentes? 
\end{enumerate}

%excercicio9
\item Suponha que $\bullet$ seja uma opera\cao bin\'aria associativa em $A$. Seja $x$ um elemento fixo pertencente a $A$. Definimos uma outra rela\cao bin\'aria $\bullet_x$ em $A$ por
\[
a\bullet_x b=a\bullet(x\bullet b).
\]
Mostre que $\bullet_x$ \'e associativa.

%excercicio10
\item Mostre que a rela\cao $R$ do teorema \ref{functeo21} \'e uma rela\cao de equival\^encia.

%excercicio11
\item Seja $\bullet$ uma opera\cao bin\'aria em $A$. Se $B\subseteq A$, podemos considerar que a restri\cao de $\bullet$ a $B$, $\bullet|_B$. Esta restri\cao pode ou n\ao ser uma opera\cao bin\'aria em $B$. Se $\bullet|_B$ for uma opera\cao bin\'aria em $B$, dizemos que $B$ \'e fechado com respeito a $\bullet$. 
\begin{enumerate}[a)]
\item D\^e uma defini\cao precisa da restri\cao mencionada acima.
\item Sejam $+,-$ as opera\coes alg\'ebricas usuais em $\mathbb{Z}$. Mostre que $\mathbb{N}$ \'e fechado com respeito a $+$, mas n\ao \'e fechado com respeiro a $-$.
\item Se $\bullet$ \'e uma opera\cao bin\'aria em $A$ com $B\subseteq A$, mostre que $B$ \'e fechado com respeito a $\bullet$ se e somente se 
\[
\{x\bullet y: x,y\in B\}\subseteq B.
\]
\end{enumerate}

%excercicio12
\item Seja $f:A\to B$. Mostre que $f$ pode ser decomposta em uma sobreje\caoi, uma bije\cao e uma inje\caoi, isto \'e, existem fun\coes $\alpha, \beta$ e $\gamma$ tais que $f=\gamma\bola\beta\bola\alpha$ onde $\alpha$ \'e uma sobrej\caoi, $\beta$ \'e uma bije\cao e $\gamma$ \'e uma inje\caoi. [Dica: veja o teorema \ref{functeo21}.]

%excercicio13
\item Em \'algebra com frequ\^encia usamos a regra ``igual adicionado a igual \'e igual'', ou mais precisamente, se $a,b,c,d\in\mathbb{R}$ com $a=b$ e $c=d$ ent\ao $a+c=b+d$. Prove que esta afirma\cao est\'a correta.

%excercicio14
\item {\bf Acredite se quiser:}  

\noindent \textit{\textbf{Conjectura:}} Seja $f:A\to B$ e $C,D\subseteq A$. Ent\ao $f(C\inter D)=f(C)\inter f(D)$.

\noindent \textit{\textbf{``Demonstra\caoi'':}} Sejam $C,D\in A$ e suponha $x\in f(C\inter D)$. Ent\ao existe $y\in C\inter D$ tal que $f(y)=x$. Claramente $y\in C$ e $y\in D$, assim $f(y)\in f(C)$ e $f(y)\in f(D)$, logo $x\in f(C)\inter f(D)$. Agora, suponha $x\in f(C)\inter f(D)$. Ent\ao existe $y\in C$ tal que $f(y)=x$ e existe $y\in D$ tal que $f(y)=x$. Mas $y\in C\inter D$, logo $x\in f(C\inter D)$. 

\noindent \textit{\textbf{``contra-exemplo'':}} Sejam $A=\{1,2\}$, $B=\{1,2,3\}$ e seja $f:A\to B$ dada por $f(1)=1$ e $f(2)=1$. Se $C\{a\}$ e $D=\{2\}$, ent\ao $f(C\inter D)\varnothing$ enquanto que $f(C)\inter f(D)=\{1\}$. 

%excercicio15
\item {\bf Acredite se quiser:}  

\noindent \textit{\textbf{Conjectura:}} Sejam $A$ e $B$ conjuntos com $f:A\to B$. Ent\ao $f^{-1}\bola f$ (estas s\ao as fun\coes induzidas por conjuntos) \'e uma rela\cao de equival\^encia em $\mathbb{P}(A)$.

\noindent \textit{\textbf{``Demonstra\caoi'':}} Sejam $A$, $B$ e $f$ como descritos acima e por conveni\^encia, vamos denotar a composi\cao de fun\coes induzidas por comjuntos, $f^{-1}\bola f$, por $R$. Seja $C\in \mathbb{P}(A)$. Como $f^{-1}(f(C))=C$, $(C,C)\in R$ e portanto $R$ \'e reflexiva. Se $C,D \in \mathbb{P}(A)$, com $(C,D)\in R$ ent\ao temos que $f^{-1}(f(C))=D$. Assim,
\[
f(D)=f^{-1}(f(C))=f^{-1}\bola f(C)=I_B(f(C))=f(C).
\]
Como $f(C)=f(D)$ ent\ao $f^{-1}(f(C))=f^{-1}(f(D))$ logo, $(D,C)\in R$ e portanto $R$ \'e sim\'etrica. Agora suponha que $(C,D)\in R$ e $(D,E)\in R$. Assim $f^{-1}(f(C))=D$ e $f^{-1}(f(D))=E$. Portanto,
\[
E=f^{-1}(f(D))= (f^{-1}\bola f)(f^{-1}\bola f(C))=f^{-1}\bola (f\bola f^{-1})\bola f(C)=f^{-1}\bola I_B(f(C))=f^{-1}(f(C))
\]
e assim $(C,E)\in R$ e portanto $R$ \'e transitiva, consequentemente uma rela\cao de equival\^encia.

\noindent \textit{\textbf{``contra-exemplo'':}} Sejam $A=\{1,2\}$, $B=\{1,2,3\}$ e $f:A\to B$ dada por $f(1)=1$ e $f(2)=1$. Ent\ao
\[
f^{-1}\bola f=\{(\varnothing,\varnothing),(\{1\},\{1,2\}), (\{1,2\},\{1,2\}), (\{2\},\{1,2\})\},
\]
que \'e sim\'etrica mas n\ao reflexiva.

%excercicio16
\item Seja $\mathcal{Q}=\{(m,n): m,n\in \mathbb{Z} \textrm{ com } n\neq 0\}$. Definimos uma rela\cao $R$ em $\mathcal{Q}$ por
\[
(m,n)R(x,y) \textrm { se e somente se } my=nx.
\]
\begin{enumerate}[a)]
\item Mostre que $R$ \'e ima rela\cao de equival\^encia.
\item Encontre tr\^es elementos de $[(1,2)]_R$ e tr\^es elementos de $[(1,-1)]_R$.

{\bf{\it Resposta:} (Resposta Parcial) ${\bf (1,2),(2,4)}$ e ${\bf (5,10)}$ est\ao todos em ${\bf [(1,2)]_R}$.}

\item Mostre que $\forall n\in\mathbb{Z}, n\neq 0$, $[(x,y)]_R=[(nx,ny)]_R$.

{\bf{\it Resposta:} (Resposta Parcial) Seja ${\bf n\in\mathbb{N}}$, ${\bf n\neq 0}$ e suponha que ${\bf (w,z)\in[(nx,ny)]_R}$ En\tao ${\bf wny=znx}$. Como ${\bf n\neq 0}$, podemos dividir por ${\bf n}$, obtendo ${\bf wy=zx}$ portanto ${\bf (w,z)R(x,y)}$ ou ${\bf [(nx,ny)]_R\subseteq[(x,y)]_R}$.}

\item Definimos a oera\cao bin\'aria $\star$ em $[\mathcal{Q}]_R$ por
\[
[(x,y)]_R\star [(m,n)]_R=[(xm,yn)]_R
\]
Mostre que esta opera\cao bin\'aria est\'a ``bem-definida'', isto \'e, se
\[
[(x,y)]_R=[(w,z)]_R \textrm{ e } [(m,n)]_R=[(p,q)]_R
\]
en\tao
\[
[(x,y)]_R\star[(m,n)]_R=[(w,z)]_R\star[(p,q)]_R.
\]
\item Podemos tentar definir outra opera\cao bin\'aria em $[\mathcal{Q}]_R$ por
\[
[(x,y)]_R\oplus[(w,z)]_R=[(x+w,y+z)]_R.
\]
Mostre, por exemplo, que esta ``opera\cao bin\'aria'' n\ao est\'a bem definida, e assim, n\ao \'e de fato uma opera\cao bin\'aria.
\item  Tentamos novamente defifinido
\[
[(x,y)]_R+[(w,z)]_R=[(xz,yw,yz)]_R.
\]
Mostre que esta opera\cao bin\'aria est\'a bem definida.
\end{enumerate}

\indent [Nota: O leitor alerta pode ter feito a identifica\cao de $\mathcal{Q}$ com $\mathbb{Q}$, o conjunto dos n\'umeros racionais, com $m,n$ fazendo o papel de $m/n$. De fato, o que pensamos ser o n\'umero $1/2$ \'e realmente uma classe de esquival\^encia e igauldade de n\'umeros racionais \'e igualdade de classe de equival\^encia. Por isso no ensino b\'asico aprendemos que $1/2=3/6$.]

%excercicio17
\item Seja $f:\mathbb{N}\to\mathbb{N}_5$ (veja exerc\ih cio \ref{equivalenciaex8} da se\cao \ref{equivalencia} para esta nota\caoi) dada por $f(x)=[x]_5$. Sejam $R$ e $\alpha$ como no teorema \ref{functeo21}.
\begin{enumerate}[a)]
\item Encontre $[2]_R$ e $[9]_R$.
\item Encontre $\alpha(4)$ e $\alpha(13)$.
\item Definimos a opera\cao bin\'aria $+$ em $[\mathbb{N}]_R$ por
\[
[x]_R+[y]_R=[x+y]_R.
\]
Mostre que $+$ \'e de fato uma opera\cao bin\'aria.
\item Mostre que $[5]_R$ \'e a identidade para $+$.
\item Mostre que $\forall x,y\in \mathbb{N}$, $\alpha(x+y)=\alpha(x)+\alpha(y)$.
\end{enumerate}
\end{enumerate}

%%%%%%%%%%%%%%%%%%%%%%%%%%%%%%%%%%%%%%%%%%%%%%%%%%%%%%%%%%%%%%%%%%%%%%%%
%%%%%%%%%%%%%%%%%%%%%%%%% secao 3.2 %%%%%%%%%%%%%%%%%%%%%%%%%%%%%%%%%%%%
%%%%%%%%%%%%%%%%%%%%%%%%%%%%%%%%%%%%%%%%%%%%%%%%%%%%%%%%%%%%%%%%%%%%%%%%
\paragraph{Exerc\ih cios \ref{inducao}}

\begin{enumerate}[{\bf 1.}]
%excercicio1
\item Demonstre as seguintes proposi\cois:
\begin{enumerate}[a)]
\item $\forall n\in\mathbb{N}, 1^2+2^2+3^2+\ldots+n^2=\frac{1}{6}n(n+1)(2n+1)$.

{\bf{\it Resposta:} Quando ${\bf n=1}$ temos ${\bf 1^2=\frac{1}{6}(1+1)(2+1)}$, que claramente é verdade. Agora suponha que ${\bf k\in\mathbb{N}}$ e ${\bf 1^2+2^2+3^2+\ldots+k^2=\frac{1}{6}k(k+1)(2k+1)}$, então
\begin{eqnarray*}
{\bf 1^2+2^2+3^2+\ldots+k^2+(k+1)^2}&=&{\bf  \left[\frac{1}{6}k(k+1)(2k+1)\right]+(k+1)^2}\\
                          &=&{\bf  \frac{1}{6}(k+1)\left[k(2k+1)+6k+6\right]}\\
                          &=&{\bf  \frac{1}{6}(k+1)\left(2k^2+k+6k+6\right)}\\
                          &=&{\bf  \frac{1}{6}(k+1)\left(2k^2+7k+6\right)}\\
                          &=&{\bf  \frac{1}{6}(k+1)(k+2)(2k+3)}\\
                          &=&{\bf  \frac{1}{6}(k+1)[(k+1)+1][(2(k+1)+1]},
\end{eqnarray*}  
logo \'e verdade para ${\bf k+1}$, que completa o passo de indu\cao e assim a demonstra\cao por indu\caoi.}

\item $\forall n\in\mathbb{N}, 1^3+2^3+3^3+\ldots+n^3=(\frac{1}{2}n(n+1))^2$.
\item $\forall n\in\mathbb{N}, 1+3+5+\ldots+(2n-1)=n^2$.

{\bf{\it Resposta:} Quando ${\bf n=1}$ temos ${\bf 1=1^2}$, que claramente é verdade. Agora suponha que ${\bf k\in\mathbb{N}}$ e ${\bf 1+3+5+\ldots+(2k-1)=k^2}$, então
\begin{eqnarray*}
{\bf 1+3+5+\ldots+(2k-1)+(2k+1)}&=&{\bf  k^2+2k+1}\\
                          &=&{\bf  (k+1)^2},
\end{eqnarray*}  
logo \'e verdade para ${\bf k+1}$, que completa o passo de indu\cao e assim a demonstra\cao por indu\caoi.}

\item $\forall n\in\mathbb{N}, 1+2^{-1}+2^{-2}+\ldots+2^{-n}\leq 2$.

{\bf{\it Resposta:} Quando ${\bf n=1}$ temos ${\bf 1+\frac{1}{2}\leq 2}$ que \'e verdade. Agora suponha ${\bf k\in\mathbb{N}}$ e ${\bf 1+2^{-1}+2^{-2}+\ldots+2^{-k}\leq 2}$. Multiplicando esta igualdade por ${\bf \frac{1}{2}}$ obtemos ${\bf 2^{-1}+2^{-2}+\ldots+2^{-(k+1)}\leq 1}$ e adicionando ${\bf 1}$ de cada lado vemos que a desigualdade vale para ${\bf k+1}$.}

\item $\forall n\in\mathbb{N}, n\geq 2, \forall x,y\in\mathbb{R}, x^n-y^n=(x-y)(x^{n-1}+x^{n-2}y+\ldots+xy^{n-2}+y^{n-1})$.
\item $\forall n\in\mathbb{N}, 2|n(n+1)$.

{\bf{\it Resposta:} Quando ${\bf n=1}$ temos ${\bf 2|(1\cdot 2)}$, que claramente é verdade. Agora suponha que ${\bf k\in\mathbb{N}}$ e ${\bf 2|k(k+1)}$, portanto ${\bf \exists m\in\mathbb{Z}}$ tal que ${\bf k(k+1)=2m}$, então
\begin{eqnarray*}
{\bf (k+1)(k+2)}&=&{\bf  k^2+3k+2}\\
                          &=&{\bf  (k^2+k)+(2k+2)}\\
                          &=&{\bf  2m+2k+2}\\
                          &=&{\bf  2(m+k+1)},
\end{eqnarray*}  
logo \'e verdade para ${\bf k+1}$, que completa o passo de indu\cao e assim a demonstra\cao por indu\caoi.}

\item $\forall n\in\mathbb{N}, 7|(3^{2n+1}+2^{n+2})$ [Dica: $9=7+2$].
\item $\forall n\in\mathbb{N}, 11|(8\cdot 10^{2n}+6\cdot 10^{2n-1}+9)$.

{\bf{\it Resposta:} Quando ${\bf n=1}$ temos ${\bf 11|(800+60+9)}$ ou ${\bf 11|869}$,  que claramente é verdade. Agora suponha que ${\bf k\in\mathbb{N}}$ e ${\bf 11|(8\cdot 10^{2k}+6\cdot 10^{2k-1}+9}$, portanto ${\bf \exists m\in\mathbb{Z}}$ tal que ${\bf 8\cdot 10^{2k}+6\cdot 10^{2k-1}+9=11m}$, então
\begin{eqnarray*}
{\bf 8\cdot 10^{2(k+1)}+6\cdot 10^{2(k+1)-1}+9}&=&{\bf  8\cdot 10^{2k+2}+6\cdot 10^{2k+1}+9}\\
                          &=&{\bf  100(8\cdot 10^{2k}+6\cdot 10^{2k-1}+9)-891}\\
                          &=&{\bf  100[11m-891]}\\
                          &=&{\bf  100[11m-11\cdot 81]}\\
                          &=&{\bf  11[100(m-81)]},
\end{eqnarray*}  
logo \'e verdade para ${\bf k+1}$, que completa o passo de indu\cao e assim a demonstra\cao por indu\caoi.}

\item $\forall n\in\mathbb{N}, D^n_x x^n=n!$.
\item $\forall n\in\mathbb{N}, 2^n>n$.
\item $\forall n\in\mathbb{N}, \forall a,b\in\mathbb{R}, a>b>0$ implica $a^n>b^n$.
\item $\forall n\in\mathbb{N}, n^n\geq n!$.

{\bf{\it Resposta:} Quando ${\bf n=1}$ temos ${\bf 1^1\geq 1!}$ que \'e verdade. Agora suponha ${\bf k\in\mathbb{N}}$ e ${\bf k^k\geq k!}$. Assim, ${\bf  (k+1)^{k+1}=(k+1)^{k}(k+1)\geq k^k(k+1)\geq k!(k+1)=(k+1)!},$ logo \'e verdade para ${\bf k+1}$, que completa o passo de indu\cao e assim a demonstra\cao por indu\caoi.}

\item $\forall n\in\mathbb{N}, 9|(2\cdot 10^n+3\cdot 10^{n-1}+4)$.
\item $\forall n\in\mathbb{N}, (1+1^{-1})(1+2^{-1})(1+3^{-1})\ldots(1+n^{-1})=n+1$.
\item $\forall n\in\mathbb{N}, 3+11+17+\ldots+(8n-5)=4n^2-n$.
\item $\forall n\in\mathbb{N}, 1+1/2^2+1/3^2+\ldots+1/n^2\leq 2-1/n$.
\item $\forall n\in\mathbb{N}, \forall a,b\in\mathbb{R}, a\geq 0, b\geq 0, a^n+b^n\geq(a+b/2)^n$.
\item $\forall n\in\mathbb{N}, \forall a\in\mathbb{R}, a\neq 1, 1+a+a^2+\ldots+a^n=(1-a^{n+1})/(1-a)$.

{\bf{\it Resposta:} Quando ${\bf n=1}$ temos ${\bf 1+a=\frac{1-a^2}{1-a}=\frac{(1-a)(1+a)}{1-a}}$, que claramente é verdade. Agora suponha que ${\bf k\in\mathbb{N}}$ e ${\bf 1+a+a^2+\ldots+a^k=(1-a^{k+1})/(1-a)}$, então
\begin{eqnarray*}
{\bf 1+a+a^2+\ldots+a^k+a^{k+1}}&=&{\bf  \frac{1-a^{k+1}}{1-a}+a^{k+1}}\\
                          &=&{\bf  \frac{1-a^{k+1}+a^{k+1}-a^{k+2}}{1-a}}\\
                          &=&{\bf  \frac{1-a^{k+2}}{1-a}},
\end{eqnarray*}  
logo \'e verdade para ${\bf k+1}$, que completa o passo de indu\cao e assim a demonstra\cao por indu\caoi.}

\item $\forall n\in\mathbb{N}, (1\cdot 3\cdot 5)+(3\cdot 5\cdot 7)+\ldots+[(2n-1)\cdot (2n+1)\cdot (2n+3)]=n(2n^3+8n^2+7n-2)$.
\item $\forall n\in\mathbb{N}, 1/(1\cdot 3)+1/(2\cdot 4)+\ldots1/[n\cdot(n+2)]=(3n^2+5n)/[4(n+1)(n+2)]$.
\item $\forall n\in\mathbb{N}, (1-\frac{1}{2})(1-\frac{1}{3})\ldots(1-\frac{1}{n})=\frac{1}{n}$.

{\bf{\it Resposta:} Quando ${\bf n=2}$ temos ${\bf 1-\frac{1}{2}=\frac{1}{2} }$, que claramente é verdade. Agora suponha que ${\bf k\in\mathbb{N}}$ e ${\bf (1-\frac{1}{2})(1-\frac{1}{3})\ldots(1-\frac{1}{k})=\frac{1}{k}}$, então
\begin{eqnarray*}
{\bf \left(1-\frac{1}{2}\right)\left(1-\frac{1}{3}\right)\ldots\left(1-\frac{1}{k}\right)\left(1-\frac{1}{k+1}\right)}&=&{\bf  \frac{1}{k}\left(1-\frac{1}{k+1}\right)}\\
                          &=&{\bf  \frac{1}{k}-\frac{1}{k(k+1)}}\\
                          &=&{\bf  \frac{(k+1)-1}{k(k+1)}}\\
                          &=&{\bf  \frac{k}{k(k+1)}}\\
                          &=&{\bf  \frac{1}{k+1}},
\end{eqnarray*}  
logo \'e verdade para ${\bf k+1}$, que completa o passo de indu\cao e assim a demonstra\cao por indu\caoi.}

\item $\forall n\in\mathbb{N}, (1-\frac{1}{2^2})(1-\frac{1}{3^2})\ldots(1-\frac{1}{n^2})=\frac{1}{2}(1+\frac{1}{n})$.
\end{enumerate}

%excercicio2
\item Mostre que para todos os n\'umeros naturais $n$, $n\geq 2$, existem inteiros n\ao negativos $a$ e $b$ tai que $n=2a+3b$.

%excercicio3
\item Encontre $n_0$ tal que $\forall n\in\mathbb{N}, n\geq n_0, n^2<(\frac{5}{4})^n$ e demonstre que o resultados est\'a correto.

%excercicio4
\item Suponha que a sequ\^encia de n\'umeros $(a_n)$ recursivamente como se segue: $a_1=1$ e para $n\geq 2$, seja $a_n=a_{n-1}+2\sqrt{a_{n-1}}+1$. Mostre que $\forall n\in\mathbb{N}, a_n$ \'e um inteiro.

{\bf{\it Resposta:} Dica - Compute primeiro alguns poucos valores ${\bf a_n}$ e tente demonstrar um resultado mais forte no qual o resultado desejado ser\'a uma consequencia.}

%excercicio5
\item Para $n\in\mathbb{N}$, seja $a_n=1+2^{-1}+3^{-1}+\ldots+n^{-1}$. Mostre que para cada $M\in\mathbb{N}$ existe um $n\in\mathbb{N}$ tal que $a_n>M$.

%excercicio6
\item {\bf Acredite se quiser:}   

\noindent \textit{\textbf{Conjectura:}} $\forall n\in\mathbb{N}$, $n\geq 783$, $3n^4+15n-7$ \'e par. {\bf{\it Resposta:} Falso.}

\noindent \textit{\textbf{``Demonstra\caoi'':}} Quando $n=783$, $3n^4+15n-7=1.127.634.377.502$, que \'e par. Agora suponha, $n\geq 783$ e que $3n^4+15n-7$ seja par, assim existe $m\in\mathbb{N}$ tal que $3n^4+15n-7=2m$. Ent\ao
\begin{eqnarray*}
3(n+1)^4+15(n+1)-7&=& 3(n^4+4n^3+6n^2+4n+1)+15n+15-7\\
                  &=& 3n^4+15n-7+12n^3+18n^2+12n+18 \\
                  &=& 2(m+6n^3+9n^2+6n+9),
\end{eqnarray*}  
que \'e par.   {\bf{\it Resposta:} Falso.}

\noindent \textit{\textbf{``contra-exemplo'':}} Quando $n=1000$, $3n^4+15n-7$ \'e \ih mpar, pois $3n^4+15n$ \'e claramente divis\ih vel por $1000$, assim quando o $7$ \'e subtra\ih do, o resultado ser\'a \ih mpar. {\bf{\it Resposta:} Verdadeiro.}
\end{enumerate}
%%%%%%%%%%%%%%%%%%%%%%%%%%%%%%%%%%%%%%%%%%%%%%%%%%%%%%%%%%%%%%%%%%%%%%%%
%%%%%%%%%%%%%%%%%%%%%%%%% secao 3.3 %%%%%%%%%%%%%%%%%%%%%%%%%%%%%%%%%%%%
%%%%%%%%%%%%%%%%%%%%%%%%%%%%%%%%%%%%%%%%%%%%%%%%%%%%%%%%%%%%%%%%%%%%%%%%
\paragraph{Exerc\ih cios \ref{eqvinducao}}

\begin{enumerate}[{\bf 1.}]

%excercicio1
\item Da demonstra\cao do teorema \ref{indteo1} definiu-se os conjuntos $T$ e $S^C$. Mostre que $T\subseteq S^C$.

%excercicio2
\item Mostre que $\mathbb{Z}$ n\ao tem o princ\ih pio da boa ordena\cao v\'alido, isto \'e, de um exemplo de um subconjunto n\ao vazio de $\mathbb{Z}$ que n\ao tenha um elemento m\ih nimo.

{\bf{\it Resposta:} Dica - Considere o conjunto ${\bf S=\{n\in\mathbb{Z}: n<0\}}$.}

%excercicio3
\item Use o princ\ih pio da boa ordena\cao para mostrar que $\sqrt{3}$ \'e irracional. Tente a mesma t\'ecnica usada na demonstra\cao do teorema \ref{indteo4}. Mostre onde esta t\'ecnica falharia se ela fosse utilizada para mostrar que $\sqrt{4}$ \'e irracional.

%excercicio4
\item Use o princ\ih pio da boa ordena\cao para mostrar que $\sqrt{17}$ \'e irracional.

%excercicio5
\item Demonstre os excerc\ih cio \ref{inducaoexce1a} e \ref{inducaoexce1a} da se\cao \ref{inducao} usando o princ\ih pio da boa ordena\caoi.

%excercicio6
\item Demonstre as seguintes proposi\cois usando qualquer m\'etodo de sua prefer\^encia:
\begin{enumerate}[a)]
\item $\forall n\in\mathbb{N}, 1^4+2^4+\ldots+n^4=\displaystyle\frac{n(n+1)(2n+1)(3n^2+3n-1)}{30}$.
\item $\forall n\in\mathbb{N}, 1^5+2^5+\ldots+n^5=\displaystyle\frac{n^2(n+1)^2(2n^2+2n-1)}{12}$.
\item $\forall n\in\mathbb{N}, 1^6+2^6+\ldots+n^6=\displaystyle\frac{n^7}{7}+\frac{n^6}{2}+\frac{n^5}{2}-\frac{n^3}{6}+\frac{n}{42}$.
\item $\forall n\in\mathbb{N}, 1\cdot 2+ 2\cdot 3+\ldots+n(n+1)=\displaystyle\frac{n(n+1)(n+2)}{3}$.
\item $\forall n\in\mathbb{N}, 2304|(7^{2n}-48n-1)$.
\item $\forall n\in\mathbb{N}, \displaystyle\frac{1}{\sqrt{1}}+\frac{1}{\sqrt{2}}+\ldots+\frac{1}{\sqrt{n}}\leq 2\sqrt{n}-1$.
\item $\forall n\in\mathbb{N}, \forall k\in\mathbb{N}, 1^k+2^k+\ldots+n^k\leq n^{k+1}$.
\end{enumerate}

%excercicio7
\item Sejam $\alpha,\beta$ as solu\coes da equa\cao $x^2-x-1=0$, com $\alpha >0$. Para todo $n\in\mathbb{N}$, seja $F_n=(\alpha^n-\beta^n)/(\alpha-\beta)$.
\begin{enumerate}[a)]
\item Encontre $F_1,F_3$ e $F_4$. [Nota: Estes n\'umeros s\ao conhecidos como os {\it n\'umeros de Fibonacci}.] 

{\bf{\it Resposta:} Dica - Note que ${\bf \alpha\beta=-1}$ e ${\bf \alpha+\beta=1}$.}

\item Mostre que $\forall n\in\mathbb{N}$, $F_{n+2}=F_{n+1}+F_n$. [Nota: Esta recorr\^encia ser\'a \'util para o restante deste excerc\ih cio.]  
\item Mostre que $\forall n\in\mathbb{N}$, $F_n$ \'e um inteiro.
\item Mostre que $\forall n\in\mathbb{N}$, $F_n<(\frac{13}{8})^n$.
\item Mostre que $\forall n\in\mathbb{N}$, $F^2_{n+1}-F_nF_{n+2}=(-1)^n$.
\item Mostre que $\forall n\in\mathbb{N}$, $2|F_{3n}$, $2\not|\espaco F_{3n+1}$ e $2\not|\espaco F_{3n+2}$.
\item Mostre que $\forall n\in\mathbb{N}$, 
\[
\sum_{i=1}^{n}F_i=F_{n+2}-1.
\]
\item Mostre que $\forall m,n\in\mathbb{N}$, $F_mF_n+F_{m+1}F_{n+1}=F_{m+n+1}$. 
\item Suponha que definimos $S_n=F_{1}^{2}+F_{2}^{2}+\ldots+F_{n}^{2}$. Encontre uma f\'ormula fechada para $S_n$ e demonstre que seu resultados est\'a correto.
\end{enumerate}

%excercicio8
\item Suponha que definimos a sequ\^encia de n\'umeros $(r_n)$ recursivamente como se segue: $r_1=1$, $r_2=1/4$ e para $n\geq 2$,
\[
r_{n+1}=\frac{r_nr_{n-1}}{r_n+r_{n-1}+2\sqrt{r_nr_{n-1}}}.
\]
Mostre que $\forall n\in\mathbb{N}$, $r_n=F^{-2}_{n+1}$. ($F_n$ do excerc\ih cio anterior).

{\bf{\it Resposta:} Dica - Use a recorr\^encia para ${\bf F_n}$ do exerc\ih cio ${\bf 7}$.}

%excercicio9
\item Mostre que para todo $n\in\mathbb{N}:$
\begin{enumerate}[a)]
\item $6400|(9^{2n}-80n-1)$.
\item $3|(4^n+2)$.
\item $13|(4^{2n+1}+3^{n+2})$.
\item $24|(16^n+9^{3n-2}-1)$.
\end{enumerate}

%excercicio10
\item Extenda o algoritmo da divis\ao (teorema \ref{indteo5}) para incluir o caso quando $a\leq 0$. Tamb\'em mostre que $q$ e $r$ s\ao \'unicos.

%excercicio11
\item Defina a sequ\^encia $(a_n)$ por $a_1=a_2=1$, e para $n\geq 3$, $a_n=4a_{n-1}+5a_{n-2}$. Mostre que para $n\geq 3$, $a_n=\frac{1}{15}5^n+\frac{2}{3}(-1)^{n+1}$.

%excercicio12
\item {\bf Acredite se quiser:}  

\noindent \textit{\textbf{Conjectura:}} $\forall n\in\mathbb{N}, n$ \'e um primo ou $\exists p,q\in\mathbb{Z} \mid  n=2^p3^q$. 

\noindent \textit{\textbf{``Demonstra\caoi'':}} Claramente a asserc\ao \'e verdadeira quando $n=1$, pois $1=2^03^0$. Agora, suponha que isto seja verdade quando $1,2,\ldots,k$. Se $k+1$ for primo, a demonstra\cao estar\'a terminada, potanto suponha que $k+1$ n\ao seja primo. Ent\ao $k+1=ab$, onde $1<a<k+1$ e $1<b<k+1$. Pela hip\'otese de indu\caoi, $a=2^p3^q$ e $b=2^r3^s$ para $p,q,r,s\in\mathbb{Z}$. Assim $k+1=2^{p+r}3^{q+s}$, que completa a demonstra\caoi.

\noindent \textit{\textbf{``contra-exemplo'':}} Considere $25$. $25$ n\ao \'e primo e como $2\not|\espaco 25$ e $3\not|\espaco 25$, $25\neq 2^p3^q$ para quaisquer inteiros $p$ e $q$.
\end{enumerate}

